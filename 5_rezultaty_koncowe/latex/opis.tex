% opis rozwiazywanego problemu  

\section{Opis projektu}
Projekt dotyczy komputerowej symulacji zachowania cieczy oraz wizualizacji jej stanu i~rozkładu ciśnienia w zbiorniku z płynem.

Symulacja obejmuje ruch cieczy w przekroju 2D wybranego naczynia. Ciecz jest przedstawiona na płaszczyźnie jako zbiór oddziaływujących ze sobą cząsteczek.
Jej zachowanie jest możliwie zbliżone do rzeczywistego dzięki
modelowaniu ruchu płynu przy pomocy metody numerycznej SPH (\textit{smoothed particle hydrodynamics - wygładzona hydrodynamika cząstek}).
W programie modelowane są właściwości fizyczne cieczy: gęstość i~lepkość. Dlatego też można badać zachowania płynów o różnych parametrach. 
Dodatkowo prowadzony jest pomiar ciśnienia cieczy. Ciśnienie wizualizowane jest jako odcień koloru płynu.
Im jest on ciemniejszy, tym wyższe ciśnienie odzwierciedla.

Aplikacja została napisana w języku \textsf{C++}, przy użyciu biblioteki \textsf{Qt}.