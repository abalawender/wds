\documentclass[a4paper]{article}
\usepackage[utf8]{inputenc}
\usepackage[polish]{babel}
\usepackage{polski}
%\usepackage{indentfirst}
%\usepackage{default}
%\usepackage{graphicx}
%\usepackage{inconsolata}
\usepackage{geometry}
\usepackage{parskip}
\setlength{\parskip}{1ex}
\usepackage[T1]{fontenc}
\usepackage{color}
\definecolor{bluekeywords}{rgb}{0.13,0.13,1}
\definecolor{greencomments}{rgb}{0,0.5,0}
\definecolor{redstrings}{rgb}{0.9,0,0}

\begin{document}

\begin{titlepage}
    \begin{center}
        \bfseries
        \huge Politechnika Wrocławska
        \vskip.2in
        \textsc{\LARGE Wydział Elektroniki}
        \vskip.2in
        \Large Wizualizacja Danych Sensorycznych
        \vskip1.5in
        \emph{\huge Wizualizacja rozkładu ciśnienia cieczy na podstawie symulacji komputerowej}
    \end{center}

    \vskip1.4in

    \begin{minipage}{.50\textwidth}
        \begin{flushleft}
            \bfseries\large Prowadzący:\par \emph{Dr inż. Bogdan Kreczmer}
        \end{flushleft}
    \end{minipage}
    %\hskip.4\textwidth
    \begin{minipage}{.45\textwidth}
        \begin{flushright}
            \bfseries\large Studenci:\par \emph{ Adam Balawender \\ Krzysztof Kwieciński }
        \end{flushright}
    \end{minipage}

    \vskip1.3in

    \centering
    \bfseries
	\Large Semestr letni 2014/2015


\end{titlepage}

\section{Wstępny opis}
W...

\section{Cele}
Symulacja zachowania cieczy jako zbioru reagujących ze sobą cząstek. Modelowanie właściwości fizycznych wody. Wizualizacja rozkładu ciśnień w zbiorniku.

\end{document}
