% opis rozwiazywanego problemu  

\section{Opis projektu}
Zgodnie z tematem projektu zajmiemy się komputerową symulacją zachowania cieczy oraz wizualizacją jej stanu i~rozkładu ciśnienia w zbiorniku z płynem.

Symulacja będzie obejmowała ruch cieczy w przekroju 2D wybranego naczynia. Ciecz zostanie przedstawiona na płaszczyźnie jako zbiór oddziaływujących ze sobą cząsteczek.
Postaramy się, żeby jej zachowanie było możliwie zbliżone do rzeczywistego.
Ruch płynu zostanie zamodelowany metodą numeryczną SPH (\textit{smoothed particle hydrodynamics - wygładzona hydrodynamika cząstek}).
Pozwoli to na realistyczne odwzorowanie zachowania cieczy.
Możliwe będzie badanie cieczy o różnych parametrach, dlatego też modelowane będą jej właściwości fizyczne: gęstość i~lepkość.
Dodatkowo mierzone będzie ciśnienie cieczy i zostanie ono zwizualizowane jako odcień koloru płynu.
Im będzie on ciemniejszy, tym wyższe ciśnienie będzie odzwierciedlał.

Aplikacja zostanie napisana w języku \textsf{C++}, przy użyciu biblioteki \textsf{Qt}.