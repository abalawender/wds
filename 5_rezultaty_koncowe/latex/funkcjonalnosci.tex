% opis funkcjonalnosci aplikacji  

\section{Funkcjonalności aplikacji}
Najistotniejsze funkcjonalności aplikacji to:
\begin{itemize}
    \item symulacja zachowania cieczy w zależności od zadanych warunków początkowych,
    \item symulacja zachowania cieczy przy poruszaniu zbiornikiem,
    \item symulacja zachowania cieczy na Ziemi lub w~stanie nieważkości,
    \item programowa możliwość przedefiniowania parametrów cieczy (gęstości, lepkości),
    \item zmiana szybkości symulacji,
    \item możliwość obserwacji wyniku symulacji (położenia cząsteczek i~rozkładu ciśnień).
\end{itemize}

Aplikacja umożliwia zasymulowanie zachowania cieczy zarówno od zadanych warunków początkowych, jak i~poprzez interakcję użytkownika z programem (poruszanie zbiornikiem za pomocą \textit{slidera}). Symulacja odzwierciedla zachowanie cząsteczek na Ziemi lub w~Kosmosie.  Programowo dostępna jest zmiana wszystkich założonych parametrów cząsteczki cieczy. Podczas działania aplikacji w~jej okienku obserwować można komputerowo zamodelowany ruch płynu wraz z rozkładem panujących w~nim ciśnień. Symulacja może być zatrzymywana i ponownie uruchamiana, a jej szybkość zmieniana. Rysunek \ref{fig:gui} przedstawia przykładowy zrzut ekranu działającej aplikacji.