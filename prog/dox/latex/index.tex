\begin{DoxyAuthor}{Autor}
Adam Balawender, 

Krzysztof Kwieciński, 

Ai\-R, A\-R\-R, W4 
\end{DoxyAuthor}
\begin{DoxyDate}{Data}
11.\-06.\-2015 
\end{DoxyDate}
\begin{DoxyVersion}{Wersja}
1.\-0
\end{DoxyVersion}
Aplikacja dotyczy komputerowej symulacji zachowania cieczy oraz wizualizacji jej stanu i rozkładu ciśnienia w zbiorniku z płynem.\hypertarget{index_etykieta-opis-projektu}{}\section{Opis projektu}\label{index_etykieta-opis-projektu}
Projekt dotyczy komputerowej symulacji zachowania cieczy oraz wizualizacji jej stanu i rozkładu ciśnienia w zbiorniku z płynem.

Symulacja obejmuje ruch cieczy w przekroju 2\-D wybranego naczynia. Ciecz jest przedstawiona na płaszczyźnie jako zbiór oddziaływujących ze sobą cząsteczek. Jej zachowanie jest możliwie zbliżone do rzeczywistego dzięki modelowaniu ruchu płynu przy pomocy metody numerycznej S\-P\-H (smoothed particle hydrodynamics -\/ wygładzona hydrodynamika cząstek). W programie modelowane są właściwości fizyczne cieczy\-: gęstość i lepkość. Dlatego też można badać zachowania płynów o różnych parametrach. Dodatkowo prowadzony jest pomiar ciśnienia cieczy. Ciśnienie wizualizowane jest jako odcień koloru płynu. Im jest on ciemniejszy, tym wyższe ciśnienie odzwierciedla.

Aplikacja została napisana w języku C++, przy użyciu biblioteki Qt.\hypertarget{index_etykieta-funkcjonalnosci-aplikacji}{}\section{Funkcjonalnosci aplikacji}\label{index_etykieta-funkcjonalnosci-aplikacji}
Najistotniejsze funkcjonalności aplikacji to\-:
\begin{DoxyItemize}
\item symulacja zachowania cieczy w zależności od zadanych warunków początkowych,
\item symulacja zachowania cieczy przy poruszaniu zbiornikiem,
\item symulacja zachowania cieczy na Ziemi lub w stanie nieważkości,
\item programowa możliwość przedefiniowania parametrów cieczy (gęstości, lepkości),
\item zmiana szybkości symulacji,
\item możliwość obserwacji wyniku symulacji (położenia cząsteczek i rozkładu ciśnień).
\end{DoxyItemize}

Aplikacja umożliwia zasymulowanie zachowania cieczy zarówno od zadanych warunków początkowych, jak i poprzez interakcję użytkownika z programem (poruszanie zbiornikiem za pomocą slidera). Symulacja odzwierciedla zachowanie cząsteczek na Ziemi lub w Kosmosie. Programowo dostępna jest zmiana wszystkich założonych parametrów cząsteczki cieczy. Podczas działania aplikacji w jej okienku obserwować można komputerowo zamodelowany ruch płynu wraz z rozkładem panujących w nim ciśnień. Symulacja może być zatrzymywana i ponownie uruchamiana, a jej szybkość zmieniana. Rysunek 2 przedstawia przykładowy zrzut ekranu działającej aplikacji. 