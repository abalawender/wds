\begin{DoxyAuthor}{Autor}
Adam Balawender, 

Krzysztof Kwieciński, 

Ai\+R, A\+R\+R, W4 
\end{DoxyAuthor}
\begin{DoxyDate}{Data}
11.\+05.\+2015 
\end{DoxyDate}
\begin{DoxyVersion}{Wersja}
0.\+1
\end{DoxyVersion}
Aplikacja dotyczy komputerowej symulacji zachowania cieczy oraz wizualizacji jej stanu i rozkładu ciśnienia w zbiorniku z płynem.\hypertarget{index_etykieta-opis-projektu}{}\section{Opis projektu}\label{index_etykieta-opis-projektu}
Symulacja będzie obejmowała ruch cieczy w przekroju 2\+D wybranego naczynia. Ciecz zostanie przedstawiona na płaszczyźnie jako zbiór oddziaływujących ze sobą cząsteczek. Postaramy się, żeby jej zachowanie było możliwie zbliżone do rzeczywistego. Ruch płynu zostanie zamodelowany metodą numeryczną S\+P\+H (particle hydrodynamics -\/ wygładzona hydrodynamika cząstek). Pozwoli to na realistyczne odwzorowanie zachowania cieczy. Możliwe będzie badanie cieczy o różnych parametrach, dlatego też modelowane będą jej właściwości fizyczne\+: gęstość i lepkość. Dodatkowo mierzone będzie ciśnienie cieczy i zostanie ono zwizualizowane jako odcień koloru płynu. Im będzie on ciemniejszy, tym wyższe ciśnienie będzie odzwierciedlał.\hypertarget{index_etykieta-funkcjonalnosci-aplikacji}{}\section{Funkcjonalnosci aplikacji}\label{index_etykieta-funkcjonalnosci-aplikacji}
Najistotniejszymi funkcjonalnościami aplikacji będą\+:
\begin{DoxyItemize}
\item symulacja zachowania cieczy w zależności od zadanych warunków początkowych,
\item item możliwość przedefiniowania parametrów cieczy (gęstości, lepkości),
\item możliwość obserwacji wyniku symulacji (położenia cząsteczek i rozkładu ciśnień). 
\end{DoxyItemize}