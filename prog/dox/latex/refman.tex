\documentclass[twoside]{book}

% Packages required by doxygen
\usepackage{calc}
\usepackage{doxygen}
\usepackage{graphicx}
\usepackage[utf8]{inputenc}
\usepackage{makeidx}
\usepackage{multicol}
\usepackage{multirow}
\usepackage{textcomp}
\usepackage[table]{xcolor}

% NLS support packages
\usepackage{polski}
\usepackage[T1]{fontenc}

% Font selection
\usepackage[T1]{fontenc}
\usepackage{mathptmx}
\usepackage[scaled=.90]{helvet}
\usepackage{courier}
\usepackage{amssymb}
\usepackage{sectsty}
\renewcommand{\familydefault}{\sfdefault}
\allsectionsfont{%
  \fontseries{bc}\selectfont%
  \color{darkgray}%
}
\renewcommand{\DoxyLabelFont}{%
  \fontseries{bc}\selectfont%
  \color{darkgray}%
}

% Page & text layout
\usepackage{geometry}
\geometry{%
  a4paper,%
  top=2.5cm,%
  bottom=2.5cm,%
  left=2.5cm,%
  right=2.5cm%
}
\tolerance=750
\hfuzz=15pt
\hbadness=750
\setlength{\emergencystretch}{15pt}
\setlength{\parindent}{0cm}
\setlength{\parskip}{0.2cm}
\makeatletter
\renewcommand{\paragraph}{%
  \@startsection{paragraph}{4}{0ex}{-1.0ex}{1.0ex}{%
    \normalfont\normalsize\bfseries\SS@parafont%
  }%
}
\renewcommand{\subparagraph}{%
  \@startsection{subparagraph}{5}{0ex}{-1.0ex}{1.0ex}{%
    \normalfont\normalsize\bfseries\SS@subparafont%
  }%
}
\makeatother

% Headers & footers
\usepackage{fancyhdr}
\pagestyle{fancyplain}
\fancyhead[LE]{\fancyplain{}{\bfseries\thepage}}
\fancyhead[CE]{\fancyplain{}{}}
\fancyhead[RE]{\fancyplain{}{\bfseries\leftmark}}
\fancyhead[LO]{\fancyplain{}{\bfseries\rightmark}}
\fancyhead[CO]{\fancyplain{}{}}
\fancyhead[RO]{\fancyplain{}{\bfseries\thepage}}
\fancyfoot[LE]{\fancyplain{}{}}
\fancyfoot[CE]{\fancyplain{}{}}
\fancyfoot[RE]{\fancyplain{}{\bfseries\scriptsize Wygenerowano Cz, 4 cze 2015 23\-:55\-:51 dla Symulacja\-Zbiornika programem Doxygen }}
\fancyfoot[LO]{\fancyplain{}{\bfseries\scriptsize Wygenerowano Cz, 4 cze 2015 23\-:55\-:51 dla Symulacja\-Zbiornika programem Doxygen }}
\fancyfoot[CO]{\fancyplain{}{}}
\fancyfoot[RO]{\fancyplain{}{}}
\renewcommand{\footrulewidth}{0.4pt}
\renewcommand{\chaptermark}[1]{%
  \markboth{#1}{}%
}
\renewcommand{\sectionmark}[1]{%
  \markright{\thesection\ #1}%
}

% Indices & bibliography
\usepackage{natbib}
\usepackage[titles]{tocloft}
\setcounter{tocdepth}{3}
\setcounter{secnumdepth}{5}
\makeindex

% Hyperlinks (required, but should be loaded last)
\usepackage{ifpdf}
\ifpdf
  \usepackage[pdftex,pagebackref=true]{hyperref}
\else
  \usepackage[ps2pdf,pagebackref=true]{hyperref}
\fi
\hypersetup{%
  colorlinks=true,%
  linkcolor=blue,%
  citecolor=blue,%
  unicode%
}

% Custom commands
\newcommand{\clearemptydoublepage}{%
  \newpage{\pagestyle{empty}\cleardoublepage}%
}


%===== C O N T E N T S =====

\begin{document}

% Titlepage & ToC
\hypersetup{pageanchor=false}
\pagenumbering{roman}
\begin{titlepage}
\vspace*{7cm}
\begin{center}%
{\Large Symulacja\-Zbiornika \\[1ex]\large 0.\-1 }\\
\vspace*{1cm}
{\large Wygenerowano przez Doxygen 1.8.6}\\
\vspace*{0.5cm}
{\small Cz, 4 cze 2015 23:55:51}\\
\end{center}
\end{titlepage}
\clearemptydoublepage
\tableofcontents
\clearemptydoublepage
\pagenumbering{arabic}
\hypersetup{pageanchor=true}

%--- Begin generated contents ---
\chapter{Wizualizacja rozkładu ciśnienia cieczy na podstawie symulacji komputerowej}
\label{index}\hypertarget{index}{}\begin{DoxyAuthor}{Autor}
Adam Balawender, 

Krzysztof Kwieciński, 

Ai\-R, A\-R\-R, W4 
\end{DoxyAuthor}
\begin{DoxyDate}{Data}
11.\-05.\-2015 
\end{DoxyDate}
\begin{DoxyVersion}{Wersja}
0.\-1
\end{DoxyVersion}
Aplikacja dotyczy komputerowej symulacji zachowania cieczy oraz wizualizacji jej stanu i rozkładu ciśnienia w zbiorniku z płynem.\hypertarget{index_etykieta-opis-projektu}{}\section{Opis projektu}\label{index_etykieta-opis-projektu}
Symulacja będzie obejmowała ruch cieczy w przekroju 2\-D wybranego naczynia. Ciecz zostanie przedstawiona na płaszczyźnie jako zbiór oddziaływujących ze sobą cząsteczek. Postaramy się, żeby jej zachowanie było możliwie zbliżone do rzeczywistego. Ruch płynu zostanie zamodelowany metodą numeryczną S\-P\-H (particle hydrodynamics -\/ wygładzona hydrodynamika cząstek). Pozwoli to na realistyczne odwzorowanie zachowania cieczy. Możliwe będzie badanie cieczy o różnych parametrach, dlatego też modelowane będą jej właściwości fizyczne\-: gęstość i lepkość. Dodatkowo mierzone będzie ciśnienie cieczy i zostanie ono zwizualizowane jako odcień koloru płynu. Im będzie on ciemniejszy, tym wyższe ciśnienie będzie odzwierciedlał.\hypertarget{index_etykieta-funkcjonalnosci-aplikacji}{}\section{Funkcjonalnosci aplikacji}\label{index_etykieta-funkcjonalnosci-aplikacji}
Najistotniejszymi funkcjonalnościami aplikacji będą\-:
\begin{DoxyItemize}
\item symulacja zachowania cieczy w zależności od zadanych warunków początkowych,
\item item możliwość przedefiniowania parametrów cieczy (gęstości, lepkości),
\item możliwość obserwacji wyniku symulacji (położenia cząsteczek i rozkładu ciśnień). 
\end{DoxyItemize}
\chapter{Indeks przestrzeni nazw}
\input{namespaces}
\chapter{Indeks hierarchiczny}
\section{Hierarchia klas}
Ta lista dziedziczenia posortowana jest z grubsza, choć nie całkowicie, alfabetycznie\-:\begin{DoxyCompactList}
\item Q\-Main\-Window\begin{DoxyCompactList}
\item \contentsline{section}{Okno\-Glowne}{\pageref{class_okno_glowne}}{}
\end{DoxyCompactList}
\item Q\-Widget\begin{DoxyCompactList}
\item \contentsline{section}{Kanwa}{\pageref{class_kanwa}}{}
\item \contentsline{section}{Zbiornik}{\pageref{class_zbiornik}}{}
\end{DoxyCompactList}
\item \contentsline{section}{Vector$<$ T\-Y\-P\-E $>$}{\pageref{class_vector}}{}
\end{DoxyCompactList}

\chapter{Indeks klas}
\section{Lista klas}
Tutaj znajdują się klasy, struktury, unie i interfejsy wraz z ich krótkimi opisami\-:\begin{DoxyCompactList}
\item\contentsline{section}{\hyperlink{class_czasteczka}{Czasteczka} \\*Klasa modelująca czasteczke }{\pageref{class_czasteczka}}{}
\item\contentsline{section}{\hyperlink{class_kolor}{Kolor} \\*Klasa modelująca kolor }{\pageref{class_kolor}}{}
\item\contentsline{section}{\hyperlink{class_okno_glowne}{Okno\-Glowne} \\*Klasa modelujaca głowne okno aplikacji }{\pageref{class_okno_glowne}}{}
\item\contentsline{section}{\hyperlink{class_vector}{Vector$<$ T\-Y\-P\-E $>$} }{\pageref{class_vector}}{}
\item\contentsline{section}{\hyperlink{class_zbiornik}{Zbiornik} \\*Klasa modelująca zbiornik }{\pageref{class_zbiornik}}{}
\end{DoxyCompactList}

\chapter{Indeks plików}
\section{Lista plików}
Tutaj znajduje się lista wszystkich plików z ich krótkimi opisami\+:\begin{DoxyCompactList}
\item\contentsline{section}{\hyperlink{czasteczka_8cpp}{czasteczka.\+cpp} \\*Zawiera definicje metod klasy \hyperlink{class_czasteczka}{Czasteczka} }{\pageref{czasteczka_8cpp}}{}
\item\contentsline{section}{\hyperlink{czasteczka_8hh}{czasteczka.\+hh} \\*Zawiera definicje klasy \hyperlink{class_czasteczka}{Czasteczka} oraz deklaracje jej metod }{\pageref{czasteczka_8hh}}{}
\item\contentsline{section}{\hyperlink{flagi_8hh}{flagi.\+hh} \\*Zawiera globalne zmienne opisujace symulacje }{\pageref{flagi_8hh}}{}
\item\contentsline{section}{\hyperlink{kolor_8hh}{kolor.\+hh} \\*Zawiera definicje klasy \hyperlink{class_kolor}{Kolor} oraz deklaracje jej metod }{\pageref{kolor_8hh}}{}
\item\contentsline{section}{\hyperlink{main_8cpp}{main.\+cpp} \\*Zawiera ogolna strukture funkcji main }{\pageref{main_8cpp}}{}
\item\contentsline{section}{\hyperlink{moc__okno__glowne_8cpp}{moc\+\_\+okno\+\_\+glowne.\+cpp} }{\pageref{moc__okno__glowne_8cpp}}{}
\item\contentsline{section}{\hyperlink{moc__zbiornik_8cpp}{moc\+\_\+zbiornik.\+cpp} }{\pageref{moc__zbiornik_8cpp}}{}
\item\contentsline{section}{\hyperlink{okno__glowne_8cpp}{okno\+\_\+glowne.\+cpp} \\*Zawiera definicje metod klasy \hyperlink{class_okno_glowne}{Okno\+Glowne} }{\pageref{okno__glowne_8cpp}}{}
\item\contentsline{section}{\hyperlink{okno__glowne_8hh}{okno\+\_\+glowne.\+hh} \\*Zawiera definicje klasy \hyperlink{class_okno_glowne}{Okno\+Glowne} i deklaracje jej metod }{\pageref{okno__glowne_8hh}}{}
\item\contentsline{section}{\hyperlink{simulation_8cpp}{simulation.\+cpp} \\*Plik z kodem symulatora cieczy }{\pageref{simulation_8cpp}}{}
\item\contentsline{section}{\hyperlink{simulation_8hh}{simulation.\+hh} \\*Plik z definicją klasy symulatora cieczy }{\pageref{simulation_8hh}}{}
\item\contentsline{section}{\hyperlink{ui__dmainwindow_8h}{ui\+\_\+dmainwindow.\+h} }{\pageref{ui__dmainwindow_8h}}{}
\item\contentsline{section}{\hyperlink{vector_8cpp}{vector.\+cpp} }{\pageref{vector_8cpp}}{}
\item\contentsline{section}{\hyperlink{vector_8hh}{vector.\+hh} }{\pageref{vector_8hh}}{}
\item\contentsline{section}{\hyperlink{zbiornik_8cpp}{zbiornik.\+cpp} \\*Zawiera definicje metod klasy \hyperlink{class_zbiornik}{Zbiornik} }{\pageref{zbiornik_8cpp}}{}
\item\contentsline{section}{\hyperlink{zbiornik_8hh}{zbiornik.\+hh} \\*Zawiera definicję klasy \hyperlink{class_zbiornik}{Zbiornik} i deklaracje jej metod }{\pageref{zbiornik_8hh}}{}
\end{DoxyCompactList}

\chapter{Dokumentacja przestrzeni nazw}
\hypertarget{namespace_ui}{}\section{Dokumentacja przestrzeni nazw Ui}
\label{namespace_ui}\index{Ui@{Ui}}
\subsection*{Komponenty}
\begin{DoxyCompactItemize}
\item 
class \hyperlink{class_ui_1_1_d_main_window}{D\+Main\+Window}
\end{DoxyCompactItemize}

\chapter{Dokumentacja klas}
\hypertarget{class_czasteczka}{}\section{Dokumentacja klasy Czasteczka}
\label{class_czasteczka}\index{Czasteczka@{Czasteczka}}


Klasa modelująca czasteczke.  




{\ttfamily \#include $<$czasteczka.\+hh$>$}



Diagram współpracy dla Czasteczka\+:\nopagebreak
\begin{figure}[H]
\begin{center}
\leavevmode
\includegraphics[width=186pt]{class_czasteczka__coll__graph}
\end{center}
\end{figure}
\subsection*{Metody publiczne}
\begin{DoxyCompactItemize}
\item 
\hyperlink{class_czasteczka_a1e73beeb1253fbb91788eabd65bcb0bf}{Czasteczka} (\hyperlink{class_vector}{Vector} \hyperlink{class_czasteczka_a2568977045d9bfe26054029c7657ad17}{xy}, int r, const \hyperlink{class_kolor}{Kolor} \&rgb)
\begin{DoxyCompactList}\small\item\em Konstruktor. \end{DoxyCompactList}\item 
void \hyperlink{class_czasteczka_a3d143798d7b407fe9dff7b61ab2b8071}{Rysuj\+Czasteczke} (Q\+Painter \&Rysownik, const int \hyperlink{class_czasteczka_a1b550be0425165a213fa1cec985b5478}{Promien}, const \hyperlink{class_kolor}{Kolor} \hyperlink{class_czasteczka_a546104013fe302440214f8809d7ec602}{R\+G\+B}, const double x, const double y)
\begin{DoxyCompactList}\small\item\em Metoda rysujaca czasteczke. \end{DoxyCompactList}\item 
\hyperlink{class_vector}{Vector} \hyperlink{class_czasteczka_a2568977045d9bfe26054029c7657ad17}{xy} () const 
\begin{DoxyCompactList}\small\item\em Interfejs pozwalajacy na odczyt prywatnych danych. \end{DoxyCompactList}\item 
\hyperlink{class_vector}{Vector} \& \hyperlink{class_czasteczka_a103b1c45bfeb9d5fec4acf74cd9d53f0}{xy} ()
\begin{DoxyCompactList}\small\item\em Interfejs pozwalajacy na zmiane prywatnych danych. \end{DoxyCompactList}\item 
int \hyperlink{class_czasteczka_a1b550be0425165a213fa1cec985b5478}{Promien} () const 
\begin{DoxyCompactList}\small\item\em Interfejs pozwalajacy na odczyt prywatnych danych. \end{DoxyCompactList}\item 
int \& \hyperlink{class_czasteczka_abd36efdd1f58cdf8a6a8379bfe663f3b}{Promien} ()
\begin{DoxyCompactList}\small\item\em Interfejs pozwalajacy na zmiane prywatnych danych. \end{DoxyCompactList}\item 
\hyperlink{class_kolor}{Kolor} \hyperlink{class_czasteczka_a546104013fe302440214f8809d7ec602}{R\+G\+B} () const 
\begin{DoxyCompactList}\small\item\em Interfejs pozwalajacy na odczyt prywatnych danych. \end{DoxyCompactList}\item 
\hyperlink{class_kolor}{Kolor} \& \hyperlink{class_czasteczka_aa70b19b0f59c5e4b244a9ff203d6de41}{R\+G\+B} ()
\begin{DoxyCompactList}\small\item\em Interfejs pozwalajacy na zmiane prywatnych danych. \end{DoxyCompactList}\end{DoxyCompactItemize}
\subsection*{Atrybuty prywatne}
\begin{DoxyCompactItemize}
\item 
\hyperlink{class_vector}{Vector} \hyperlink{class_czasteczka_a025a3ee895f8ee9c765814cfca1fd5e1}{\+\_\+xy}
\begin{DoxyCompactList}\small\item\em Atrybut opisujacy polozenie czasteczki. \end{DoxyCompactList}\item 
int \hyperlink{class_czasteczka_a5a1d126d89bd571c79a5691c45e2f469}{\+\_\+\+Promien}
\begin{DoxyCompactList}\small\item\em Atrybut okreslajacy promien czasteczki. \end{DoxyCompactList}\item 
\hyperlink{class_kolor}{Kolor} \hyperlink{class_czasteczka_ab9c93cfb3cf0360579ad0def2a94178c}{\+\_\+\+R\+G\+B}
\begin{DoxyCompactList}\small\item\em Atrybut opisujacy kolor czasteczki. \end{DoxyCompactList}\end{DoxyCompactItemize}


\subsection{Opis szczegółowy}
Klasa zawierajaca atrybuty o metody czasteczki. 

Definicja w linii 30 pliku czasteczka.\+hh.



\subsection{Dokumentacja konstruktora i destruktora}
\hypertarget{class_czasteczka_a1e73beeb1253fbb91788eabd65bcb0bf}{}\index{Czasteczka@{Czasteczka}!Czasteczka@{Czasteczka}}
\index{Czasteczka@{Czasteczka}!Czasteczka@{Czasteczka}}
\subsubsection[{Czasteczka}]{\setlength{\rightskip}{0pt plus 5cm}Czasteczka\+::\+Czasteczka (
\begin{DoxyParamCaption}
\item[{{\bf Vector}}]{xy, }
\item[{int}]{r, }
\item[{const {\bf Kolor} \&}]{rgb}
\end{DoxyParamCaption}
)\hspace{0.3cm}{\ttfamily [inline]}}\label{class_czasteczka_a1e73beeb1253fbb91788eabd65bcb0bf}
Konstruktor parametryczny, inicjalizujacy czasteczke podanymi wlasnosciami. 
\begin{DoxyParams}[1]{Parametry}
\mbox{\tt in}  & {\em xy} & -\/ wektor polozenia czasteczki \\
\hline
\mbox{\tt in}  & {\em r} & -\/ promien czasteczki \\
\hline
\mbox{\tt in}  & {\em rgb} & -\/ kolor czasteczki \\
\hline
\end{DoxyParams}


Definicja w linii 41 pliku czasteczka.\+hh.



\subsection{Dokumentacja funkcji składowych}
\hypertarget{class_czasteczka_a1b550be0425165a213fa1cec985b5478}{}\index{Czasteczka@{Czasteczka}!Promien@{Promien}}
\index{Promien@{Promien}!Czasteczka@{Czasteczka}}
\subsubsection[{Promien}]{\setlength{\rightskip}{0pt plus 5cm}int Czasteczka\+::\+Promien (
\begin{DoxyParamCaption}
{}
\end{DoxyParamCaption}
) const\hspace{0.3cm}{\ttfamily [inline]}}\label{class_czasteczka_a1b550be0425165a213fa1cec985b5478}
Interfejs pozwalajacy na odczyt prywatnych danych. \begin{DoxyReturn}{Zwraca}
\+\_\+\+Promien -\/ prywatny atrybut opisujacy promien czasteczki 
\end{DoxyReturn}


Definicja w linii 82 pliku czasteczka.\+hh.

\hypertarget{class_czasteczka_abd36efdd1f58cdf8a6a8379bfe663f3b}{}\index{Czasteczka@{Czasteczka}!Promien@{Promien}}
\index{Promien@{Promien}!Czasteczka@{Czasteczka}}
\subsubsection[{Promien}]{\setlength{\rightskip}{0pt plus 5cm}int\& Czasteczka\+::\+Promien (
\begin{DoxyParamCaption}
{}
\end{DoxyParamCaption}
)\hspace{0.3cm}{\ttfamily [inline]}}\label{class_czasteczka_abd36efdd1f58cdf8a6a8379bfe663f3b}
Interfejs pozwalajacy na zmiane prywatnych danych. \begin{DoxyReturn}{Zwraca}
\+\_\+\+Promien -\/ referencja na prywatny atrybut opisujacy promien czasteczki 
\end{DoxyReturn}


Definicja w linii 89 pliku czasteczka.\+hh.

\hypertarget{class_czasteczka_a546104013fe302440214f8809d7ec602}{}\index{Czasteczka@{Czasteczka}!R\+G\+B@{R\+G\+B}}
\index{R\+G\+B@{R\+G\+B}!Czasteczka@{Czasteczka}}
\subsubsection[{R\+G\+B}]{\setlength{\rightskip}{0pt plus 5cm}{\bf Kolor} Czasteczka\+::\+R\+G\+B (
\begin{DoxyParamCaption}
{}
\end{DoxyParamCaption}
) const\hspace{0.3cm}{\ttfamily [inline]}}\label{class_czasteczka_a546104013fe302440214f8809d7ec602}
Interfejs pozwalajacy na odczyt prywatnych danych. \begin{DoxyReturn}{Zwraca}
\+\_\+\+R\+G\+B -\/ prywatny atrybut opisujacy polozenie kolor czasteczki 
\end{DoxyReturn}


Definicja w linii 97 pliku czasteczka.\+hh.

\hypertarget{class_czasteczka_aa70b19b0f59c5e4b244a9ff203d6de41}{}\index{Czasteczka@{Czasteczka}!R\+G\+B@{R\+G\+B}}
\index{R\+G\+B@{R\+G\+B}!Czasteczka@{Czasteczka}}
\subsubsection[{R\+G\+B}]{\setlength{\rightskip}{0pt plus 5cm}{\bf Kolor}\& Czasteczka\+::\+R\+G\+B (
\begin{DoxyParamCaption}
{}
\end{DoxyParamCaption}
)\hspace{0.3cm}{\ttfamily [inline]}}\label{class_czasteczka_aa70b19b0f59c5e4b244a9ff203d6de41}
Interfejs pozwalajacy na zmiane prywatnych danych. \begin{DoxyReturn}{Zwraca}
\+\_\+\+R\+G\+B -\/ referencja na prywatny atrybut opisujacy promien czasteczki 
\end{DoxyReturn}


Definicja w linii 104 pliku czasteczka.\+hh.

\hypertarget{class_czasteczka_a3d143798d7b407fe9dff7b61ab2b8071}{}\index{Czasteczka@{Czasteczka}!Rysuj\+Czasteczke@{Rysuj\+Czasteczke}}
\index{Rysuj\+Czasteczke@{Rysuj\+Czasteczke}!Czasteczka@{Czasteczka}}
\subsubsection[{Rysuj\+Czasteczke}]{\setlength{\rightskip}{0pt plus 5cm}void Czasteczka\+::\+Rysuj\+Czasteczke (
\begin{DoxyParamCaption}
\item[{Q\+Painter \&}]{Rysownik, }
\item[{const int}]{Promien, }
\item[{const {\bf Kolor}}]{R\+G\+B, }
\item[{const double}]{x, }
\item[{const double}]{y}
\end{DoxyParamCaption}
)}\label{class_czasteczka_a3d143798d7b407fe9dff7b61ab2b8071}
Rysuje czasteczke o zadanych parametrach 
\begin{DoxyParams}[1]{Parametry}
\mbox{\tt in,out}  & {\em Rysownik} & -\/ referencja na obiekt klasy Q\+Painter \\
\hline
\mbox{\tt in}  & {\em Promien} & -\/ promien czasteczki \\
\hline
\mbox{\tt in}  & {\em R\+G\+B} & -\/ kolor czasteczki w formacie R\+G\+B \\
\hline
\mbox{\tt in}  & {\em x} & -\/ polozenie czasteczki na osi x \\
\hline
\mbox{\tt in}  & {\em y} & -\/ polozenie czasteczki na osi y \\
\hline
\end{DoxyParams}


Definicja w linii 16 pliku czasteczka.\+cpp.



Oto graf wywołań dla tej funkcji\+:\nopagebreak
\begin{figure}[H]
\begin{center}
\leavevmode
\includegraphics[width=324pt]{class_czasteczka_a3d143798d7b407fe9dff7b61ab2b8071_cgraph}
\end{center}
\end{figure}


\hypertarget{class_czasteczka_a2568977045d9bfe26054029c7657ad17}{}\index{Czasteczka@{Czasteczka}!xy@{xy}}
\index{xy@{xy}!Czasteczka@{Czasteczka}}
\subsubsection[{xy}]{\setlength{\rightskip}{0pt plus 5cm}{\bf Vector} Czasteczka\+::xy (
\begin{DoxyParamCaption}
{}
\end{DoxyParamCaption}
) const\hspace{0.3cm}{\ttfamily [inline]}}\label{class_czasteczka_a2568977045d9bfe26054029c7657ad17}
Interfejs pozwalajacy na odczyt prywatnych danych. \begin{DoxyReturn}{Zwraca}
\+\_\+xy -\/ prywatny atrybut opisujacy polozenie czasteczki 
\end{DoxyReturn}


Definicja w linii 67 pliku czasteczka.\+hh.

\hypertarget{class_czasteczka_a103b1c45bfeb9d5fec4acf74cd9d53f0}{}\index{Czasteczka@{Czasteczka}!xy@{xy}}
\index{xy@{xy}!Czasteczka@{Czasteczka}}
\subsubsection[{xy}]{\setlength{\rightskip}{0pt plus 5cm}{\bf Vector}\& Czasteczka\+::xy (
\begin{DoxyParamCaption}
{}
\end{DoxyParamCaption}
)\hspace{0.3cm}{\ttfamily [inline]}}\label{class_czasteczka_a103b1c45bfeb9d5fec4acf74cd9d53f0}
Interfejs pozwalajacy na zmiane prywatnych danych. \begin{DoxyReturn}{Zwraca}
\+\_\+xy -\/ referencja na prywatny atrybut opisujacy polozenie czasteczki 
\end{DoxyReturn}


Definicja w linii 74 pliku czasteczka.\+hh.



\subsection{Dokumentacja atrybutów składowych}
\hypertarget{class_czasteczka_a5a1d126d89bd571c79a5691c45e2f469}{}\index{Czasteczka@{Czasteczka}!\+\_\+\+Promien@{\+\_\+\+Promien}}
\index{\+\_\+\+Promien@{\+\_\+\+Promien}!Czasteczka@{Czasteczka}}
\subsubsection[{\+\_\+\+Promien}]{\setlength{\rightskip}{0pt plus 5cm}int Czasteczka\+::\+\_\+\+Promien\hspace{0.3cm}{\ttfamily [private]}}\label{class_czasteczka_a5a1d126d89bd571c79a5691c45e2f469}
Atrybut okreslajacy promien czasteczki. 

Definicja w linii 119 pliku czasteczka.\+hh.

\hypertarget{class_czasteczka_ab9c93cfb3cf0360579ad0def2a94178c}{}\index{Czasteczka@{Czasteczka}!\+\_\+\+R\+G\+B@{\+\_\+\+R\+G\+B}}
\index{\+\_\+\+R\+G\+B@{\+\_\+\+R\+G\+B}!Czasteczka@{Czasteczka}}
\subsubsection[{\+\_\+\+R\+G\+B}]{\setlength{\rightskip}{0pt plus 5cm}{\bf Kolor} Czasteczka\+::\+\_\+\+R\+G\+B\hspace{0.3cm}{\ttfamily [private]}}\label{class_czasteczka_ab9c93cfb3cf0360579ad0def2a94178c}
Atrybut opisujacy kolor czasteczki. 

Definicja w linii 126 pliku czasteczka.\+hh.

\hypertarget{class_czasteczka_a025a3ee895f8ee9c765814cfca1fd5e1}{}\index{Czasteczka@{Czasteczka}!\+\_\+xy@{\+\_\+xy}}
\index{\+\_\+xy@{\+\_\+xy}!Czasteczka@{Czasteczka}}
\subsubsection[{\+\_\+xy}]{\setlength{\rightskip}{0pt plus 5cm}{\bf Vector} Czasteczka\+::\+\_\+xy\hspace{0.3cm}{\ttfamily [private]}}\label{class_czasteczka_a025a3ee895f8ee9c765814cfca1fd5e1}
Atrybut opisujacy polozenie czasteczki w kartezjanskim ukladzie wspolrzednych. 

Definicja w linii 104 pliku czasteczka.\+hh.



Dokumentacja dla tej klasy została wygenerowana z plików\+:\begin{DoxyCompactItemize}
\item 
\hyperlink{czasteczka_8hh}{czasteczka.\+hh}\item 
\hyperlink{czasteczka_8cpp}{czasteczka.\+cpp}\end{DoxyCompactItemize}

\hypertarget{class_d_main_window}{\section{Dokumentacja klasy D\-Main\-Window}
\label{class_d_main_window}\index{D\-Main\-Window@{D\-Main\-Window}}
}


{\ttfamily \#include $<$dmainwindow.\-h$>$}



Diagram dziedziczenia dla D\-Main\-Window
\nopagebreak
\begin{figure}[H]
\begin{center}
\leavevmode
\includegraphics[width=160pt]{class_d_main_window__inherit__graph}
\end{center}
\end{figure}


Diagram współpracy dla D\-Main\-Window\-:
\nopagebreak
\begin{figure}[H]
\begin{center}
\leavevmode
\includegraphics[width=274pt]{class_d_main_window__coll__graph}
\end{center}
\end{figure}
\subsection*{Metody publiczne}
\begin{DoxyCompactItemize}
\item 
\hyperlink{class_d_main_window_a5eb52e50cb0f15345279e161eed68389}{D\-Main\-Window} (Q\-Widget $\ast$parent=0)
\item 
\hyperlink{class_d_main_window_acca34964f553e9e18371434ad8dc171c}{$\sim$\-D\-Main\-Window} ()
\end{DoxyCompactItemize}
\subsection*{Atrybuty prywatne}
\begin{DoxyCompactItemize}
\item 
\hyperlink{class_ui_1_1_d_main_window}{Ui\-::\-D\-Main\-Window} $\ast$ \hyperlink{class_d_main_window_a371b4fb59a61b0521beb17a8681366b9}{ui}
\end{DoxyCompactItemize}


\subsection{Opis szczegółowy}


Definicja w linii 11 pliku dmainwindow.\-h.



\subsection{Dokumentacja konstruktora i destruktora}
\hypertarget{class_d_main_window_a5eb52e50cb0f15345279e161eed68389}{\index{D\-Main\-Window@{D\-Main\-Window}!D\-Main\-Window@{D\-Main\-Window}}
\index{D\-Main\-Window@{D\-Main\-Window}!DMainWindow@{D\-Main\-Window}}
\subsubsection[{D\-Main\-Window}]{\setlength{\rightskip}{0pt plus 5cm}D\-Main\-Window\-::\-D\-Main\-Window (
\begin{DoxyParamCaption}
\item[{Q\-Widget $\ast$}]{parent = {\ttfamily 0}}
\end{DoxyParamCaption}
)\hspace{0.3cm}{\ttfamily [explicit]}}}\label{class_d_main_window_a5eb52e50cb0f15345279e161eed68389}


Definicja w linii 4 pliku dmainwindow.\-cpp.



Oto graf wywołań dla tej funkcji\-:
\nopagebreak
\begin{figure}[H]
\begin{center}
\leavevmode
\includegraphics[width=350pt]{class_d_main_window_a5eb52e50cb0f15345279e161eed68389_cgraph}
\end{center}
\end{figure}


\hypertarget{class_d_main_window_acca34964f553e9e18371434ad8dc171c}{\index{D\-Main\-Window@{D\-Main\-Window}!$\sim$\-D\-Main\-Window@{$\sim$\-D\-Main\-Window}}
\index{$\sim$\-D\-Main\-Window@{$\sim$\-D\-Main\-Window}!DMainWindow@{D\-Main\-Window}}
\subsubsection[{$\sim$\-D\-Main\-Window}]{\setlength{\rightskip}{0pt plus 5cm}D\-Main\-Window\-::$\sim$\-D\-Main\-Window (
\begin{DoxyParamCaption}
{}
\end{DoxyParamCaption}
)}}\label{class_d_main_window_acca34964f553e9e18371434ad8dc171c}


Definicja w linii 11 pliku dmainwindow.\-cpp.



\subsection{Dokumentacja atrybutów składowych}
\hypertarget{class_d_main_window_a371b4fb59a61b0521beb17a8681366b9}{\index{D\-Main\-Window@{D\-Main\-Window}!ui@{ui}}
\index{ui@{ui}!DMainWindow@{D\-Main\-Window}}
\subsubsection[{ui}]{\setlength{\rightskip}{0pt plus 5cm}{\bf Ui\-::\-D\-Main\-Window}$\ast$ D\-Main\-Window\-::ui\hspace{0.3cm}{\ttfamily [private]}}}\label{class_d_main_window_a371b4fb59a61b0521beb17a8681366b9}


Definicja w linii 20 pliku dmainwindow.\-h.



Dokumentacja dla tej klasy została wygenerowana z plików\-:\begin{DoxyCompactItemize}
\item 
\hyperlink{dmainwindow_8h}{dmainwindow.\-h}\item 
\hyperlink{dmainwindow_8cpp}{dmainwindow.\-cpp}\end{DoxyCompactItemize}

\hypertarget{class_ui_1_1_d_main_window}{\section{Dokumentacja klasy Ui\-:\-:D\-Main\-Window}
\label{class_ui_1_1_d_main_window}\index{Ui\-::\-D\-Main\-Window@{Ui\-::\-D\-Main\-Window}}
}


{\ttfamily \#include $<$ui\-\_\-dmainwindow.\-h$>$}



Diagram dziedziczenia dla Ui\-:\-:D\-Main\-Window\nopagebreak
\begin{figure}[H]
\begin{center}
\leavevmode
\includegraphics[width=176pt]{class_ui_1_1_d_main_window__inherit__graph}
\end{center}
\end{figure}


Diagram współpracy dla Ui\-:\-:D\-Main\-Window\-:\nopagebreak
\begin{figure}[H]
\begin{center}
\leavevmode
\includegraphics[width=176pt]{class_ui_1_1_d_main_window__coll__graph}
\end{center}
\end{figure}
\subsection*{Dodatkowe Dziedziczone Składowe}


\subsection{Opis szczegółowy}


Definicja w linii 200 pliku ui\-\_\-dmainwindow.\-h.



Dokumentacja dla tej klasy została wygenerowana z pliku\-:\begin{DoxyCompactItemize}
\item 
\hyperlink{ui__dmainwindow_8h}{ui\-\_\-dmainwindow.\-h}\end{DoxyCompactItemize}

\hypertarget{class_kolor}{}\section{Dokumentacja klasy Kolor}
\label{class_kolor}\index{Kolor@{Kolor}}


Klasa modelująca kolor.  




{\ttfamily \#include $<$kolor.\+hh$>$}

\subsection*{Metody publiczne}
\begin{DoxyCompactItemize}
\item 
\hyperlink{class_kolor_adde4f304856649a8a148288c271d4775}{Kolor} (int r, int g, int b)
\begin{DoxyCompactList}\small\item\em Konstruktor. \end{DoxyCompactList}\item 
\hyperlink{class_kolor_a79baff3add17cc61abb8bc71f0a426e6}{Kolor} (const \hyperlink{class_kolor}{Kolor} \&rgb)
\begin{DoxyCompactList}\small\item\em Konstruktor kopiujacy. \end{DoxyCompactList}\end{DoxyCompactItemize}
\subsection*{Atrybuty publiczne}
\begin{DoxyCompactItemize}
\item 
int \hyperlink{class_kolor_ad887fb53be523b39fbead6a24671751e}{\+\_\+r}
\begin{DoxyCompactList}\small\item\em Atrybut opisujacy wartosc odcieniu czerwonego. \end{DoxyCompactList}\item 
int \hyperlink{class_kolor_a568f73268d43f0e76c8ae75f0ef20229}{\+\_\+g}
\begin{DoxyCompactList}\small\item\em Atrybut opisujacy wartosc odcieniu zielonego. \end{DoxyCompactList}\item 
int \hyperlink{class_kolor_a543b5984743ac9d471409c697382038b}{\+\_\+b}
\begin{DoxyCompactList}\small\item\em Atrybut opisujacy wartosc odcieniu niebieskiego. \end{DoxyCompactList}\end{DoxyCompactItemize}


\subsection{Opis szczegółowy}
Klasa opisuje kolor w formacie R\+G\+B. 

Definicja w linii 24 pliku kolor.\+hh.



\subsection{Dokumentacja konstruktora i destruktora}
\hypertarget{class_kolor_adde4f304856649a8a148288c271d4775}{}\index{Kolor@{Kolor}!Kolor@{Kolor}}
\index{Kolor@{Kolor}!Kolor@{Kolor}}
\subsubsection[{Kolor}]{\setlength{\rightskip}{0pt plus 5cm}Kolor\+::\+Kolor (
\begin{DoxyParamCaption}
\item[{int}]{r, }
\item[{int}]{g, }
\item[{int}]{b}
\end{DoxyParamCaption}
)\hspace{0.3cm}{\ttfamily [inline]}}\label{class_kolor_adde4f304856649a8a148288c271d4775}
Konstruktor parametryczny, inicjalizujacy kolor podanymi wartosciami. 
\begin{DoxyParams}[1]{Parametry}
\mbox{\tt in}  & {\em r} & -\/ wartosc odcieniu czerwonego, \mbox{[}0, 255\mbox{]} \\
\hline
\mbox{\tt in}  & {\em g} & -\/ wartosc odcieniu zielonego, \mbox{[}0, 255\mbox{]} \\
\hline
\mbox{\tt in}  & {\em b} & -\/ wartosc odcieniu niebieskiego, \mbox{[}0, 255\mbox{]} \\
\hline
\end{DoxyParams}


Definicja w linii 35 pliku kolor.\+hh.

\hypertarget{class_kolor_a79baff3add17cc61abb8bc71f0a426e6}{}\index{Kolor@{Kolor}!Kolor@{Kolor}}
\index{Kolor@{Kolor}!Kolor@{Kolor}}
\subsubsection[{Kolor}]{\setlength{\rightskip}{0pt plus 5cm}Kolor\+::\+Kolor (
\begin{DoxyParamCaption}
\item[{const {\bf Kolor} \&}]{rgb}
\end{DoxyParamCaption}
)\hspace{0.3cm}{\ttfamily [inline]}}\label{class_kolor_a79baff3add17cc61abb8bc71f0a426e6}
Konstruktor kopiujacy. 
\begin{DoxyParams}[1]{Parametry}
\mbox{\tt in}  & {\em rgb} & -\/ obiekt do skopiowania \\
\hline
\end{DoxyParams}


Definicja w linii 43 pliku kolor.\+hh.



\subsection{Dokumentacja atrybutów składowych}
\hypertarget{class_kolor_a543b5984743ac9d471409c697382038b}{}\index{Kolor@{Kolor}!\+\_\+b@{\+\_\+b}}
\index{\+\_\+b@{\+\_\+b}!Kolor@{Kolor}}
\subsubsection[{\+\_\+b}]{\setlength{\rightskip}{0pt plus 5cm}int Kolor\+::\+\_\+b}\label{class_kolor_a543b5984743ac9d471409c697382038b}
Atrybut opisujacy wartosc odcieniu niebieskiego. 

Definicja w linii 66 pliku kolor.\+hh.

\hypertarget{class_kolor_a568f73268d43f0e76c8ae75f0ef20229}{}\index{Kolor@{Kolor}!\+\_\+g@{\+\_\+g}}
\index{\+\_\+g@{\+\_\+g}!Kolor@{Kolor}}
\subsubsection[{\+\_\+g}]{\setlength{\rightskip}{0pt plus 5cm}int Kolor\+::\+\_\+g}\label{class_kolor_a568f73268d43f0e76c8ae75f0ef20229}
Atrybut opisujacy wartosc odcieniu zielonego. 

Definicja w linii 59 pliku kolor.\+hh.

\hypertarget{class_kolor_ad887fb53be523b39fbead6a24671751e}{}\index{Kolor@{Kolor}!\+\_\+r@{\+\_\+r}}
\index{\+\_\+r@{\+\_\+r}!Kolor@{Kolor}}
\subsubsection[{\+\_\+r}]{\setlength{\rightskip}{0pt plus 5cm}int Kolor\+::\+\_\+r}\label{class_kolor_ad887fb53be523b39fbead6a24671751e}
Atrybut opisujacy wartosc odcieniu czerwonego. 

Definicja w linii 43 pliku kolor.\+hh.



Dokumentacja dla tej klasy została wygenerowana z pliku\+:\begin{DoxyCompactItemize}
\item 
\hyperlink{kolor_8hh}{kolor.\+hh}\end{DoxyCompactItemize}

\hypertarget{class_okno_glowne}{\section{Dokumentacja klasy Okno\-Glowne}
\label{class_okno_glowne}\index{Okno\-Glowne@{Okno\-Glowne}}
}


Klasa modelujaca głowne okno aplikacji.  




{\ttfamily \#include $<$okno\-\_\-glowne.\-hh$>$}



Diagram dziedziczenia dla Okno\-Glowne\nopagebreak
\begin{figure}[H]
\begin{center}
\leavevmode
\includegraphics[width=160pt]{class_okno_glowne__inherit__graph}
\end{center}
\end{figure}


Diagram współpracy dla Okno\-Glowne\-:\nopagebreak
\begin{figure}[H]
\begin{center}
\leavevmode
\includegraphics[width=350pt]{class_okno_glowne__coll__graph}
\end{center}
\end{figure}
\subsection*{Sloty publiczne}
\begin{DoxyCompactItemize}
\item 
void \hyperlink{class_okno_glowne_a2a49d3696ef8a42325313842768f2c92}{Gdy\-Odpowiedni\-Czas} ()
\begin{DoxyCompactList}\small\item\em Slot odpowiadajacy za aktualizacje danych. . \end{DoxyCompactList}\item 
void \hyperlink{class_okno_glowne_a2a59f13292adfead4ac821780220044a}{Gdy\-Napis} (const Q\-String \&)
\begin{DoxyCompactList}\small\item\em Slot odpowiadajacy za wyswietlenie napisu po otrzymaniu sygnalu. \end{DoxyCompactList}\item 
void \hyperlink{class_okno_glowne_ac837b1f8c8b0288d07987e059966431b}{on\-\_\-play\-Button\-\_\-clicked} ()
\begin{DoxyCompactList}\small\item\em Slot odpowiadajacy za obsluge stanu play. \end{DoxyCompactList}\item 
void \hyperlink{class_okno_glowne_ae8bd560de9aa835ba8b194b8f7da094c}{on\-\_\-pause\-Button\-\_\-clicked} ()
\begin{DoxyCompactList}\small\item\em Slot odpowiadajacy za obsluge stanu pauza. \end{DoxyCompactList}\item 
void \hyperlink{class_okno_glowne_a63255adc6263a1ee6f67c96b91446b73}{on\-\_\-stop\-Button\-\_\-clicked} ()
\begin{DoxyCompactList}\small\item\em Slot odpowiadajacy za obsluge stanu stop. \end{DoxyCompactList}\item 
void \hyperlink{class_okno_glowne_a0c07c0f31c7b79e053ffb6e606ff5293}{on\-\_\-grav\-Tgl\-Button\-\_\-clicked} ()
\begin{DoxyCompactList}\small\item\em Slot odpowiadajacy za obsluge stanu grav\-Tgl. \end{DoxyCompactList}\item 
void \hyperlink{class_okno_glowne_a726ce3fbe89c3fb7364c39e99c0ad658}{on\-\_\-slider\-Szybkosc\-Sym\-\_\-value\-Changed} (int a)
\begin{DoxyCompactList}\small\item\em Slot odpowiadajacy za wczytanie danych z pliku. \end{DoxyCompactList}\item 
void \hyperlink{class_okno_glowne_a13edcaa9e3c75c793db2b147b2ffd12b}{on\-\_\-slider\-Kat\-Obrotu\-\_\-value\-Changed} (int a)
\begin{DoxyCompactList}\small\item\em Slot odpowiadajacy za zmiane wartosci slidera. \end{DoxyCompactList}\item 
void \hyperlink{class_okno_glowne_a8ed8fc49c9c3d3e187639880ce286c88}{on\-\_\-action\-\_\-\-Save\-\_\-triggered} ()
\begin{DoxyCompactList}\small\item\em Slot odpowiadajacy za przycisniecie przycisku Save. \end{DoxyCompactList}\end{DoxyCompactItemize}
\subsection*{Sygnały}
\begin{DoxyCompactItemize}
\item 
void \hyperlink{class_okno_glowne_aa602a0c5a940f0af4ab7390bfc1a4b9d}{Zglos\-Napis} (const Q\-String \&)
\begin{DoxyCompactList}\small\item\em Sygnal zglaszajacy napis. \end{DoxyCompactList}\end{DoxyCompactItemize}
\subsection*{Metody publiczne}
\begin{DoxyCompactItemize}
\item 
\hyperlink{class_okno_glowne_a8dcfe4e0f18dfaf0c535c4549991b550}{Okno\-Glowne} (Q\-Widget $\ast$w\-Rodzic=N\-U\-L\-L)
\begin{DoxyCompactList}\small\item\em Konstruktor. \end{DoxyCompactList}\item 
virtual void \hyperlink{class_okno_glowne_a570c795e3829c3bd7896551c0624abe2}{paint\-Event} (Q\-Paint\-Event $\ast$event)
\begin{DoxyCompactList}\small\item\em Wirtualna metoda paint\-Event wyrysowujaca obiekty na ekranie. \end{DoxyCompactList}\item 
void \hyperlink{class_okno_glowne_a6062f76fdf15ad8bc0543cfd2a2fe150}{Zapisz\-Symulacje\-Do\-Pliku} ()
\begin{DoxyCompactList}\small\item\em Metoda zapisujaca aktualny stan symulacji do automatycznie generowanego pliku. \end{DoxyCompactList}\item 
void \hyperlink{class_okno_glowne_a1b8098c27e9656235bb056aeb79a8ece}{Wczytaj\-Symulacje\-Z\-Pliku} (const std\-::string nazwa\-\_\-pliku)
\begin{DoxyCompactList}\small\item\em Metoda wczytujaca z wybranego pliku stan symulacji. \end{DoxyCompactList}\item 
void \hyperlink{class_okno_glowne_ac57587eba95f28512d71705a87f8a508}{Zapisz\-Symulacje\-Do\-Pliku} (const std\-::string nazwa\-\_\-pliku)
\begin{DoxyCompactList}\small\item\em Metoda zapisujaca stan symulacji do wybranego pliku. \end{DoxyCompactList}\end{DoxyCompactItemize}
\subsection*{Atrybuty prywatne}
\begin{DoxyCompactItemize}
\item 
\hyperlink{class_zbiornik}{Zbiornik} $\ast$ \hyperlink{class_okno_glowne_af2d1275209898ebdd5ab9de8ef78dffd}{w\-Zbiornik}
\begin{DoxyCompactList}\small\item\em Wskaznik na zbiornik. \end{DoxyCompactList}\item 
Q\-Menu\-Bar $\ast$ \hyperlink{class_okno_glowne_a5a87098d9d4bd868670f5a5e72023a0a}{menu\-Bar}
\begin{DoxyCompactList}\small\item\em Wskaznik na pasek menu. \end{DoxyCompactList}\item 
Q\-Action $\ast$ \hyperlink{class_okno_glowne_a2c2d825b6e5e0faa5eb368be4fc73b78}{action\-\_\-\-Save}
\begin{DoxyCompactList}\small\item\em Wskaznik na akcje przycisku menu Save. \end{DoxyCompactList}\item 
Q\-Action $\ast$ \hyperlink{class_okno_glowne_a579ef9901f57057368cb522ea5a9a5c3}{action\-\_\-\-Exit}
\begin{DoxyCompactList}\small\item\em Wskaznik na akcje przycisku menu Exit. \end{DoxyCompactList}\item 
Q\-Menu $\ast$ \hyperlink{class_okno_glowne_a1ba162db2d0b06b0f8963e61b3806875}{menu\-\_\-\-File}
\begin{DoxyCompactList}\small\item\em Wskaznik na akcje przycisku menu File. \end{DoxyCompactList}\item 
Q\-Menu $\ast$ \hyperlink{class_okno_glowne_a93afadd0ec22ce6a7e29acc5dd2423a2}{menu\-\_\-\-Edit}
\begin{DoxyCompactList}\small\item\em Wskaznik na akcje przycisku menu Edit. \end{DoxyCompactList}\item 
Q\-Menu $\ast$ \hyperlink{class_okno_glowne_ab17be6714913af0cdf4e7de7cb6210d1}{menu\-\_\-\-Help}
\begin{DoxyCompactList}\small\item\em Wskaznik na akcje przycisku menu Help. \end{DoxyCompactList}\item 
Q\-Status\-Bar $\ast$ \hyperlink{class_okno_glowne_a40a10989bc6b318ac24e2457d7adb53b}{status\-Bar}
\begin{DoxyCompactList}\small\item\em Wskaznik na pasek statusowy. \end{DoxyCompactList}\item 
Q\-Tool\-Bar $\ast$ \hyperlink{class_okno_glowne_a6a37dd1f32605092fff7feac712bf429}{tool\-Bar}
\begin{DoxyCompactList}\small\item\em Wskaznik na pasek narzedziowy. \end{DoxyCompactList}\item 
Q\-H\-Box\-Layout $\ast$ \hyperlink{class_okno_glowne_aacb5ddb6d0eb560a47917cc1b457239a}{horizontal\-Layout}
\begin{DoxyCompactList}\small\item\em Wskaznik na obszar do horyzontalnego rozmieszczenia przyciskow. \end{DoxyCompactList}\item 
Q\-Widget $\ast$ \hyperlink{class_okno_glowne_a12ac2d00b9ca186176ccc710a928a723}{horizontal\-Layout\-Widget}
\begin{DoxyCompactList}\small\item\em Wskaznik na widget odpowiedzialny za horyzontalne wyswietlenie przyciskow. \end{DoxyCompactList}\item 
Q\-Push\-Button $\ast$ \hyperlink{class_okno_glowne_a50f936486c1bc3b3278823a8eb90841e}{play\-Button}
\begin{DoxyCompactList}\small\item\em Wskaznik na przycisk play. \end{DoxyCompactList}\item 
Q\-Push\-Button $\ast$ \hyperlink{class_okno_glowne_a0dde8df8a49b8f47f17f8e748fd15967}{pause\-Button}
\begin{DoxyCompactList}\small\item\em Wskaznik na przycisk pause. \end{DoxyCompactList}\item 
Q\-Push\-Button $\ast$ \hyperlink{class_okno_glowne_a3051d73dc0e0a27dc30ada43cc6b63c4}{stop\-Button}
\begin{DoxyCompactList}\small\item\em Wskaznik na przycisk stop. \end{DoxyCompactList}\item 
Q\-Check\-Box $\ast$ \hyperlink{class_okno_glowne_ae6e8681e77286aae1f4025ee5ba1ad69}{grav\-Tgl\-Button}
\begin{DoxyCompactList}\small\item\em Wskaznik na check box grav\-Tgl. \end{DoxyCompactList}\item 
Q\-Push\-Button $\ast$ \hyperlink{class_okno_glowne_accbadc3bc4d418cfe1bce2be61881917}{load\-Button}
\begin{DoxyCompactList}\small\item\em Wskaznik na przycisk Wczytaj. \end{DoxyCompactList}\item 
Q\-Push\-Button $\ast$ \hyperlink{class_okno_glowne_a81e6650fa592f04bf0adc3bebd3346d6}{save\-Button}
\begin{DoxyCompactList}\small\item\em Wskaznik na przycisk Zapisz. \end{DoxyCompactList}\item 
Q\-Slider $\ast$ \hyperlink{class_okno_glowne_aaee43ea7074cff126b069c60657d698d}{slider\-Kat\-Obrotu}
\begin{DoxyCompactList}\small\item\em Wskaznik na slider. \end{DoxyCompactList}\item 
Q\-Slider $\ast$ \hyperlink{class_okno_glowne_a85328893065393400d5a0344004ca78b}{slider\-Szybkosc\-Sym}
\begin{DoxyCompactList}\small\item\em Wskaznik na slider. \end{DoxyCompactList}\item 
Q\-L\-C\-D\-Number $\ast$ \hyperlink{class_okno_glowne_ab100c00d4ba33d896fd0985ac366296a}{lcd\-Szybkosc\-Sym}
\begin{DoxyCompactList}\small\item\em Wskaznik na L\-C\-D z szybkoscia symulacji. \end{DoxyCompactList}\item 
Q\-Label $\ast$ \hyperlink{class_okno_glowne_ad7b0708ffdf61f3bef1349cc353a6c4e}{label\-Szybkosc\-Sym}
\begin{DoxyCompactList}\small\item\em Wskaznik na etykiete z szybkoscia symulacji. \end{DoxyCompactList}\item 
Q\-Label $\ast$ \hyperlink{class_okno_glowne_aca07e1dc5cbe30d6952f9b952073bb79}{label\-Czas\-Sym}
\begin{DoxyCompactList}\small\item\em Wskaznik na etykiete z czasem symulacji. \end{DoxyCompactList}\item 
Q\-L\-C\-D\-Number $\ast$ \hyperlink{class_okno_glowne_ab34fefe738e38b1b0d4ce764481cc0c6}{lcd\-Czas\-Sym}
\begin{DoxyCompactList}\small\item\em Wskaznik na L\-C\-D z czasem trwania symulacji. \end{DoxyCompactList}\item 
Q\-Label $\ast$ \hyperlink{class_okno_glowne_ab01460f1222d0ec2892abf21efb23078}{label\-Liczba\-Czasteczek}
\begin{DoxyCompactList}\small\item\em Wskaznik na etykiete z liczbe czasteczek. \end{DoxyCompactList}\item 
Q\-L\-C\-D\-Number $\ast$ \hyperlink{class_okno_glowne_adbdd9fc009725804e015d267dc8375dc}{lcd\-Liczba\-Czasteczek}
\begin{DoxyCompactList}\small\item\em Wskaznik na L\-C\-D z liczbe czasteczek. \end{DoxyCompactList}\item 
Q\-Line\-Edit $\ast$ \hyperlink{class_okno_glowne_a0112b8be70a26552b03f38fab43a3301}{line\-Edit}
\begin{DoxyCompactList}\small\item\em Wskaznik na linijke do wpisywania tekstu. \end{DoxyCompactList}\item 
Q\-Timer \hyperlink{class_okno_glowne_a5d047f90666212f58e69d11af3285d9b}{\-\_\-\-Stoper}
\begin{DoxyCompactList}\small\item\em Miernik czasu. \end{DoxyCompactList}\item 
double \hyperlink{class_okno_glowne_a6a0922607c0970ecdfe8adec7a773c7f}{\-\_\-old\-\_\-width}
\begin{DoxyCompactList}\small\item\em Stara szerokosc okienka. \end{DoxyCompactList}\item 
double \hyperlink{class_okno_glowne_a7dae1b25dbade179eb6dfc30ffeab14b}{\-\_\-old\-\_\-height}
\begin{DoxyCompactList}\small\item\em Stara wysokosc okienka. \end{DoxyCompactList}\end{DoxyCompactItemize}
\subsection*{Statyczne atrybuty prywatne}
\begin{DoxyCompactItemize}
\item 
static int \hyperlink{class_okno_glowne_ae615cbd9c9f9ab06b365c4692ff68729}{licznik\-\_\-plikow}
\begin{DoxyCompactList}\small\item\em Sluzy do generowania unikatowych nazw plikow wyjsciowych. \end{DoxyCompactList}\end{DoxyCompactItemize}


\subsection{Opis szczegółowy}
Dzieki tej klasie wyswietlane jest okno glowne aplikacji. 

Definicja w linii 68 pliku okno\-\_\-glowne.\-hh.



\subsection{Dokumentacja konstruktora i destruktora}
\hypertarget{class_okno_glowne_a8dcfe4e0f18dfaf0c535c4549991b550}{\index{Okno\-Glowne@{Okno\-Glowne}!Okno\-Glowne@{Okno\-Glowne}}
\index{Okno\-Glowne@{Okno\-Glowne}!OknoGlowne@{Okno\-Glowne}}
\subsubsection[{Okno\-Glowne}]{\setlength{\rightskip}{0pt plus 5cm}Okno\-Glowne\-::\-Okno\-Glowne (
\begin{DoxyParamCaption}
\item[{Q\-Widget $\ast$}]{w\-Rodzic = {\ttfamily NULL}}
\end{DoxyParamCaption}
)}}\label{class_okno_glowne_a8dcfe4e0f18dfaf0c535c4549991b550}
Konstruktor parametryczny. 
\begin{DoxyParams}[1]{Parametry}
\mbox{\tt in,out}  & {\em w\-Rodzic} & -\/ wskaznik na rodzica \\
\hline
\end{DoxyParams}


Definicja w linii 16 pliku okno\-\_\-glowne.\-cpp.



Oto graf wywołań dla tej funkcji\-:\nopagebreak
\begin{figure}[H]
\begin{center}
\leavevmode
\includegraphics[width=350pt]{class_okno_glowne_a8dcfe4e0f18dfaf0c535c4549991b550_cgraph}
\end{center}
\end{figure}




\subsection{Dokumentacja funkcji składowych}
\hypertarget{class_okno_glowne_a2a59f13292adfead4ac821780220044a}{\index{Okno\-Glowne@{Okno\-Glowne}!Gdy\-Napis@{Gdy\-Napis}}
\index{Gdy\-Napis@{Gdy\-Napis}!OknoGlowne@{Okno\-Glowne}}
\subsubsection[{Gdy\-Napis}]{\setlength{\rightskip}{0pt plus 5cm}void Okno\-Glowne\-::\-Gdy\-Napis (
\begin{DoxyParamCaption}
\item[{const Q\-String \&}]{Napis}
\end{DoxyParamCaption}
)\hspace{0.3cm}{\ttfamily [slot]}}}\label{class_okno_glowne_a2a59f13292adfead4ac821780220044a}
Odpowiada za wyswietlenie napisu na belce statusowej. 
\begin{DoxyParams}[1]{Parametry}
\mbox{\tt in}  & {\em Napis} & -\/ napis do wyswietlenia \\
\hline
\end{DoxyParams}


Definicja w linii 264 pliku okno\-\_\-glowne.\-cpp.



Oto graf wywoływań tej funkcji\-:\nopagebreak
\begin{figure}[H]
\begin{center}
\leavevmode
\includegraphics[width=350pt]{class_okno_glowne_a2a59f13292adfead4ac821780220044a_icgraph}
\end{center}
\end{figure}


\hypertarget{class_okno_glowne_a2a49d3696ef8a42325313842768f2c92}{\index{Okno\-Glowne@{Okno\-Glowne}!Gdy\-Odpowiedni\-Czas@{Gdy\-Odpowiedni\-Czas}}
\index{Gdy\-Odpowiedni\-Czas@{Gdy\-Odpowiedni\-Czas}!OknoGlowne@{Okno\-Glowne}}
\subsubsection[{Gdy\-Odpowiedni\-Czas}]{\setlength{\rightskip}{0pt plus 5cm}void Okno\-Glowne\-::\-Gdy\-Odpowiedni\-Czas (
\begin{DoxyParamCaption}
{}
\end{DoxyParamCaption}
)\hspace{0.3cm}{\ttfamily [slot]}}}\label{class_okno_glowne_a2a49d3696ef8a42325313842768f2c92}
Odpowiada za uaktualnienie okienka w odpowiednich momentach. 

Definicja w linii 247 pliku okno\-\_\-glowne.\-cpp.



Oto graf wywołań dla tej funkcji\-:\nopagebreak
\begin{figure}[H]
\begin{center}
\leavevmode
\includegraphics[width=350pt]{class_okno_glowne_a2a49d3696ef8a42325313842768f2c92_cgraph}
\end{center}
\end{figure}




Oto graf wywoływań tej funkcji\-:\nopagebreak
\begin{figure}[H]
\begin{center}
\leavevmode
\includegraphics[width=350pt]{class_okno_glowne_a2a49d3696ef8a42325313842768f2c92_icgraph}
\end{center}
\end{figure}


\hypertarget{class_okno_glowne_a8ed8fc49c9c3d3e187639880ce286c88}{\index{Okno\-Glowne@{Okno\-Glowne}!on\-\_\-action\-\_\-\-Save\-\_\-triggered@{on\-\_\-action\-\_\-\-Save\-\_\-triggered}}
\index{on\-\_\-action\-\_\-\-Save\-\_\-triggered@{on\-\_\-action\-\_\-\-Save\-\_\-triggered}!OknoGlowne@{Okno\-Glowne}}
\subsubsection[{on\-\_\-action\-\_\-\-Save\-\_\-triggered}]{\setlength{\rightskip}{0pt plus 5cm}void Okno\-Glowne\-::on\-\_\-action\-\_\-\-Save\-\_\-triggered (
\begin{DoxyParamCaption}
{}
\end{DoxyParamCaption}
)\hspace{0.3cm}{\ttfamily [slot]}}}\label{class_okno_glowne_a8ed8fc49c9c3d3e187639880ce286c88}
Odpowiada za wykonanie odpowiednich czynnosci po przycisnieciu Save. 

Definicja w linii 236 pliku okno\-\_\-glowne.\-cpp.



Oto graf wywołań dla tej funkcji\-:\nopagebreak
\begin{figure}[H]
\begin{center}
\leavevmode
\includegraphics[width=350pt]{class_okno_glowne_a8ed8fc49c9c3d3e187639880ce286c88_cgraph}
\end{center}
\end{figure}


\hypertarget{class_okno_glowne_a0c07c0f31c7b79e053ffb6e606ff5293}{\index{Okno\-Glowne@{Okno\-Glowne}!on\-\_\-grav\-Tgl\-Button\-\_\-clicked@{on\-\_\-grav\-Tgl\-Button\-\_\-clicked}}
\index{on\-\_\-grav\-Tgl\-Button\-\_\-clicked@{on\-\_\-grav\-Tgl\-Button\-\_\-clicked}!OknoGlowne@{Okno\-Glowne}}
\subsubsection[{on\-\_\-grav\-Tgl\-Button\-\_\-clicked}]{\setlength{\rightskip}{0pt plus 5cm}void Okno\-Glowne\-::on\-\_\-grav\-Tgl\-Button\-\_\-clicked (
\begin{DoxyParamCaption}
{}
\end{DoxyParamCaption}
)\hspace{0.3cm}{\ttfamily [slot]}}}\label{class_okno_glowne_a0c07c0f31c7b79e053ffb6e606ff5293}
Odpowiada za zmianę parametru symulacji po kliknięciu grav\-Tgl. 

Definicja w linii 205 pliku okno\-\_\-glowne.\-cpp.

\hypertarget{class_okno_glowne_ae8bd560de9aa835ba8b194b8f7da094c}{\index{Okno\-Glowne@{Okno\-Glowne}!on\-\_\-pause\-Button\-\_\-clicked@{on\-\_\-pause\-Button\-\_\-clicked}}
\index{on\-\_\-pause\-Button\-\_\-clicked@{on\-\_\-pause\-Button\-\_\-clicked}!OknoGlowne@{Okno\-Glowne}}
\subsubsection[{on\-\_\-pause\-Button\-\_\-clicked}]{\setlength{\rightskip}{0pt plus 5cm}void Okno\-Glowne\-::on\-\_\-pause\-Button\-\_\-clicked (
\begin{DoxyParamCaption}
{}
\end{DoxyParamCaption}
)\hspace{0.3cm}{\ttfamily [slot]}}}\label{class_okno_glowne_ae8bd560de9aa835ba8b194b8f7da094c}
Odpowiada za wykonanie odpowiednich czynnosci w trakcie stanu pauza. 

Definicja w linii 194 pliku okno\-\_\-glowne.\-cpp.

\hypertarget{class_okno_glowne_ac837b1f8c8b0288d07987e059966431b}{\index{Okno\-Glowne@{Okno\-Glowne}!on\-\_\-play\-Button\-\_\-clicked@{on\-\_\-play\-Button\-\_\-clicked}}
\index{on\-\_\-play\-Button\-\_\-clicked@{on\-\_\-play\-Button\-\_\-clicked}!OknoGlowne@{Okno\-Glowne}}
\subsubsection[{on\-\_\-play\-Button\-\_\-clicked}]{\setlength{\rightskip}{0pt plus 5cm}void Okno\-Glowne\-::on\-\_\-play\-Button\-\_\-clicked (
\begin{DoxyParamCaption}
{}
\end{DoxyParamCaption}
)\hspace{0.3cm}{\ttfamily [slot]}}}\label{class_okno_glowne_ac837b1f8c8b0288d07987e059966431b}
Odpowiada za wykonanie odpowiednich czynnosci w trakcie stanu play. 

Definicja w linii 186 pliku okno\-\_\-glowne.\-cpp.

\hypertarget{class_okno_glowne_a13edcaa9e3c75c793db2b147b2ffd12b}{\index{Okno\-Glowne@{Okno\-Glowne}!on\-\_\-slider\-Kat\-Obrotu\-\_\-value\-Changed@{on\-\_\-slider\-Kat\-Obrotu\-\_\-value\-Changed}}
\index{on\-\_\-slider\-Kat\-Obrotu\-\_\-value\-Changed@{on\-\_\-slider\-Kat\-Obrotu\-\_\-value\-Changed}!OknoGlowne@{Okno\-Glowne}}
\subsubsection[{on\-\_\-slider\-Kat\-Obrotu\-\_\-value\-Changed}]{\setlength{\rightskip}{0pt plus 5cm}void Okno\-Glowne\-::on\-\_\-slider\-Kat\-Obrotu\-\_\-value\-Changed (
\begin{DoxyParamCaption}
\item[{int}]{a}
\end{DoxyParamCaption}
)\hspace{0.3cm}{\ttfamily [slot]}}}\label{class_okno_glowne_a13edcaa9e3c75c793db2b147b2ffd12b}
Odpowiada za wykonanie odpowiednich czynnosci zwiazanych z katem obrotu po zmianie wartosci slidera. 

Definicja w linii 225 pliku okno\-\_\-glowne.\-cpp.



Oto graf wywołań dla tej funkcji\-:\nopagebreak
\begin{figure}[H]
\begin{center}
\leavevmode
\includegraphics[width=350pt]{class_okno_glowne_a13edcaa9e3c75c793db2b147b2ffd12b_cgraph}
\end{center}
\end{figure}


\hypertarget{class_okno_glowne_a726ce3fbe89c3fb7364c39e99c0ad658}{\index{Okno\-Glowne@{Okno\-Glowne}!on\-\_\-slider\-Szybkosc\-Sym\-\_\-value\-Changed@{on\-\_\-slider\-Szybkosc\-Sym\-\_\-value\-Changed}}
\index{on\-\_\-slider\-Szybkosc\-Sym\-\_\-value\-Changed@{on\-\_\-slider\-Szybkosc\-Sym\-\_\-value\-Changed}!OknoGlowne@{Okno\-Glowne}}
\subsubsection[{on\-\_\-slider\-Szybkosc\-Sym\-\_\-value\-Changed}]{\setlength{\rightskip}{0pt plus 5cm}void Okno\-Glowne\-::on\-\_\-slider\-Szybkosc\-Sym\-\_\-value\-Changed (
\begin{DoxyParamCaption}
\item[{int}]{a}
\end{DoxyParamCaption}
)\hspace{0.3cm}{\ttfamily [slot]}}}\label{class_okno_glowne_a726ce3fbe89c3fb7364c39e99c0ad658}
Odpowiada za wczytanie danych z pliku.

Slot odpowiadajacy za wczytanie danych z pliku.

Odpowiada za wczytanie danych z pliku.

Slot odpowiadajacy za zmiane wartosci slidera.

Odpowiada za wykonanie odpowiednich czynnosci zwiazanych z szybkoscia symulacji po zmianie wartosci slidera. 

Definicja w linii 229 pliku okno\-\_\-glowne.\-cpp.



Oto graf wywołań dla tej funkcji\-:\nopagebreak
\begin{figure}[H]
\begin{center}
\leavevmode
\includegraphics[width=350pt]{class_okno_glowne_a726ce3fbe89c3fb7364c39e99c0ad658_cgraph}
\end{center}
\end{figure}


\hypertarget{class_okno_glowne_a63255adc6263a1ee6f67c96b91446b73}{\index{Okno\-Glowne@{Okno\-Glowne}!on\-\_\-stop\-Button\-\_\-clicked@{on\-\_\-stop\-Button\-\_\-clicked}}
\index{on\-\_\-stop\-Button\-\_\-clicked@{on\-\_\-stop\-Button\-\_\-clicked}!OknoGlowne@{Okno\-Glowne}}
\subsubsection[{on\-\_\-stop\-Button\-\_\-clicked}]{\setlength{\rightskip}{0pt plus 5cm}void Okno\-Glowne\-::on\-\_\-stop\-Button\-\_\-clicked (
\begin{DoxyParamCaption}
{}
\end{DoxyParamCaption}
)\hspace{0.3cm}{\ttfamily [slot]}}}\label{class_okno_glowne_a63255adc6263a1ee6f67c96b91446b73}
Odpowiada za wykonanie odpowiednich czynnosci w trakcie stanu stop. 

Definicja w linii 198 pliku okno\-\_\-glowne.\-cpp.



Oto graf wywołań dla tej funkcji\-:\nopagebreak
\begin{figure}[H]
\begin{center}
\leavevmode
\includegraphics[width=350pt]{class_okno_glowne_a63255adc6263a1ee6f67c96b91446b73_cgraph}
\end{center}
\end{figure}


\hypertarget{class_okno_glowne_a570c795e3829c3bd7896551c0624abe2}{\index{Okno\-Glowne@{Okno\-Glowne}!paint\-Event@{paint\-Event}}
\index{paint\-Event@{paint\-Event}!OknoGlowne@{Okno\-Glowne}}
\subsubsection[{paint\-Event}]{\setlength{\rightskip}{0pt plus 5cm}void Okno\-Glowne\-::paint\-Event (
\begin{DoxyParamCaption}
\item[{Q\-Paint\-Event $\ast$}]{event}
\end{DoxyParamCaption}
)\hspace{0.3cm}{\ttfamily [virtual]}}}\label{class_okno_glowne_a570c795e3829c3bd7896551c0624abe2}
Odziedziczona wirtualna metoda paint\-Event. Rysuje zbiornik i przyciski w nowym miejscu. 
\begin{DoxyParams}[1]{Parametry}
\mbox{\tt in,out}  & {\em event} & -\/ wskaznik obiekt klasy Q\-Paint\-Event \\
\hline
\end{DoxyParams}


Definicja w linii 268 pliku okno\-\_\-glowne.\-cpp.



Oto graf wywołań dla tej funkcji\-:\nopagebreak
\begin{figure}[H]
\begin{center}
\leavevmode
\includegraphics[width=350pt]{class_okno_glowne_a570c795e3829c3bd7896551c0624abe2_cgraph}
\end{center}
\end{figure}


\hypertarget{class_okno_glowne_a1b8098c27e9656235bb056aeb79a8ece}{\index{Okno\-Glowne@{Okno\-Glowne}!Wczytaj\-Symulacje\-Z\-Pliku@{Wczytaj\-Symulacje\-Z\-Pliku}}
\index{Wczytaj\-Symulacje\-Z\-Pliku@{Wczytaj\-Symulacje\-Z\-Pliku}!OknoGlowne@{Okno\-Glowne}}
\subsubsection[{Wczytaj\-Symulacje\-Z\-Pliku}]{\setlength{\rightskip}{0pt plus 5cm}void Okno\-Glowne\-::\-Wczytaj\-Symulacje\-Z\-Pliku (
\begin{DoxyParamCaption}
\item[{const std\-::string}]{nazwa\-\_\-pliku}
\end{DoxyParamCaption}
)}}\label{class_okno_glowne_a1b8098c27e9656235bb056aeb79a8ece}
Wczytuje stan symulacji (czas, liczba czasteczek, dane czasteczek). 

Definicja w linii 327 pliku okno\-\_\-glowne.\-cpp.



Oto graf wywołań dla tej funkcji\-:\nopagebreak
\begin{figure}[H]
\begin{center}
\leavevmode
\includegraphics[width=350pt]{class_okno_glowne_a1b8098c27e9656235bb056aeb79a8ece_cgraph}
\end{center}
\end{figure}


\hypertarget{class_okno_glowne_a6062f76fdf15ad8bc0543cfd2a2fe150}{\index{Okno\-Glowne@{Okno\-Glowne}!Zapisz\-Symulacje\-Do\-Pliku@{Zapisz\-Symulacje\-Do\-Pliku}}
\index{Zapisz\-Symulacje\-Do\-Pliku@{Zapisz\-Symulacje\-Do\-Pliku}!OknoGlowne@{Okno\-Glowne}}
\subsubsection[{Zapisz\-Symulacje\-Do\-Pliku}]{\setlength{\rightskip}{0pt plus 5cm}void Okno\-Glowne\-::\-Zapisz\-Symulacje\-Do\-Pliku (
\begin{DoxyParamCaption}
{}
\end{DoxyParamCaption}
)}}\label{class_okno_glowne_a6062f76fdf15ad8bc0543cfd2a2fe150}
Zapisuje aktualny stan symulacji (czas, liczba czasteczek, dane czasteczek) do automatycznie generowanego pliku. 

Definicja w linii 285 pliku okno\-\_\-glowne.\-cpp.



Oto graf wywołań dla tej funkcji\-:\nopagebreak
\begin{figure}[H]
\begin{center}
\leavevmode
\includegraphics[width=350pt]{class_okno_glowne_a6062f76fdf15ad8bc0543cfd2a2fe150_cgraph}
\end{center}
\end{figure}




Oto graf wywoływań tej funkcji\-:\nopagebreak
\begin{figure}[H]
\begin{center}
\leavevmode
\includegraphics[width=350pt]{class_okno_glowne_a6062f76fdf15ad8bc0543cfd2a2fe150_icgraph}
\end{center}
\end{figure}


\hypertarget{class_okno_glowne_ac57587eba95f28512d71705a87f8a508}{\index{Okno\-Glowne@{Okno\-Glowne}!Zapisz\-Symulacje\-Do\-Pliku@{Zapisz\-Symulacje\-Do\-Pliku}}
\index{Zapisz\-Symulacje\-Do\-Pliku@{Zapisz\-Symulacje\-Do\-Pliku}!OknoGlowne@{Okno\-Glowne}}
\subsubsection[{Zapisz\-Symulacje\-Do\-Pliku}]{\setlength{\rightskip}{0pt plus 5cm}void Okno\-Glowne\-::\-Zapisz\-Symulacje\-Do\-Pliku (
\begin{DoxyParamCaption}
\item[{const std\-::string}]{nazwa\-\_\-pliku}
\end{DoxyParamCaption}
)}}\label{class_okno_glowne_ac57587eba95f28512d71705a87f8a508}
Zapisuje stan symulacji (czas, liczba czasteczek, dane czasteczek). 

Definicja w linii 308 pliku okno\-\_\-glowne.\-cpp.



Oto graf wywołań dla tej funkcji\-:\nopagebreak
\begin{figure}[H]
\begin{center}
\leavevmode
\includegraphics[width=350pt]{class_okno_glowne_ac57587eba95f28512d71705a87f8a508_cgraph}
\end{center}
\end{figure}


\hypertarget{class_okno_glowne_aa602a0c5a940f0af4ab7390bfc1a4b9d}{\index{Okno\-Glowne@{Okno\-Glowne}!Zglos\-Napis@{Zglos\-Napis}}
\index{Zglos\-Napis@{Zglos\-Napis}!OknoGlowne@{Okno\-Glowne}}
\subsubsection[{Zglos\-Napis}]{\setlength{\rightskip}{0pt plus 5cm}void Okno\-Glowne\-::\-Zglos\-Napis (
\begin{DoxyParamCaption}
\item[{const Q\-String \&}]{\-\_\-t1}
\end{DoxyParamCaption}
)\hspace{0.3cm}{\ttfamily [signal]}}}\label{class_okno_glowne_aa602a0c5a940f0af4ab7390bfc1a4b9d}
Sygnal zglaszajacy napis do odpowiedniego slotu. 
\begin{DoxyParams}[1]{Parametry}
\mbox{\tt in}  & {\em \-\_\-t1} & -\/ napis do zgloszenia \\
\hline
\end{DoxyParams}


Definicja w linii 122 pliku moc\-\_\-okno\-\_\-glowne.\-cpp.



Oto graf wywoływań tej funkcji\-:\nopagebreak
\begin{figure}[H]
\begin{center}
\leavevmode
\includegraphics[width=350pt]{class_okno_glowne_aa602a0c5a940f0af4ab7390bfc1a4b9d_icgraph}
\end{center}
\end{figure}




\subsection{Dokumentacja atrybutów składowych}
\hypertarget{class_okno_glowne_a7dae1b25dbade179eb6dfc30ffeab14b}{\index{Okno\-Glowne@{Okno\-Glowne}!\-\_\-old\-\_\-height@{\-\_\-old\-\_\-height}}
\index{\-\_\-old\-\_\-height@{\-\_\-old\-\_\-height}!OknoGlowne@{Okno\-Glowne}}
\subsubsection[{\-\_\-old\-\_\-height}]{\setlength{\rightskip}{0pt plus 5cm}double Okno\-Glowne\-::\-\_\-old\-\_\-height\hspace{0.3cm}{\ttfamily [private]}}}\label{class_okno_glowne_a7dae1b25dbade179eb6dfc30ffeab14b}
Stara wysokosc okienka. Potrzebna do umieszczania elementów w odpowiednich miejscach po zmianie wymiarów okienka. 

Definicja w linii 414 pliku okno\-\_\-glowne.\-hh.

\hypertarget{class_okno_glowne_a6a0922607c0970ecdfe8adec7a773c7f}{\index{Okno\-Glowne@{Okno\-Glowne}!\-\_\-old\-\_\-width@{\-\_\-old\-\_\-width}}
\index{\-\_\-old\-\_\-width@{\-\_\-old\-\_\-width}!OknoGlowne@{Okno\-Glowne}}
\subsubsection[{\-\_\-old\-\_\-width}]{\setlength{\rightskip}{0pt plus 5cm}double Okno\-Glowne\-::\-\_\-old\-\_\-width\hspace{0.3cm}{\ttfamily [private]}}}\label{class_okno_glowne_a6a0922607c0970ecdfe8adec7a773c7f}
Stara szerokosc okienka. Potrzebne do umieszczania elementów w odpowiednich miejscach po zmianie wymiarów okienka. 

Definicja w linii 407 pliku okno\-\_\-glowne.\-hh.

\hypertarget{class_okno_glowne_a5d047f90666212f58e69d11af3285d9b}{\index{Okno\-Glowne@{Okno\-Glowne}!\-\_\-\-Stoper@{\-\_\-\-Stoper}}
\index{\-\_\-\-Stoper@{\-\_\-\-Stoper}!OknoGlowne@{Okno\-Glowne}}
\subsubsection[{\-\_\-\-Stoper}]{\setlength{\rightskip}{0pt plus 5cm}Q\-Timer Okno\-Glowne\-::\-\_\-\-Stoper\hspace{0.3cm}{\ttfamily [private]}}}\label{class_okno_glowne_a5d047f90666212f58e69d11af3285d9b}
Miernik czasu. 

Definicja w linii 398 pliku okno\-\_\-glowne.\-hh.

\hypertarget{class_okno_glowne_a579ef9901f57057368cb522ea5a9a5c3}{\index{Okno\-Glowne@{Okno\-Glowne}!action\-\_\-\-Exit@{action\-\_\-\-Exit}}
\index{action\-\_\-\-Exit@{action\-\_\-\-Exit}!OknoGlowne@{Okno\-Glowne}}
\subsubsection[{action\-\_\-\-Exit}]{\setlength{\rightskip}{0pt plus 5cm}Q\-Action$\ast$ Okno\-Glowne\-::action\-\_\-\-Exit\hspace{0.3cm}{\ttfamily [private]}}}\label{class_okno_glowne_a579ef9901f57057368cb522ea5a9a5c3}
Wskaznik na akcje przycisku menu Exit. 

Definicja w linii 235 pliku okno\-\_\-glowne.\-hh.

\hypertarget{class_okno_glowne_a2c2d825b6e5e0faa5eb368be4fc73b78}{\index{Okno\-Glowne@{Okno\-Glowne}!action\-\_\-\-Save@{action\-\_\-\-Save}}
\index{action\-\_\-\-Save@{action\-\_\-\-Save}!OknoGlowne@{Okno\-Glowne}}
\subsubsection[{action\-\_\-\-Save}]{\setlength{\rightskip}{0pt plus 5cm}Q\-Action$\ast$ Okno\-Glowne\-::action\-\_\-\-Save\hspace{0.3cm}{\ttfamily [private]}}}\label{class_okno_glowne_a2c2d825b6e5e0faa5eb368be4fc73b78}
Wskaznik na akcje przycisku menu Save. 

Definicja w linii 228 pliku okno\-\_\-glowne.\-hh.

\hypertarget{class_okno_glowne_ae6e8681e77286aae1f4025ee5ba1ad69}{\index{Okno\-Glowne@{Okno\-Glowne}!grav\-Tgl\-Button@{grav\-Tgl\-Button}}
\index{grav\-Tgl\-Button@{grav\-Tgl\-Button}!OknoGlowne@{Okno\-Glowne}}
\subsubsection[{grav\-Tgl\-Button}]{\setlength{\rightskip}{0pt plus 5cm}Q\-Check\-Box$\ast$ Okno\-Glowne\-::grav\-Tgl\-Button\hspace{0.3cm}{\ttfamily [private]}}}\label{class_okno_glowne_ae6e8681e77286aae1f4025ee5ba1ad69}
Wskaznik na check box grav\-Tgl. Zeruje wektor grawitacji. 

Definicja w linii 313 pliku okno\-\_\-glowne.\-hh.

\hypertarget{class_okno_glowne_aacb5ddb6d0eb560a47917cc1b457239a}{\index{Okno\-Glowne@{Okno\-Glowne}!horizontal\-Layout@{horizontal\-Layout}}
\index{horizontal\-Layout@{horizontal\-Layout}!OknoGlowne@{Okno\-Glowne}}
\subsubsection[{horizontal\-Layout}]{\setlength{\rightskip}{0pt plus 5cm}Q\-H\-Box\-Layout$\ast$ Okno\-Glowne\-::horizontal\-Layout\hspace{0.3cm}{\ttfamily [private]}}}\label{class_okno_glowne_aacb5ddb6d0eb560a47917cc1b457239a}
Wskaznik na obszar do horyzontalnego rozmieszczenia przyciskow. 

Definicja w linii 277 pliku okno\-\_\-glowne.\-hh.

\hypertarget{class_okno_glowne_a12ac2d00b9ca186176ccc710a928a723}{\index{Okno\-Glowne@{Okno\-Glowne}!horizontal\-Layout\-Widget@{horizontal\-Layout\-Widget}}
\index{horizontal\-Layout\-Widget@{horizontal\-Layout\-Widget}!OknoGlowne@{Okno\-Glowne}}
\subsubsection[{horizontal\-Layout\-Widget}]{\setlength{\rightskip}{0pt plus 5cm}Q\-Widget$\ast$ Okno\-Glowne\-::horizontal\-Layout\-Widget\hspace{0.3cm}{\ttfamily [private]}}}\label{class_okno_glowne_a12ac2d00b9ca186176ccc710a928a723}
Wskaznik na widget odpowiedzialny za horyzontalne wyswietlenie przyciskow. 

Definicja w linii 284 pliku okno\-\_\-glowne.\-hh.

\hypertarget{class_okno_glowne_aca07e1dc5cbe30d6952f9b952073bb79}{\index{Okno\-Glowne@{Okno\-Glowne}!label\-Czas\-Sym@{label\-Czas\-Sym}}
\index{label\-Czas\-Sym@{label\-Czas\-Sym}!OknoGlowne@{Okno\-Glowne}}
\subsubsection[{label\-Czas\-Sym}]{\setlength{\rightskip}{0pt plus 5cm}Q\-Label$\ast$ Okno\-Glowne\-::label\-Czas\-Sym\hspace{0.3cm}{\ttfamily [private]}}}\label{class_okno_glowne_aca07e1dc5cbe30d6952f9b952073bb79}
Etykieta dla czasu symulacji. 

Definicja w linii 362 pliku okno\-\_\-glowne.\-hh.

\hypertarget{class_okno_glowne_ab01460f1222d0ec2892abf21efb23078}{\index{Okno\-Glowne@{Okno\-Glowne}!label\-Liczba\-Czasteczek@{label\-Liczba\-Czasteczek}}
\index{label\-Liczba\-Czasteczek@{label\-Liczba\-Czasteczek}!OknoGlowne@{Okno\-Glowne}}
\subsubsection[{label\-Liczba\-Czasteczek}]{\setlength{\rightskip}{0pt plus 5cm}Q\-Label$\ast$ Okno\-Glowne\-::label\-Liczba\-Czasteczek\hspace{0.3cm}{\ttfamily [private]}}}\label{class_okno_glowne_ab01460f1222d0ec2892abf21efb23078}
Etykieta dla liczby symulowanych czasteczek. 

Definicja w linii 376 pliku okno\-\_\-glowne.\-hh.

\hypertarget{class_okno_glowne_ad7b0708ffdf61f3bef1349cc353a6c4e}{\index{Okno\-Glowne@{Okno\-Glowne}!label\-Szybkosc\-Sym@{label\-Szybkosc\-Sym}}
\index{label\-Szybkosc\-Sym@{label\-Szybkosc\-Sym}!OknoGlowne@{Okno\-Glowne}}
\subsubsection[{label\-Szybkosc\-Sym}]{\setlength{\rightskip}{0pt plus 5cm}Q\-Label$\ast$ Okno\-Glowne\-::label\-Szybkosc\-Sym\hspace{0.3cm}{\ttfamily [private]}}}\label{class_okno_glowne_ad7b0708ffdf61f3bef1349cc353a6c4e}
Etykieta dla szybkosci symulacji. 

Definicja w linii 355 pliku okno\-\_\-glowne.\-hh.

\hypertarget{class_okno_glowne_ab34fefe738e38b1b0d4ce764481cc0c6}{\index{Okno\-Glowne@{Okno\-Glowne}!lcd\-Czas\-Sym@{lcd\-Czas\-Sym}}
\index{lcd\-Czas\-Sym@{lcd\-Czas\-Sym}!OknoGlowne@{Okno\-Glowne}}
\subsubsection[{lcd\-Czas\-Sym}]{\setlength{\rightskip}{0pt plus 5cm}Q\-L\-C\-D\-Number$\ast$ Okno\-Glowne\-::lcd\-Czas\-Sym\hspace{0.3cm}{\ttfamily [private]}}}\label{class_okno_glowne_ab34fefe738e38b1b0d4ce764481cc0c6}
Wskaznik na L\-C\-D z szybkoscia symulacji. Wyswietla jej czas trwania. 

Definicja w linii 369 pliku okno\-\_\-glowne.\-hh.

\hypertarget{class_okno_glowne_adbdd9fc009725804e015d267dc8375dc}{\index{Okno\-Glowne@{Okno\-Glowne}!lcd\-Liczba\-Czasteczek@{lcd\-Liczba\-Czasteczek}}
\index{lcd\-Liczba\-Czasteczek@{lcd\-Liczba\-Czasteczek}!OknoGlowne@{Okno\-Glowne}}
\subsubsection[{lcd\-Liczba\-Czasteczek}]{\setlength{\rightskip}{0pt plus 5cm}Q\-L\-C\-D\-Number$\ast$ Okno\-Glowne\-::lcd\-Liczba\-Czasteczek\hspace{0.3cm}{\ttfamily [private]}}}\label{class_okno_glowne_adbdd9fc009725804e015d267dc8375dc}
Wyswietla liczbe symulowanych czasteczek. 

Definicja w linii 383 pliku okno\-\_\-glowne.\-hh.

\hypertarget{class_okno_glowne_ab100c00d4ba33d896fd0985ac366296a}{\index{Okno\-Glowne@{Okno\-Glowne}!lcd\-Szybkosc\-Sym@{lcd\-Szybkosc\-Sym}}
\index{lcd\-Szybkosc\-Sym@{lcd\-Szybkosc\-Sym}!OknoGlowne@{Okno\-Glowne}}
\subsubsection[{lcd\-Szybkosc\-Sym}]{\setlength{\rightskip}{0pt plus 5cm}Q\-L\-C\-D\-Number$\ast$ Okno\-Glowne\-::lcd\-Szybkosc\-Sym\hspace{0.3cm}{\ttfamily [private]}}}\label{class_okno_glowne_ab100c00d4ba33d896fd0985ac366296a}
Wskaznik na L\-C\-D z szybkoscia symulacji. Wyswietla jej szybkosc. 

Definicja w linii 348 pliku okno\-\_\-glowne.\-hh.

\hypertarget{class_okno_glowne_ae615cbd9c9f9ab06b365c4692ff68729}{\index{Okno\-Glowne@{Okno\-Glowne}!licznik\-\_\-plikow@{licznik\-\_\-plikow}}
\index{licznik\-\_\-plikow@{licznik\-\_\-plikow}!OknoGlowne@{Okno\-Glowne}}
\subsubsection[{licznik\-\_\-plikow}]{\setlength{\rightskip}{0pt plus 5cm}int Okno\-Glowne\-::licznik\-\_\-plikow\hspace{0.3cm}{\ttfamily [static]}, {\ttfamily [private]}}}\label{class_okno_glowne_ae615cbd9c9f9ab06b365c4692ff68729}
Sluzy do generowania unikatowych nazw plikow wyjsciowych. 

Definicja w linii 207 pliku okno\-\_\-glowne.\-hh.

\hypertarget{class_okno_glowne_a0112b8be70a26552b03f38fab43a3301}{\index{Okno\-Glowne@{Okno\-Glowne}!line\-Edit@{line\-Edit}}
\index{line\-Edit@{line\-Edit}!OknoGlowne@{Okno\-Glowne}}
\subsubsection[{line\-Edit}]{\setlength{\rightskip}{0pt plus 5cm}Q\-Line\-Edit$\ast$ Okno\-Glowne\-::line\-Edit\hspace{0.3cm}{\ttfamily [private]}}}\label{class_okno_glowne_a0112b8be70a26552b03f38fab43a3301}
Wskazuje linijke do wpisywania tekstu. 

Definicja w linii 390 pliku okno\-\_\-glowne.\-hh.

\hypertarget{class_okno_glowne_accbadc3bc4d418cfe1bce2be61881917}{\index{Okno\-Glowne@{Okno\-Glowne}!load\-Button@{load\-Button}}
\index{load\-Button@{load\-Button}!OknoGlowne@{Okno\-Glowne}}
\subsubsection[{load\-Button}]{\setlength{\rightskip}{0pt plus 5cm}Q\-Push\-Button$\ast$ Okno\-Glowne\-::load\-Button\hspace{0.3cm}{\ttfamily [private]}}}\label{class_okno_glowne_accbadc3bc4d418cfe1bce2be61881917}
Wskaznik na przycisk Wczytaj. Wczytuje symulacje. 

Definicja w linii 320 pliku okno\-\_\-glowne.\-hh.

\hypertarget{class_okno_glowne_a93afadd0ec22ce6a7e29acc5dd2423a2}{\index{Okno\-Glowne@{Okno\-Glowne}!menu\-\_\-\-Edit@{menu\-\_\-\-Edit}}
\index{menu\-\_\-\-Edit@{menu\-\_\-\-Edit}!OknoGlowne@{Okno\-Glowne}}
\subsubsection[{menu\-\_\-\-Edit}]{\setlength{\rightskip}{0pt plus 5cm}Q\-Menu$\ast$ Okno\-Glowne\-::menu\-\_\-\-Edit\hspace{0.3cm}{\ttfamily [private]}}}\label{class_okno_glowne_a93afadd0ec22ce6a7e29acc5dd2423a2}
Wskaznik na akcje przycisku menu Edit. 

Definicja w linii 249 pliku okno\-\_\-glowne.\-hh.

\hypertarget{class_okno_glowne_a1ba162db2d0b06b0f8963e61b3806875}{\index{Okno\-Glowne@{Okno\-Glowne}!menu\-\_\-\-File@{menu\-\_\-\-File}}
\index{menu\-\_\-\-File@{menu\-\_\-\-File}!OknoGlowne@{Okno\-Glowne}}
\subsubsection[{menu\-\_\-\-File}]{\setlength{\rightskip}{0pt plus 5cm}Q\-Menu$\ast$ Okno\-Glowne\-::menu\-\_\-\-File\hspace{0.3cm}{\ttfamily [private]}}}\label{class_okno_glowne_a1ba162db2d0b06b0f8963e61b3806875}
Wskaznik na akcje przycisku menu File. 

Definicja w linii 242 pliku okno\-\_\-glowne.\-hh.

\hypertarget{class_okno_glowne_ab17be6714913af0cdf4e7de7cb6210d1}{\index{Okno\-Glowne@{Okno\-Glowne}!menu\-\_\-\-Help@{menu\-\_\-\-Help}}
\index{menu\-\_\-\-Help@{menu\-\_\-\-Help}!OknoGlowne@{Okno\-Glowne}}
\subsubsection[{menu\-\_\-\-Help}]{\setlength{\rightskip}{0pt plus 5cm}Q\-Menu$\ast$ Okno\-Glowne\-::menu\-\_\-\-Help\hspace{0.3cm}{\ttfamily [private]}}}\label{class_okno_glowne_ab17be6714913af0cdf4e7de7cb6210d1}
Wskaznik na akcje przycisku menu Help. 

Definicja w linii 256 pliku okno\-\_\-glowne.\-hh.

\hypertarget{class_okno_glowne_a5a87098d9d4bd868670f5a5e72023a0a}{\index{Okno\-Glowne@{Okno\-Glowne}!menu\-Bar@{menu\-Bar}}
\index{menu\-Bar@{menu\-Bar}!OknoGlowne@{Okno\-Glowne}}
\subsubsection[{menu\-Bar}]{\setlength{\rightskip}{0pt plus 5cm}Q\-Menu\-Bar$\ast$ Okno\-Glowne\-::menu\-Bar\hspace{0.3cm}{\ttfamily [private]}}}\label{class_okno_glowne_a5a87098d9d4bd868670f5a5e72023a0a}
Wskaznik na pasek menu. 

Definicja w linii 221 pliku okno\-\_\-glowne.\-hh.

\hypertarget{class_okno_glowne_a0dde8df8a49b8f47f17f8e748fd15967}{\index{Okno\-Glowne@{Okno\-Glowne}!pause\-Button@{pause\-Button}}
\index{pause\-Button@{pause\-Button}!OknoGlowne@{Okno\-Glowne}}
\subsubsection[{pause\-Button}]{\setlength{\rightskip}{0pt plus 5cm}Q\-Push\-Button$\ast$ Okno\-Glowne\-::pause\-Button\hspace{0.3cm}{\ttfamily [private]}}}\label{class_okno_glowne_a0dde8df8a49b8f47f17f8e748fd15967}
Wskaznik na przycisk pause. Wstrzymuje symulacje. 

Definicja w linii 298 pliku okno\-\_\-glowne.\-hh.

\hypertarget{class_okno_glowne_a50f936486c1bc3b3278823a8eb90841e}{\index{Okno\-Glowne@{Okno\-Glowne}!play\-Button@{play\-Button}}
\index{play\-Button@{play\-Button}!OknoGlowne@{Okno\-Glowne}}
\subsubsection[{play\-Button}]{\setlength{\rightskip}{0pt plus 5cm}Q\-Push\-Button$\ast$ Okno\-Glowne\-::play\-Button\hspace{0.3cm}{\ttfamily [private]}}}\label{class_okno_glowne_a50f936486c1bc3b3278823a8eb90841e}
Wskaznik na przycisk play. Uruchamia symulacje. 

Definicja w linii 291 pliku okno\-\_\-glowne.\-hh.

\hypertarget{class_okno_glowne_a81e6650fa592f04bf0adc3bebd3346d6}{\index{Okno\-Glowne@{Okno\-Glowne}!save\-Button@{save\-Button}}
\index{save\-Button@{save\-Button}!OknoGlowne@{Okno\-Glowne}}
\subsubsection[{save\-Button}]{\setlength{\rightskip}{0pt plus 5cm}Q\-Push\-Button$\ast$ Okno\-Glowne\-::save\-Button\hspace{0.3cm}{\ttfamily [private]}}}\label{class_okno_glowne_a81e6650fa592f04bf0adc3bebd3346d6}
Wskaznik na przycisk Zapisz. Zapisuje symulacje. 

Definicja w linii 327 pliku okno\-\_\-glowne.\-hh.

\hypertarget{class_okno_glowne_aaee43ea7074cff126b069c60657d698d}{\index{Okno\-Glowne@{Okno\-Glowne}!slider\-Kat\-Obrotu@{slider\-Kat\-Obrotu}}
\index{slider\-Kat\-Obrotu@{slider\-Kat\-Obrotu}!OknoGlowne@{Okno\-Glowne}}
\subsubsection[{slider\-Kat\-Obrotu}]{\setlength{\rightskip}{0pt plus 5cm}Q\-Slider$\ast$ Okno\-Glowne\-::slider\-Kat\-Obrotu\hspace{0.3cm}{\ttfamily [private]}}}\label{class_okno_glowne_aaee43ea7074cff126b069c60657d698d}
Wskaznik na slider. Steruje katem obrotu. 

Definicja w linii 334 pliku okno\-\_\-glowne.\-hh.

\hypertarget{class_okno_glowne_a85328893065393400d5a0344004ca78b}{\index{Okno\-Glowne@{Okno\-Glowne}!slider\-Szybkosc\-Sym@{slider\-Szybkosc\-Sym}}
\index{slider\-Szybkosc\-Sym@{slider\-Szybkosc\-Sym}!OknoGlowne@{Okno\-Glowne}}
\subsubsection[{slider\-Szybkosc\-Sym}]{\setlength{\rightskip}{0pt plus 5cm}Q\-Slider$\ast$ Okno\-Glowne\-::slider\-Szybkosc\-Sym\hspace{0.3cm}{\ttfamily [private]}}}\label{class_okno_glowne_a85328893065393400d5a0344004ca78b}
Wskaznik na slider. Steruje szybkoscia symulacji. 

Definicja w linii 341 pliku okno\-\_\-glowne.\-hh.

\hypertarget{class_okno_glowne_a40a10989bc6b318ac24e2457d7adb53b}{\index{Okno\-Glowne@{Okno\-Glowne}!status\-Bar@{status\-Bar}}
\index{status\-Bar@{status\-Bar}!OknoGlowne@{Okno\-Glowne}}
\subsubsection[{status\-Bar}]{\setlength{\rightskip}{0pt plus 5cm}Q\-Status\-Bar$\ast$ Okno\-Glowne\-::status\-Bar\hspace{0.3cm}{\ttfamily [private]}}}\label{class_okno_glowne_a40a10989bc6b318ac24e2457d7adb53b}
Wskaznik na pasek statusowy. 

Definicja w linii 263 pliku okno\-\_\-glowne.\-hh.

\hypertarget{class_okno_glowne_a3051d73dc0e0a27dc30ada43cc6b63c4}{\index{Okno\-Glowne@{Okno\-Glowne}!stop\-Button@{stop\-Button}}
\index{stop\-Button@{stop\-Button}!OknoGlowne@{Okno\-Glowne}}
\subsubsection[{stop\-Button}]{\setlength{\rightskip}{0pt plus 5cm}Q\-Push\-Button$\ast$ Okno\-Glowne\-::stop\-Button\hspace{0.3cm}{\ttfamily [private]}}}\label{class_okno_glowne_a3051d73dc0e0a27dc30ada43cc6b63c4}
Wskaznik na przycisk stop. Zatrzymuje symulacje. 

Definicja w linii 305 pliku okno\-\_\-glowne.\-hh.

\hypertarget{class_okno_glowne_a6a37dd1f32605092fff7feac712bf429}{\index{Okno\-Glowne@{Okno\-Glowne}!tool\-Bar@{tool\-Bar}}
\index{tool\-Bar@{tool\-Bar}!OknoGlowne@{Okno\-Glowne}}
\subsubsection[{tool\-Bar}]{\setlength{\rightskip}{0pt plus 5cm}Q\-Tool\-Bar$\ast$ Okno\-Glowne\-::tool\-Bar\hspace{0.3cm}{\ttfamily [private]}}}\label{class_okno_glowne_a6a37dd1f32605092fff7feac712bf429}
Wskaznik na pasek narzedziowy. 

Definicja w linii 270 pliku okno\-\_\-glowne.\-hh.

\hypertarget{class_okno_glowne_af2d1275209898ebdd5ab9de8ef78dffd}{\index{Okno\-Glowne@{Okno\-Glowne}!w\-Zbiornik@{w\-Zbiornik}}
\index{w\-Zbiornik@{w\-Zbiornik}!OknoGlowne@{Okno\-Glowne}}
\subsubsection[{w\-Zbiornik}]{\setlength{\rightskip}{0pt plus 5cm}{\bf Zbiornik}$\ast$ Okno\-Glowne\-::w\-Zbiornik\hspace{0.3cm}{\ttfamily [private]}}}\label{class_okno_glowne_af2d1275209898ebdd5ab9de8ef78dffd}
Wskaznik na zbiornik. 

Definicja w linii 214 pliku okno\-\_\-glowne.\-hh.



Dokumentacja dla tej klasy została wygenerowana z plików\-:\begin{DoxyCompactItemize}
\item 
\hyperlink{okno__glowne_8hh}{okno\-\_\-glowne.\-hh}\item 
\hyperlink{moc__okno__glowne_8cpp}{moc\-\_\-okno\-\_\-glowne.\-cpp}\item 
\hyperlink{okno__glowne_8cpp}{okno\-\_\-glowne.\-cpp}\end{DoxyCompactItemize}

\hypertarget{structparams__t}{\section{Dokumentacja struktury params\-\_\-t}
\label{structparams__t}\index{params\-\_\-t@{params\-\_\-t}}
}


{\ttfamily \#include $<$simulation.\-hh$>$}

\subsection*{Atrybuty publiczne}
\begin{DoxyCompactItemize}
\item 
unsigned \hyperlink{structparams__t_a2cecc28f4ca024657cf567047e2aba59}{nframes}
\item 
unsigned \hyperlink{structparams__t_a06a1a567fd5ba13905514227e2bb710a}{npframe}
\item 
float \hyperlink{structparams__t_a27d76064f2ae0cb93a0956027cfcc19b}{h}
\item 
float \hyperlink{structparams__t_a81fc6596e9b1446442ebf3eef2c3fb01}{dt}
\item 
float \hyperlink{structparams__t_a2eb309edb681d0a998f23fc692a73781}{rho0}
\item 
float \hyperlink{structparams__t_a97ee2783cf89cee1151be3250e9054b3}{k}
\item 
float \hyperlink{structparams__t_a971359c29b2f946b477e4a1b3605fa3f}{mu}
\item 
float \hyperlink{structparams__t_a9f3f70c0cdedcb053c9d45c2e41e67b6}{gx}
\item 
float \hyperlink{structparams__t_a0da484b4cc6a542875aa7b92e200f507}{gy}
\item 
float \hyperlink{structparams__t_afe4a59fe43565a71a0a7a155714e2af1}{mass}
\end{DoxyCompactItemize}


\subsection{Opis szczegółowy}


Definicja w linii 29 pliku simulation.\-hh.



\subsection{Dokumentacja atrybutów składowych}
\hypertarget{structparams__t_a81fc6596e9b1446442ebf3eef2c3fb01}{\index{params\-\_\-t@{params\-\_\-t}!dt@{dt}}
\index{dt@{dt}!params_t@{params\-\_\-t}}
\subsubsection[{dt}]{\setlength{\rightskip}{0pt plus 5cm}float params\-\_\-t\-::dt}}\label{structparams__t_a81fc6596e9b1446442ebf3eef2c3fb01}


Definicja w linii 33 pliku simulation.\-hh.

\hypertarget{structparams__t_a9f3f70c0cdedcb053c9d45c2e41e67b6}{\index{params\-\_\-t@{params\-\_\-t}!gx@{gx}}
\index{gx@{gx}!params_t@{params\-\_\-t}}
\subsubsection[{gx}]{\setlength{\rightskip}{0pt plus 5cm}float params\-\_\-t\-::gx}}\label{structparams__t_a9f3f70c0cdedcb053c9d45c2e41e67b6}


Definicja w linii 37 pliku simulation.\-hh.

\hypertarget{structparams__t_a0da484b4cc6a542875aa7b92e200f507}{\index{params\-\_\-t@{params\-\_\-t}!gy@{gy}}
\index{gy@{gy}!params_t@{params\-\_\-t}}
\subsubsection[{gy}]{\setlength{\rightskip}{0pt plus 5cm}float params\-\_\-t\-::gy}}\label{structparams__t_a0da484b4cc6a542875aa7b92e200f507}


Definicja w linii 38 pliku simulation.\-hh.

\hypertarget{structparams__t_a27d76064f2ae0cb93a0956027cfcc19b}{\index{params\-\_\-t@{params\-\_\-t}!h@{h}}
\index{h@{h}!params_t@{params\-\_\-t}}
\subsubsection[{h}]{\setlength{\rightskip}{0pt plus 5cm}float params\-\_\-t\-::h}}\label{structparams__t_a27d76064f2ae0cb93a0956027cfcc19b}


Definicja w linii 32 pliku simulation.\-hh.

\hypertarget{structparams__t_a97ee2783cf89cee1151be3250e9054b3}{\index{params\-\_\-t@{params\-\_\-t}!k@{k}}
\index{k@{k}!params_t@{params\-\_\-t}}
\subsubsection[{k}]{\setlength{\rightskip}{0pt plus 5cm}float params\-\_\-t\-::k}}\label{structparams__t_a97ee2783cf89cee1151be3250e9054b3}


Definicja w linii 35 pliku simulation.\-hh.

\hypertarget{structparams__t_afe4a59fe43565a71a0a7a155714e2af1}{\index{params\-\_\-t@{params\-\_\-t}!mass@{mass}}
\index{mass@{mass}!params_t@{params\-\_\-t}}
\subsubsection[{mass}]{\setlength{\rightskip}{0pt plus 5cm}float params\-\_\-t\-::mass}}\label{structparams__t_afe4a59fe43565a71a0a7a155714e2af1}


Definicja w linii 39 pliku simulation.\-hh.

\hypertarget{structparams__t_a971359c29b2f946b477e4a1b3605fa3f}{\index{params\-\_\-t@{params\-\_\-t}!mu@{mu}}
\index{mu@{mu}!params_t@{params\-\_\-t}}
\subsubsection[{mu}]{\setlength{\rightskip}{0pt plus 5cm}float params\-\_\-t\-::mu}}\label{structparams__t_a971359c29b2f946b477e4a1b3605fa3f}


Definicja w linii 36 pliku simulation.\-hh.

\hypertarget{structparams__t_a2cecc28f4ca024657cf567047e2aba59}{\index{params\-\_\-t@{params\-\_\-t}!nframes@{nframes}}
\index{nframes@{nframes}!params_t@{params\-\_\-t}}
\subsubsection[{nframes}]{\setlength{\rightskip}{0pt plus 5cm}unsigned params\-\_\-t\-::nframes}}\label{structparams__t_a2cecc28f4ca024657cf567047e2aba59}


Definicja w linii 30 pliku simulation.\-hh.

\hypertarget{structparams__t_a06a1a567fd5ba13905514227e2bb710a}{\index{params\-\_\-t@{params\-\_\-t}!npframe@{npframe}}
\index{npframe@{npframe}!params_t@{params\-\_\-t}}
\subsubsection[{npframe}]{\setlength{\rightskip}{0pt plus 5cm}unsigned params\-\_\-t\-::npframe}}\label{structparams__t_a06a1a567fd5ba13905514227e2bb710a}


Definicja w linii 31 pliku simulation.\-hh.

\hypertarget{structparams__t_a2eb309edb681d0a998f23fc692a73781}{\index{params\-\_\-t@{params\-\_\-t}!rho0@{rho0}}
\index{rho0@{rho0}!params_t@{params\-\_\-t}}
\subsubsection[{rho0}]{\setlength{\rightskip}{0pt plus 5cm}float params\-\_\-t\-::rho0}}\label{structparams__t_a2eb309edb681d0a998f23fc692a73781}


Definicja w linii 34 pliku simulation.\-hh.



Dokumentacja dla tej struktury została wygenerowana z pliku\-:\begin{DoxyCompactItemize}
\item 
\hyperlink{simulation_8hh}{simulation.\-hh}\end{DoxyCompactItemize}

\hypertarget{classsimulation}{\section{Dokumentacja klasy simulation}
\label{classsimulation}\index{simulation@{simulation}}
}


Diagram współpracy dla simulation\-:\nopagebreak
\begin{figure}[H]
\begin{center}
\leavevmode
\includegraphics[width=199pt]{classsimulation__coll__graph}
\end{center}
\end{figure}
\subsection*{Typy publiczne}
\begin{DoxyCompactItemize}
\item 
typedef bool(simulation\-::$\ast$ \hyperlink{classsimulation_ab63cd8ab861eb5f0096982628dca37dc}{indicate\-\_\-fun\-\_\-t} )(float, float)
\end{DoxyCompactItemize}
\subsection*{Metody publiczne}
\begin{DoxyCompactItemize}
\item 
\hyperlink{classsimulation_a30fdcc611ae9a656a3aebb8445e4607d}{simulation} (const unsigned \&\-\_\-n, \hyperlink{structparams__t}{params\-\_\-t} $\ast$\-\_\-param)
\begin{DoxyCompactList}\small\item\em konstruktor klasy \end{DoxyCompactList}\item 
void \hyperlink{classsimulation_a311b1405c73090e5f82891e35ddee975}{compute\-\_\-density} (\hyperlink{structparams__t}{params\-\_\-t} $\ast$\-\_\-p)
\begin{DoxyCompactList}\small\item\em funkcja \end{DoxyCompactList}\item 
void \hyperlink{classsimulation_a2ef9f4e969452551babc02a311f4442c}{compute\-\_\-accel} (\hyperlink{structparams__t}{params\-\_\-t} $\ast$\hyperlink{classsimulation_a861b82cc3c0e7e58abfba464a133dae3}{params})
\item 
void \hyperlink{classsimulation_a2668bf704b28e3c0abe405ab12bd10b7}{damp\-\_\-reflect\-\_\-x} (float \-\_\-barrier, unsigned i)
\item 
void \hyperlink{classsimulation_a1322c3c317601efbe751747f19f5cafa}{damp\-\_\-reflect\-\_\-y} (float \-\_\-barrier, unsigned i)
\item 
void \hyperlink{classsimulation_a6f0b34625a4ef21e7b9c7c960bb91c4b}{reflect\-\_\-bc} ()
\item 
void \hyperlink{classsimulation_af9dbe76734d2efdbddc7da64eeb91968}{leapfrog\-\_\-start} (float dt)
\item 
void \hyperlink{classsimulation_a8acf4a82697573cb4ae6b1be6a8d30a7}{leapfrog\-\_\-step} (float dt)
\item 
void \hyperlink{classsimulation_a6997cb4eeae3330e93916cb05fea5efe}{check\-\_\-state} ()
\item 
bool \hyperlink{classsimulation_a98d5516416b67721c545ae7b88d74114}{box\-\_\-indicator} (float x, float y)
\item 
\hyperlink{classsimulation}{simulation} \& \hyperlink{classsimulation_a66aa4b9fae7d660028ed8972ebf20a91}{place\-\_\-particles} (\hyperlink{structparams__t}{params\-\_\-t} $\ast$\hyperlink{classsimulation_a861b82cc3c0e7e58abfba464a133dae3}{params}, \hyperlink{classsimulation_ab63cd8ab861eb5f0096982628dca37dc}{indicate\-\_\-fun\-\_\-t} indicate\-\_\-fun)
\item 
void \hyperlink{classsimulation_a2e12616e089dd7f9742c25933bf630ef}{init} ()
\item 
void \hyperlink{classsimulation_a6d1112d250e89120ac05cf9f42f7d8d7}{go} ()
\item 
const unsigned \& \hyperlink{classsimulation_abe09252527aa58fbec8144063be3e950}{get\-N} () const 
\end{DoxyCompactItemize}
\subsection*{Atrybuty prywatne}
\begin{DoxyCompactItemize}
\item 
unsigned \hyperlink{classsimulation_a22eb97765a5c60adf3d995f7a110da70}{n}
\item 
float $\ast$ \hyperlink{classsimulation_a44081d4edd92e17a3e1067b976031a00}{rho}
\begin{DoxyCompactList}\small\item\em Liczba cząstek. \end{DoxyCompactList}\item 
\hyperlink{class_vector}{Vector} $\ast$ \hyperlink{classsimulation_a5412fd01febe99f12ae38e30eb692ff0}{p}
\begin{DoxyCompactList}\small\item\em Gęstości. \end{DoxyCompactList}\item 
\hyperlink{class_vector}{Vector} $\ast$ \hyperlink{classsimulation_ae6da1f15728f49be7b0793700866ede9}{vh}
\begin{DoxyCompactList}\small\item\em Pozycje. \end{DoxyCompactList}\item 
\hyperlink{class_vector}{Vector} $\ast$ \hyperlink{classsimulation_a39dbad79b1b8667840638a35e839a3f7}{v}
\begin{DoxyCompactList}\small\item\em Prędkości (half step) \end{DoxyCompactList}\item 
\hyperlink{class_vector}{Vector} $\ast$ \hyperlink{classsimulation_a7b5ca0e5fc096989be7966a73c360b7f}{a}
\begin{DoxyCompactList}\small\item\em Prędkości (full step) \end{DoxyCompactList}\item 
\hyperlink{structparams__t}{params\-\_\-t} $\ast$ \hyperlink{classsimulation_a861b82cc3c0e7e58abfba464a133dae3}{params}
\begin{DoxyCompactList}\small\item\em Przyspieszenia. \end{DoxyCompactList}\end{DoxyCompactItemize}
\subsection*{Przyjaciele}
\begin{DoxyCompactItemize}
\item 
std\-::ostream \& \hyperlink{classsimulation_a1f6414b078a2823f5cea77fc1235e1d9}{operator$<$$<$} (std\-::ostream \&\-\_\-os, const \hyperlink{classsimulation}{simulation} \&\-\_\-s)
\end{DoxyCompactItemize}


\subsection{Opis szczegółowy}


Definicja w linii 162 pliku simulation.\-cpp.



\subsection{Dokumentacja składowych definicji typu}
\hypertarget{classsimulation_ab63cd8ab861eb5f0096982628dca37dc}{\index{simulation@{simulation}!indicate\-\_\-fun\-\_\-t@{indicate\-\_\-fun\-\_\-t}}
\index{indicate\-\_\-fun\-\_\-t@{indicate\-\_\-fun\-\_\-t}!simulation@{simulation}}
\subsubsection[{indicate\-\_\-fun\-\_\-t}]{\setlength{\rightskip}{0pt plus 5cm}typedef bool(simulation\-::$\ast$ simulation\-::indicate\-\_\-fun\-\_\-t)(float, float)}}\label{classsimulation_ab63cd8ab861eb5f0096982628dca37dc}


Definicja w linii 314 pliku simulation.\-cpp.



\subsection{Dokumentacja konstruktora i destruktora}
\hypertarget{classsimulation_a30fdcc611ae9a656a3aebb8445e4607d}{\index{simulation@{simulation}!simulation@{simulation}}
\index{simulation@{simulation}!simulation@{simulation}}
\subsubsection[{simulation}]{\setlength{\rightskip}{0pt plus 5cm}simulation\-::simulation (
\begin{DoxyParamCaption}
\item[{const unsigned \&}]{\-\_\-n, }
\item[{{\bf params\-\_\-t} $\ast$}]{\-\_\-param}
\end{DoxyParamCaption}
)\hspace{0.3cm}{\ttfamily [inline]}}}\label{classsimulation_a30fdcc611ae9a656a3aebb8445e4607d}
Alokuje pamięć na potrzeby przechowania stanu symulacji 
\begin{DoxyParams}[1]{Parametry}
\mbox{\tt in}  & {\em rozmiar} & \\
\hline
\end{DoxyParams}


Definicja w linii 181 pliku simulation.\-cpp.



\subsection{Dokumentacja funkcji składowych}
\hypertarget{classsimulation_a98d5516416b67721c545ae7b88d74114}{\index{simulation@{simulation}!box\-\_\-indicator@{box\-\_\-indicator}}
\index{box\-\_\-indicator@{box\-\_\-indicator}!simulation@{simulation}}
\subsubsection[{box\-\_\-indicator}]{\setlength{\rightskip}{0pt plus 5cm}bool simulation\-::box\-\_\-indicator (
\begin{DoxyParamCaption}
\item[{float}]{x, }
\item[{float}]{y}
\end{DoxyParamCaption}
)\hspace{0.3cm}{\ttfamily [inline]}}}\label{classsimulation_a98d5516416b67721c545ae7b88d74114}


Definicja w linii 316 pliku simulation.\-cpp.



Oto graf wywoływań tej funkcji\-:\nopagebreak
\begin{figure}[H]
\begin{center}
\leavevmode
\includegraphics[width=350pt]{classsimulation_a98d5516416b67721c545ae7b88d74114_icgraph}
\end{center}
\end{figure}


\hypertarget{classsimulation_a6997cb4eeae3330e93916cb05fea5efe}{\index{simulation@{simulation}!check\-\_\-state@{check\-\_\-state}}
\index{check\-\_\-state@{check\-\_\-state}!simulation@{simulation}}
\subsubsection[{check\-\_\-state}]{\setlength{\rightskip}{0pt plus 5cm}void simulation\-::check\-\_\-state (
\begin{DoxyParamCaption}
{}
\end{DoxyParamCaption}
)\hspace{0.3cm}{\ttfamily [inline]}}}\label{classsimulation_a6997cb4eeae3330e93916cb05fea5efe}


Definicja w linii 306 pliku simulation.\-cpp.



Oto graf wywoływań tej funkcji\-:\nopagebreak
\begin{figure}[H]
\begin{center}
\leavevmode
\includegraphics[width=350pt]{classsimulation_a6997cb4eeae3330e93916cb05fea5efe_icgraph}
\end{center}
\end{figure}


\hypertarget{classsimulation_a2ef9f4e969452551babc02a311f4442c}{\index{simulation@{simulation}!compute\-\_\-accel@{compute\-\_\-accel}}
\index{compute\-\_\-accel@{compute\-\_\-accel}!simulation@{simulation}}
\subsubsection[{compute\-\_\-accel}]{\setlength{\rightskip}{0pt plus 5cm}void simulation\-::compute\-\_\-accel (
\begin{DoxyParamCaption}
\item[{{\bf params\-\_\-t} $\ast$}]{params}
\end{DoxyParamCaption}
)\hspace{0.3cm}{\ttfamily [inline]}}}\label{classsimulation_a2ef9f4e969452551babc02a311f4442c}


Definicja w linii 213 pliku simulation.\-cpp.



Oto graf wywołań dla tej funkcji\-:\nopagebreak
\begin{figure}[H]
\begin{center}
\leavevmode
\includegraphics[width=350pt]{classsimulation_a2ef9f4e969452551babc02a311f4442c_cgraph}
\end{center}
\end{figure}




Oto graf wywoływań tej funkcji\-:\nopagebreak
\begin{figure}[H]
\begin{center}
\leavevmode
\includegraphics[width=350pt]{classsimulation_a2ef9f4e969452551babc02a311f4442c_icgraph}
\end{center}
\end{figure}


\hypertarget{classsimulation_a311b1405c73090e5f82891e35ddee975}{\index{simulation@{simulation}!compute\-\_\-density@{compute\-\_\-density}}
\index{compute\-\_\-density@{compute\-\_\-density}!simulation@{simulation}}
\subsubsection[{compute\-\_\-density}]{\setlength{\rightskip}{0pt plus 5cm}void simulation\-::compute\-\_\-density (
\begin{DoxyParamCaption}
\item[{{\bf params\-\_\-t} $\ast$}]{\-\_\-p}
\end{DoxyParamCaption}
)\hspace{0.3cm}{\ttfamily [inline]}}}\label{classsimulation_a311b1405c73090e5f82891e35ddee975}
\[ |I| = \left| \alpha \right| \] 

Definicja w linii 191 pliku simulation.\-cpp.



Oto graf wywołań dla tej funkcji\-:\nopagebreak
\begin{figure}[H]
\begin{center}
\leavevmode
\includegraphics[width=330pt]{classsimulation_a311b1405c73090e5f82891e35ddee975_cgraph}
\end{center}
\end{figure}




Oto graf wywoływań tej funkcji\-:\nopagebreak
\begin{figure}[H]
\begin{center}
\leavevmode
\includegraphics[width=350pt]{classsimulation_a311b1405c73090e5f82891e35ddee975_icgraph}
\end{center}
\end{figure}


\hypertarget{classsimulation_a2668bf704b28e3c0abe405ab12bd10b7}{\index{simulation@{simulation}!damp\-\_\-reflect\-\_\-x@{damp\-\_\-reflect\-\_\-x}}
\index{damp\-\_\-reflect\-\_\-x@{damp\-\_\-reflect\-\_\-x}!simulation@{simulation}}
\subsubsection[{damp\-\_\-reflect\-\_\-x}]{\setlength{\rightskip}{0pt plus 5cm}void simulation\-::damp\-\_\-reflect\-\_\-x (
\begin{DoxyParamCaption}
\item[{float}]{\-\_\-barrier, }
\item[{unsigned}]{i}
\end{DoxyParamCaption}
)\hspace{0.3cm}{\ttfamily [inline]}}}\label{classsimulation_a2668bf704b28e3c0abe405ab12bd10b7}


Definicja w linii 252 pliku simulation.\-cpp.



Oto graf wywołań dla tej funkcji\-:\nopagebreak
\begin{figure}[H]
\begin{center}
\leavevmode
\includegraphics[width=322pt]{classsimulation_a2668bf704b28e3c0abe405ab12bd10b7_cgraph}
\end{center}
\end{figure}




Oto graf wywoływań tej funkcji\-:\nopagebreak
\begin{figure}[H]
\begin{center}
\leavevmode
\includegraphics[width=350pt]{classsimulation_a2668bf704b28e3c0abe405ab12bd10b7_icgraph}
\end{center}
\end{figure}


\hypertarget{classsimulation_a1322c3c317601efbe751747f19f5cafa}{\index{simulation@{simulation}!damp\-\_\-reflect\-\_\-y@{damp\-\_\-reflect\-\_\-y}}
\index{damp\-\_\-reflect\-\_\-y@{damp\-\_\-reflect\-\_\-y}!simulation@{simulation}}
\subsubsection[{damp\-\_\-reflect\-\_\-y}]{\setlength{\rightskip}{0pt plus 5cm}void simulation\-::damp\-\_\-reflect\-\_\-y (
\begin{DoxyParamCaption}
\item[{float}]{\-\_\-barrier, }
\item[{unsigned}]{i}
\end{DoxyParamCaption}
)\hspace{0.3cm}{\ttfamily [inline]}}}\label{classsimulation_a1322c3c317601efbe751747f19f5cafa}


Definicja w linii 265 pliku simulation.\-cpp.



Oto graf wywołań dla tej funkcji\-:\nopagebreak
\begin{figure}[H]
\begin{center}
\leavevmode
\includegraphics[width=322pt]{classsimulation_a1322c3c317601efbe751747f19f5cafa_cgraph}
\end{center}
\end{figure}




Oto graf wywoływań tej funkcji\-:\nopagebreak
\begin{figure}[H]
\begin{center}
\leavevmode
\includegraphics[width=350pt]{classsimulation_a1322c3c317601efbe751747f19f5cafa_icgraph}
\end{center}
\end{figure}


\hypertarget{classsimulation_abe09252527aa58fbec8144063be3e950}{\index{simulation@{simulation}!get\-N@{get\-N}}
\index{get\-N@{get\-N}!simulation@{simulation}}
\subsubsection[{get\-N}]{\setlength{\rightskip}{0pt plus 5cm}const unsigned\& simulation\-::get\-N (
\begin{DoxyParamCaption}
{}
\end{DoxyParamCaption}
) const\hspace{0.3cm}{\ttfamily [inline]}}}\label{classsimulation_abe09252527aa58fbec8144063be3e950}


Definicja w linii 356 pliku simulation.\-cpp.



Oto graf wywoływań tej funkcji\-:\nopagebreak
\begin{figure}[H]
\begin{center}
\leavevmode
\includegraphics[width=278pt]{classsimulation_abe09252527aa58fbec8144063be3e950_icgraph}
\end{center}
\end{figure}


\hypertarget{classsimulation_a6d1112d250e89120ac05cf9f42f7d8d7}{\index{simulation@{simulation}!go@{go}}
\index{go@{go}!simulation@{simulation}}
\subsubsection[{go}]{\setlength{\rightskip}{0pt plus 5cm}void simulation\-::go (
\begin{DoxyParamCaption}
{}
\end{DoxyParamCaption}
)\hspace{0.3cm}{\ttfamily [inline]}}}\label{classsimulation_a6d1112d250e89120ac05cf9f42f7d8d7}


Definicja w linii 349 pliku simulation.\-cpp.



Oto graf wywołań dla tej funkcji\-:\nopagebreak
\begin{figure}[H]
\begin{center}
\leavevmode
\includegraphics[width=350pt]{classsimulation_a6d1112d250e89120ac05cf9f42f7d8d7_cgraph}
\end{center}
\end{figure}


\hypertarget{classsimulation_a2e12616e089dd7f9742c25933bf630ef}{\index{simulation@{simulation}!init@{init}}
\index{init@{init}!simulation@{simulation}}
\subsubsection[{init}]{\setlength{\rightskip}{0pt plus 5cm}void simulation\-::init (
\begin{DoxyParamCaption}
{}
\end{DoxyParamCaption}
)\hspace{0.3cm}{\ttfamily [inline]}}}\label{classsimulation_a2e12616e089dd7f9742c25933bf630ef}


Definicja w linii 343 pliku simulation.\-cpp.



Oto graf wywołań dla tej funkcji\-:\nopagebreak
\begin{figure}[H]
\begin{center}
\leavevmode
\includegraphics[width=350pt]{classsimulation_a2e12616e089dd7f9742c25933bf630ef_cgraph}
\end{center}
\end{figure}




Oto graf wywoływań tej funkcji\-:\nopagebreak
\begin{figure}[H]
\begin{center}
\leavevmode
\includegraphics[width=270pt]{classsimulation_a2e12616e089dd7f9742c25933bf630ef_icgraph}
\end{center}
\end{figure}


\hypertarget{classsimulation_af9dbe76734d2efdbddc7da64eeb91968}{\index{simulation@{simulation}!leapfrog\-\_\-start@{leapfrog\-\_\-start}}
\index{leapfrog\-\_\-start@{leapfrog\-\_\-start}!simulation@{simulation}}
\subsubsection[{leapfrog\-\_\-start}]{\setlength{\rightskip}{0pt plus 5cm}void simulation\-::leapfrog\-\_\-start (
\begin{DoxyParamCaption}
\item[{float}]{dt}
\end{DoxyParamCaption}
)\hspace{0.3cm}{\ttfamily [inline]}}}\label{classsimulation_af9dbe76734d2efdbddc7da64eeb91968}


Definicja w linii 289 pliku simulation.\-cpp.



Oto graf wywołań dla tej funkcji\-:\nopagebreak
\begin{figure}[H]
\begin{center}
\leavevmode
\includegraphics[width=350pt]{classsimulation_af9dbe76734d2efdbddc7da64eeb91968_cgraph}
\end{center}
\end{figure}




Oto graf wywoływań tej funkcji\-:\nopagebreak
\begin{figure}[H]
\begin{center}
\leavevmode
\includegraphics[width=350pt]{classsimulation_af9dbe76734d2efdbddc7da64eeb91968_icgraph}
\end{center}
\end{figure}


\hypertarget{classsimulation_a8acf4a82697573cb4ae6b1be6a8d30a7}{\index{simulation@{simulation}!leapfrog\-\_\-step@{leapfrog\-\_\-step}}
\index{leapfrog\-\_\-step@{leapfrog\-\_\-step}!simulation@{simulation}}
\subsubsection[{leapfrog\-\_\-step}]{\setlength{\rightskip}{0pt plus 5cm}void simulation\-::leapfrog\-\_\-step (
\begin{DoxyParamCaption}
\item[{float}]{dt}
\end{DoxyParamCaption}
)\hspace{0.3cm}{\ttfamily [inline]}}}\label{classsimulation_a8acf4a82697573cb4ae6b1be6a8d30a7}


Definicja w linii 297 pliku simulation.\-cpp.



Oto graf wywołań dla tej funkcji\-:\nopagebreak
\begin{figure}[H]
\begin{center}
\leavevmode
\includegraphics[width=350pt]{classsimulation_a8acf4a82697573cb4ae6b1be6a8d30a7_cgraph}
\end{center}
\end{figure}




Oto graf wywoływań tej funkcji\-:\nopagebreak
\begin{figure}[H]
\begin{center}
\leavevmode
\includegraphics[width=320pt]{classsimulation_a8acf4a82697573cb4ae6b1be6a8d30a7_icgraph}
\end{center}
\end{figure}


\hypertarget{classsimulation_a66aa4b9fae7d660028ed8972ebf20a91}{\index{simulation@{simulation}!place\-\_\-particles@{place\-\_\-particles}}
\index{place\-\_\-particles@{place\-\_\-particles}!simulation@{simulation}}
\subsubsection[{place\-\_\-particles}]{\setlength{\rightskip}{0pt plus 5cm}{\bf simulation}\& simulation\-::place\-\_\-particles (
\begin{DoxyParamCaption}
\item[{{\bf params\-\_\-t} $\ast$}]{params, }
\item[{{\bf indicate\-\_\-fun\-\_\-t}}]{indicate\-\_\-fun}
\end{DoxyParamCaption}
)\hspace{0.3cm}{\ttfamily [inline]}}}\label{classsimulation_a66aa4b9fae7d660028ed8972ebf20a91}


Definicja w linii 320 pliku simulation.\-cpp.



Oto graf wywołań dla tej funkcji\-:\nopagebreak
\begin{figure}[H]
\begin{center}
\leavevmode
\includegraphics[width=350pt]{classsimulation_a66aa4b9fae7d660028ed8972ebf20a91_cgraph}
\end{center}
\end{figure}




Oto graf wywoływań tej funkcji\-:\nopagebreak
\begin{figure}[H]
\begin{center}
\leavevmode
\includegraphics[width=350pt]{classsimulation_a66aa4b9fae7d660028ed8972ebf20a91_icgraph}
\end{center}
\end{figure}


\hypertarget{classsimulation_a6f0b34625a4ef21e7b9c7c960bb91c4b}{\index{simulation@{simulation}!reflect\-\_\-bc@{reflect\-\_\-bc}}
\index{reflect\-\_\-bc@{reflect\-\_\-bc}!simulation@{simulation}}
\subsubsection[{reflect\-\_\-bc}]{\setlength{\rightskip}{0pt plus 5cm}void simulation\-::reflect\-\_\-bc (
\begin{DoxyParamCaption}
{}
\end{DoxyParamCaption}
)\hspace{0.3cm}{\ttfamily [inline]}}}\label{classsimulation_a6f0b34625a4ef21e7b9c7c960bb91c4b}


Definicja w linii 279 pliku simulation.\-cpp.



Oto graf wywołań dla tej funkcji\-:\nopagebreak
\begin{figure}[H]
\begin{center}
\leavevmode
\includegraphics[width=350pt]{classsimulation_a6f0b34625a4ef21e7b9c7c960bb91c4b_cgraph}
\end{center}
\end{figure}




Oto graf wywoływań tej funkcji\-:\nopagebreak
\begin{figure}[H]
\begin{center}
\leavevmode
\includegraphics[width=350pt]{classsimulation_a6f0b34625a4ef21e7b9c7c960bb91c4b_icgraph}
\end{center}
\end{figure}




\subsection{Dokumentacja przyjaciół i funkcji związanych}
\hypertarget{classsimulation_a1f6414b078a2823f5cea77fc1235e1d9}{\index{simulation@{simulation}!operator$<$$<$@{operator$<$$<$}}
\index{operator$<$$<$@{operator$<$$<$}!simulation@{simulation}}
\subsubsection[{operator$<$$<$}]{\setlength{\rightskip}{0pt plus 5cm}std\-::ostream\& operator$<$$<$ (
\begin{DoxyParamCaption}
\item[{std\-::ostream \&}]{\-\_\-os, }
\item[{const {\bf simulation} \&}]{\-\_\-s}
\end{DoxyParamCaption}
)\hspace{0.3cm}{\ttfamily [friend]}}}\label{classsimulation_a1f6414b078a2823f5cea77fc1235e1d9}


Definicja w linii 360 pliku simulation.\-cpp.



\subsection{Dokumentacja atrybutów składowych}
\hypertarget{classsimulation_a7b5ca0e5fc096989be7966a73c360b7f}{\index{simulation@{simulation}!a@{a}}
\index{a@{a}!simulation@{simulation}}
\subsubsection[{a}]{\setlength{\rightskip}{0pt plus 5cm}{\bf Vector}$\ast$ simulation\-::a\hspace{0.3cm}{\ttfamily [private]}}}\label{classsimulation_a7b5ca0e5fc096989be7966a73c360b7f}


Definicja w linii 170 pliku simulation.\-cpp.

\hypertarget{classsimulation_a22eb97765a5c60adf3d995f7a110da70}{\index{simulation@{simulation}!n@{n}}
\index{n@{n}!simulation@{simulation}}
\subsubsection[{n}]{\setlength{\rightskip}{0pt plus 5cm}unsigned simulation\-::n\hspace{0.3cm}{\ttfamily [private]}}}\label{classsimulation_a22eb97765a5c60adf3d995f7a110da70}


Definicja w linii 165 pliku simulation.\-cpp.

\hypertarget{classsimulation_a5412fd01febe99f12ae38e30eb692ff0}{\index{simulation@{simulation}!p@{p}}
\index{p@{p}!simulation@{simulation}}
\subsubsection[{p}]{\setlength{\rightskip}{0pt plus 5cm}{\bf Vector}$\ast$ simulation\-::p\hspace{0.3cm}{\ttfamily [private]}}}\label{classsimulation_a5412fd01febe99f12ae38e30eb692ff0}


Definicja w linii 167 pliku simulation.\-cpp.

\hypertarget{classsimulation_a861b82cc3c0e7e58abfba464a133dae3}{\index{simulation@{simulation}!params@{params}}
\index{params@{params}!simulation@{simulation}}
\subsubsection[{params}]{\setlength{\rightskip}{0pt plus 5cm}{\bf params\-\_\-t}$\ast$ simulation\-::params\hspace{0.3cm}{\ttfamily [private]}}}\label{classsimulation_a861b82cc3c0e7e58abfba464a133dae3}


Definicja w linii 172 pliku simulation.\-cpp.

\hypertarget{classsimulation_a44081d4edd92e17a3e1067b976031a00}{\index{simulation@{simulation}!rho@{rho}}
\index{rho@{rho}!simulation@{simulation}}
\subsubsection[{rho}]{\setlength{\rightskip}{0pt plus 5cm}float$\ast$ simulation\-::rho\hspace{0.3cm}{\ttfamily [private]}}}\label{classsimulation_a44081d4edd92e17a3e1067b976031a00}


Definicja w linii 166 pliku simulation.\-cpp.

\hypertarget{classsimulation_a39dbad79b1b8667840638a35e839a3f7}{\index{simulation@{simulation}!v@{v}}
\index{v@{v}!simulation@{simulation}}
\subsubsection[{v}]{\setlength{\rightskip}{0pt plus 5cm}{\bf Vector}$\ast$ simulation\-::v\hspace{0.3cm}{\ttfamily [private]}}}\label{classsimulation_a39dbad79b1b8667840638a35e839a3f7}


Definicja w linii 169 pliku simulation.\-cpp.

\hypertarget{classsimulation_ae6da1f15728f49be7b0793700866ede9}{\index{simulation@{simulation}!vh@{vh}}
\index{vh@{vh}!simulation@{simulation}}
\subsubsection[{vh}]{\setlength{\rightskip}{0pt plus 5cm}{\bf Vector}$\ast$ simulation\-::vh\hspace{0.3cm}{\ttfamily [private]}}}\label{classsimulation_ae6da1f15728f49be7b0793700866ede9}


Definicja w linii 168 pliku simulation.\-cpp.



Dokumentacja dla tej klasy została wygenerowana z pliku\-:\begin{DoxyCompactItemize}
\item 
\hyperlink{simulation_8cpp}{simulation.\-cpp}\end{DoxyCompactItemize}

\hypertarget{class_ui___d_main_window}{\section{Dokumentacja klasy Ui\-\_\-\-D\-Main\-Window}
\label{class_ui___d_main_window}\index{Ui\-\_\-\-D\-Main\-Window@{Ui\-\_\-\-D\-Main\-Window}}
}


{\ttfamily \#include $<$ui\-\_\-dmainwindow.\-h$>$}



Diagram dziedziczenia dla Ui\-\_\-\-D\-Main\-Window
\nopagebreak
\begin{figure}[H]
\begin{center}
\leavevmode
\includegraphics[width=176pt]{class_ui___d_main_window__inherit__graph}
\end{center}
\end{figure}
\subsection*{Metody publiczne}
\begin{DoxyCompactItemize}
\item 
void \hyperlink{class_ui___d_main_window_aa53f3a89bf520704a3e79037df2fd451}{setup\-Ui} (Q\-Main\-Window $\ast$\hyperlink{class_d_main_window}{D\-Main\-Window})
\item 
void \hyperlink{class_ui___d_main_window_a406169c751ddfd205b89375c7542827c}{retranslate\-Ui} (Q\-Main\-Window $\ast$\hyperlink{class_d_main_window}{D\-Main\-Window})
\end{DoxyCompactItemize}
\subsection*{Atrybuty publiczne}
\begin{DoxyCompactItemize}
\item 
Q\-Action $\ast$ \hyperlink{class_ui___d_main_window_a7ab98279e07bdd724a091ea06012c87b}{action\-\_\-\-Save}
\item 
Q\-Action $\ast$ \hyperlink{class_ui___d_main_window_a00e6b795743b676bdf3ed853e91f7029}{action\-\_\-\-Exit}
\item 
Q\-Action $\ast$ \hyperlink{class_ui___d_main_window_ae1fa62a4d27fa0f4a5c63c7c60cfdad2}{action\-Exit}
\item 
Q\-Action $\ast$ \hyperlink{class_ui___d_main_window_a6cfb6311ca1dd6e247d80255e2667ba7}{action\-Play}
\item 
Q\-Widget $\ast$ \hyperlink{class_ui___d_main_window_a94cf40cb4e645cfa2e80f36ffbf5018e}{central\-Widget}
\item 
Q\-Widget $\ast$ \hyperlink{class_ui___d_main_window_a777a56f3b74aa5b5cd5ff2c62a2968a9}{horizontal\-Layout\-Widget}
\item 
Q\-H\-Box\-Layout $\ast$ \hyperlink{class_ui___d_main_window_a4ab6ff85d8c5edef531b3f2111a04157}{horizontal\-Layout}
\item 
Q\-Push\-Button $\ast$ \hyperlink{class_ui___d_main_window_ad87cbf39ac14374923ed2a2b11e8b1bf}{play\-Button}
\item 
Q\-Push\-Button $\ast$ \hyperlink{class_ui___d_main_window_a70e142e35db4995a1fefa082406bdef3}{pause\-Button}
\item 
Q\-Push\-Button $\ast$ \hyperlink{class_ui___d_main_window_a1fe7797fff349a0f0d47d90c8438f386}{stop\-Button}
\item 
Q\-Menu\-Bar $\ast$ \hyperlink{class_ui___d_main_window_a788ef749d82ca070e467e55cca0d47dd}{menu\-Bar}
\item 
Q\-Menu $\ast$ \hyperlink{class_ui___d_main_window_a991f4d15852faf04d8c12694b3e077ad}{menu\-\_\-\-File}
\item 
Q\-Menu $\ast$ \hyperlink{class_ui___d_main_window_a8826a3e34a5aa75fca2b8e45b7010a8b}{menu\-\_\-\-Edit}
\item 
Q\-Menu $\ast$ \hyperlink{class_ui___d_main_window_ac2997077098614d72b21d29c7a48350c}{menu\-\_\-\-Help}
\item 
Q\-Tool\-Bar $\ast$ \hyperlink{class_ui___d_main_window_a2e1da3781ee1e5913b25b85f4c29b97f}{main\-Tool\-Bar}
\item 
Q\-Status\-Bar $\ast$ \hyperlink{class_ui___d_main_window_ac9e025e7279839dd7ab1686456d1ae21}{status\-Bar}
\item 
Q\-Tool\-Bar $\ast$ \hyperlink{class_ui___d_main_window_abba1dae1dd835c7a7dd39da623cd4580}{tool\-Bar}
\end{DoxyCompactItemize}


\subsection{Opis szczegółowy}


Definicja w linii 28 pliku ui\-\_\-dmainwindow.\-h.



\subsection{Dokumentacja funkcji składowych}
\hypertarget{class_ui___d_main_window_a406169c751ddfd205b89375c7542827c}{\index{Ui\-\_\-\-D\-Main\-Window@{Ui\-\_\-\-D\-Main\-Window}!retranslate\-Ui@{retranslate\-Ui}}
\index{retranslate\-Ui@{retranslate\-Ui}!Ui_DMainWindow@{Ui\-\_\-\-D\-Main\-Window}}
\subsubsection[{retranslate\-Ui}]{\setlength{\rightskip}{0pt plus 5cm}void Ui\-\_\-\-D\-Main\-Window\-::retranslate\-Ui (
\begin{DoxyParamCaption}
\item[{Q\-Main\-Window $\ast$}]{D\-Main\-Window}
\end{DoxyParamCaption}
)\hspace{0.3cm}{\ttfamily [inline]}}}\label{class_ui___d_main_window_a406169c751ddfd205b89375c7542827c}


Definicja w linii 119 pliku ui\-\_\-dmainwindow.\-h.



Oto graf wywoływań tej funkcji\-:
\nopagebreak
\begin{figure}[H]
\begin{center}
\leavevmode
\includegraphics[width=350pt]{class_ui___d_main_window_a406169c751ddfd205b89375c7542827c_icgraph}
\end{center}
\end{figure}


\hypertarget{class_ui___d_main_window_aa53f3a89bf520704a3e79037df2fd451}{\index{Ui\-\_\-\-D\-Main\-Window@{Ui\-\_\-\-D\-Main\-Window}!setup\-Ui@{setup\-Ui}}
\index{setup\-Ui@{setup\-Ui}!Ui_DMainWindow@{Ui\-\_\-\-D\-Main\-Window}}
\subsubsection[{setup\-Ui}]{\setlength{\rightskip}{0pt plus 5cm}void Ui\-\_\-\-D\-Main\-Window\-::setup\-Ui (
\begin{DoxyParamCaption}
\item[{Q\-Main\-Window $\ast$}]{D\-Main\-Window}
\end{DoxyParamCaption}
)\hspace{0.3cm}{\ttfamily [inline]}}}\label{class_ui___d_main_window_aa53f3a89bf520704a3e79037df2fd451}


Definicja w linii 49 pliku ui\-\_\-dmainwindow.\-h.



Oto graf wywołań dla tej funkcji\-:
\nopagebreak
\begin{figure}[H]
\begin{center}
\leavevmode
\includegraphics[width=350pt]{class_ui___d_main_window_aa53f3a89bf520704a3e79037df2fd451_cgraph}
\end{center}
\end{figure}




Oto graf wywoływań tej funkcji\-:
\nopagebreak
\begin{figure}[H]
\begin{center}
\leavevmode
\includegraphics[width=350pt]{class_ui___d_main_window_aa53f3a89bf520704a3e79037df2fd451_icgraph}
\end{center}
\end{figure}




\subsection{Dokumentacja atrybutów składowych}
\hypertarget{class_ui___d_main_window_a00e6b795743b676bdf3ed853e91f7029}{\index{Ui\-\_\-\-D\-Main\-Window@{Ui\-\_\-\-D\-Main\-Window}!action\-\_\-\-Exit@{action\-\_\-\-Exit}}
\index{action\-\_\-\-Exit@{action\-\_\-\-Exit}!Ui_DMainWindow@{Ui\-\_\-\-D\-Main\-Window}}
\subsubsection[{action\-\_\-\-Exit}]{\setlength{\rightskip}{0pt plus 5cm}Q\-Action$\ast$ Ui\-\_\-\-D\-Main\-Window\-::action\-\_\-\-Exit}}\label{class_ui___d_main_window_a00e6b795743b676bdf3ed853e91f7029}


Definicja w linii 32 pliku ui\-\_\-dmainwindow.\-h.

\hypertarget{class_ui___d_main_window_a7ab98279e07bdd724a091ea06012c87b}{\index{Ui\-\_\-\-D\-Main\-Window@{Ui\-\_\-\-D\-Main\-Window}!action\-\_\-\-Save@{action\-\_\-\-Save}}
\index{action\-\_\-\-Save@{action\-\_\-\-Save}!Ui_DMainWindow@{Ui\-\_\-\-D\-Main\-Window}}
\subsubsection[{action\-\_\-\-Save}]{\setlength{\rightskip}{0pt plus 5cm}Q\-Action$\ast$ Ui\-\_\-\-D\-Main\-Window\-::action\-\_\-\-Save}}\label{class_ui___d_main_window_a7ab98279e07bdd724a091ea06012c87b}


Definicja w linii 31 pliku ui\-\_\-dmainwindow.\-h.

\hypertarget{class_ui___d_main_window_ae1fa62a4d27fa0f4a5c63c7c60cfdad2}{\index{Ui\-\_\-\-D\-Main\-Window@{Ui\-\_\-\-D\-Main\-Window}!action\-Exit@{action\-Exit}}
\index{action\-Exit@{action\-Exit}!Ui_DMainWindow@{Ui\-\_\-\-D\-Main\-Window}}
\subsubsection[{action\-Exit}]{\setlength{\rightskip}{0pt plus 5cm}Q\-Action$\ast$ Ui\-\_\-\-D\-Main\-Window\-::action\-Exit}}\label{class_ui___d_main_window_ae1fa62a4d27fa0f4a5c63c7c60cfdad2}


Definicja w linii 33 pliku ui\-\_\-dmainwindow.\-h.

\hypertarget{class_ui___d_main_window_a6cfb6311ca1dd6e247d80255e2667ba7}{\index{Ui\-\_\-\-D\-Main\-Window@{Ui\-\_\-\-D\-Main\-Window}!action\-Play@{action\-Play}}
\index{action\-Play@{action\-Play}!Ui_DMainWindow@{Ui\-\_\-\-D\-Main\-Window}}
\subsubsection[{action\-Play}]{\setlength{\rightskip}{0pt plus 5cm}Q\-Action$\ast$ Ui\-\_\-\-D\-Main\-Window\-::action\-Play}}\label{class_ui___d_main_window_a6cfb6311ca1dd6e247d80255e2667ba7}


Definicja w linii 34 pliku ui\-\_\-dmainwindow.\-h.

\hypertarget{class_ui___d_main_window_a94cf40cb4e645cfa2e80f36ffbf5018e}{\index{Ui\-\_\-\-D\-Main\-Window@{Ui\-\_\-\-D\-Main\-Window}!central\-Widget@{central\-Widget}}
\index{central\-Widget@{central\-Widget}!Ui_DMainWindow@{Ui\-\_\-\-D\-Main\-Window}}
\subsubsection[{central\-Widget}]{\setlength{\rightskip}{0pt plus 5cm}Q\-Widget$\ast$ Ui\-\_\-\-D\-Main\-Window\-::central\-Widget}}\label{class_ui___d_main_window_a94cf40cb4e645cfa2e80f36ffbf5018e}


Definicja w linii 35 pliku ui\-\_\-dmainwindow.\-h.

\hypertarget{class_ui___d_main_window_a4ab6ff85d8c5edef531b3f2111a04157}{\index{Ui\-\_\-\-D\-Main\-Window@{Ui\-\_\-\-D\-Main\-Window}!horizontal\-Layout@{horizontal\-Layout}}
\index{horizontal\-Layout@{horizontal\-Layout}!Ui_DMainWindow@{Ui\-\_\-\-D\-Main\-Window}}
\subsubsection[{horizontal\-Layout}]{\setlength{\rightskip}{0pt plus 5cm}Q\-H\-Box\-Layout$\ast$ Ui\-\_\-\-D\-Main\-Window\-::horizontal\-Layout}}\label{class_ui___d_main_window_a4ab6ff85d8c5edef531b3f2111a04157}


Definicja w linii 37 pliku ui\-\_\-dmainwindow.\-h.

\hypertarget{class_ui___d_main_window_a777a56f3b74aa5b5cd5ff2c62a2968a9}{\index{Ui\-\_\-\-D\-Main\-Window@{Ui\-\_\-\-D\-Main\-Window}!horizontal\-Layout\-Widget@{horizontal\-Layout\-Widget}}
\index{horizontal\-Layout\-Widget@{horizontal\-Layout\-Widget}!Ui_DMainWindow@{Ui\-\_\-\-D\-Main\-Window}}
\subsubsection[{horizontal\-Layout\-Widget}]{\setlength{\rightskip}{0pt plus 5cm}Q\-Widget$\ast$ Ui\-\_\-\-D\-Main\-Window\-::horizontal\-Layout\-Widget}}\label{class_ui___d_main_window_a777a56f3b74aa5b5cd5ff2c62a2968a9}


Definicja w linii 36 pliku ui\-\_\-dmainwindow.\-h.

\hypertarget{class_ui___d_main_window_a2e1da3781ee1e5913b25b85f4c29b97f}{\index{Ui\-\_\-\-D\-Main\-Window@{Ui\-\_\-\-D\-Main\-Window}!main\-Tool\-Bar@{main\-Tool\-Bar}}
\index{main\-Tool\-Bar@{main\-Tool\-Bar}!Ui_DMainWindow@{Ui\-\_\-\-D\-Main\-Window}}
\subsubsection[{main\-Tool\-Bar}]{\setlength{\rightskip}{0pt plus 5cm}Q\-Tool\-Bar$\ast$ Ui\-\_\-\-D\-Main\-Window\-::main\-Tool\-Bar}}\label{class_ui___d_main_window_a2e1da3781ee1e5913b25b85f4c29b97f}


Definicja w linii 45 pliku ui\-\_\-dmainwindow.\-h.

\hypertarget{class_ui___d_main_window_a8826a3e34a5aa75fca2b8e45b7010a8b}{\index{Ui\-\_\-\-D\-Main\-Window@{Ui\-\_\-\-D\-Main\-Window}!menu\-\_\-\-Edit@{menu\-\_\-\-Edit}}
\index{menu\-\_\-\-Edit@{menu\-\_\-\-Edit}!Ui_DMainWindow@{Ui\-\_\-\-D\-Main\-Window}}
\subsubsection[{menu\-\_\-\-Edit}]{\setlength{\rightskip}{0pt plus 5cm}Q\-Menu$\ast$ Ui\-\_\-\-D\-Main\-Window\-::menu\-\_\-\-Edit}}\label{class_ui___d_main_window_a8826a3e34a5aa75fca2b8e45b7010a8b}


Definicja w linii 43 pliku ui\-\_\-dmainwindow.\-h.

\hypertarget{class_ui___d_main_window_a991f4d15852faf04d8c12694b3e077ad}{\index{Ui\-\_\-\-D\-Main\-Window@{Ui\-\_\-\-D\-Main\-Window}!menu\-\_\-\-File@{menu\-\_\-\-File}}
\index{menu\-\_\-\-File@{menu\-\_\-\-File}!Ui_DMainWindow@{Ui\-\_\-\-D\-Main\-Window}}
\subsubsection[{menu\-\_\-\-File}]{\setlength{\rightskip}{0pt plus 5cm}Q\-Menu$\ast$ Ui\-\_\-\-D\-Main\-Window\-::menu\-\_\-\-File}}\label{class_ui___d_main_window_a991f4d15852faf04d8c12694b3e077ad}


Definicja w linii 42 pliku ui\-\_\-dmainwindow.\-h.

\hypertarget{class_ui___d_main_window_ac2997077098614d72b21d29c7a48350c}{\index{Ui\-\_\-\-D\-Main\-Window@{Ui\-\_\-\-D\-Main\-Window}!menu\-\_\-\-Help@{menu\-\_\-\-Help}}
\index{menu\-\_\-\-Help@{menu\-\_\-\-Help}!Ui_DMainWindow@{Ui\-\_\-\-D\-Main\-Window}}
\subsubsection[{menu\-\_\-\-Help}]{\setlength{\rightskip}{0pt plus 5cm}Q\-Menu$\ast$ Ui\-\_\-\-D\-Main\-Window\-::menu\-\_\-\-Help}}\label{class_ui___d_main_window_ac2997077098614d72b21d29c7a48350c}


Definicja w linii 44 pliku ui\-\_\-dmainwindow.\-h.

\hypertarget{class_ui___d_main_window_a788ef749d82ca070e467e55cca0d47dd}{\index{Ui\-\_\-\-D\-Main\-Window@{Ui\-\_\-\-D\-Main\-Window}!menu\-Bar@{menu\-Bar}}
\index{menu\-Bar@{menu\-Bar}!Ui_DMainWindow@{Ui\-\_\-\-D\-Main\-Window}}
\subsubsection[{menu\-Bar}]{\setlength{\rightskip}{0pt plus 5cm}Q\-Menu\-Bar$\ast$ Ui\-\_\-\-D\-Main\-Window\-::menu\-Bar}}\label{class_ui___d_main_window_a788ef749d82ca070e467e55cca0d47dd}


Definicja w linii 41 pliku ui\-\_\-dmainwindow.\-h.

\hypertarget{class_ui___d_main_window_a70e142e35db4995a1fefa082406bdef3}{\index{Ui\-\_\-\-D\-Main\-Window@{Ui\-\_\-\-D\-Main\-Window}!pause\-Button@{pause\-Button}}
\index{pause\-Button@{pause\-Button}!Ui_DMainWindow@{Ui\-\_\-\-D\-Main\-Window}}
\subsubsection[{pause\-Button}]{\setlength{\rightskip}{0pt plus 5cm}Q\-Push\-Button$\ast$ Ui\-\_\-\-D\-Main\-Window\-::pause\-Button}}\label{class_ui___d_main_window_a70e142e35db4995a1fefa082406bdef3}


Definicja w linii 39 pliku ui\-\_\-dmainwindow.\-h.

\hypertarget{class_ui___d_main_window_ad87cbf39ac14374923ed2a2b11e8b1bf}{\index{Ui\-\_\-\-D\-Main\-Window@{Ui\-\_\-\-D\-Main\-Window}!play\-Button@{play\-Button}}
\index{play\-Button@{play\-Button}!Ui_DMainWindow@{Ui\-\_\-\-D\-Main\-Window}}
\subsubsection[{play\-Button}]{\setlength{\rightskip}{0pt plus 5cm}Q\-Push\-Button$\ast$ Ui\-\_\-\-D\-Main\-Window\-::play\-Button}}\label{class_ui___d_main_window_ad87cbf39ac14374923ed2a2b11e8b1bf}


Definicja w linii 38 pliku ui\-\_\-dmainwindow.\-h.

\hypertarget{class_ui___d_main_window_ac9e025e7279839dd7ab1686456d1ae21}{\index{Ui\-\_\-\-D\-Main\-Window@{Ui\-\_\-\-D\-Main\-Window}!status\-Bar@{status\-Bar}}
\index{status\-Bar@{status\-Bar}!Ui_DMainWindow@{Ui\-\_\-\-D\-Main\-Window}}
\subsubsection[{status\-Bar}]{\setlength{\rightskip}{0pt plus 5cm}Q\-Status\-Bar$\ast$ Ui\-\_\-\-D\-Main\-Window\-::status\-Bar}}\label{class_ui___d_main_window_ac9e025e7279839dd7ab1686456d1ae21}


Definicja w linii 46 pliku ui\-\_\-dmainwindow.\-h.

\hypertarget{class_ui___d_main_window_a1fe7797fff349a0f0d47d90c8438f386}{\index{Ui\-\_\-\-D\-Main\-Window@{Ui\-\_\-\-D\-Main\-Window}!stop\-Button@{stop\-Button}}
\index{stop\-Button@{stop\-Button}!Ui_DMainWindow@{Ui\-\_\-\-D\-Main\-Window}}
\subsubsection[{stop\-Button}]{\setlength{\rightskip}{0pt plus 5cm}Q\-Push\-Button$\ast$ Ui\-\_\-\-D\-Main\-Window\-::stop\-Button}}\label{class_ui___d_main_window_a1fe7797fff349a0f0d47d90c8438f386}


Definicja w linii 40 pliku ui\-\_\-dmainwindow.\-h.

\hypertarget{class_ui___d_main_window_abba1dae1dd835c7a7dd39da623cd4580}{\index{Ui\-\_\-\-D\-Main\-Window@{Ui\-\_\-\-D\-Main\-Window}!tool\-Bar@{tool\-Bar}}
\index{tool\-Bar@{tool\-Bar}!Ui_DMainWindow@{Ui\-\_\-\-D\-Main\-Window}}
\subsubsection[{tool\-Bar}]{\setlength{\rightskip}{0pt plus 5cm}Q\-Tool\-Bar$\ast$ Ui\-\_\-\-D\-Main\-Window\-::tool\-Bar}}\label{class_ui___d_main_window_abba1dae1dd835c7a7dd39da623cd4580}


Definicja w linii 47 pliku ui\-\_\-dmainwindow.\-h.



Dokumentacja dla tej klasy została wygenerowana z pliku\-:\begin{DoxyCompactItemize}
\item 
\hyperlink{ui__dmainwindow_8h}{ui\-\_\-dmainwindow.\-h}\end{DoxyCompactItemize}

\hypertarget{class_vector}{\section{Dokumentacja klasy Vector}
\label{class_vector}\index{Vector@{Vector}}
}


klasa \hyperlink{class_vector}{Vector}  




{\ttfamily \#include $<$vector.\-hh$>$}

\subsection*{Metody publiczne}
\begin{DoxyCompactItemize}
\item 
\hyperlink{class_vector_af3c1b04bfbb10e29433842202365a6c4}{Vector} (float \-\_\-x=0, float \-\_\-y=0)
\begin{DoxyCompactList}\small\item\em Współrzędne wektora. \end{DoxyCompactList}\item 
float \hyperlink{class_vector_ab2878a1bb81982dc83363646e25ce665}{get\-X} () const 
\begin{DoxyCompactList}\small\item\em zwraga pierwszą współrzędną wektora \end{DoxyCompactList}\item 
float \hyperlink{class_vector_a86293fe7a035979fd252be6071488b6a}{get\-Y} () const 
\begin{DoxyCompactList}\small\item\em pobiera drugą współrzędną wektora \end{DoxyCompactList}\item 
float \& \hyperlink{class_vector_aeca06c929d4ab3078a828723a88621e6}{get\-X} ()
\begin{DoxyCompactList}\small\item\em pobiera pierwszą współrzędną wektora \end{DoxyCompactList}\item 
float \& \hyperlink{class_vector_ab0cc77ce300a60de0ab734555886ad5d}{get\-Y} ()
\begin{DoxyCompactList}\small\item\em pobiera drugą współrzędną wektora \end{DoxyCompactList}\item 
\hyperlink{class_vector}{Vector} \& \hyperlink{class_vector_a4eeab5be24ee846de3012e67a4e34820}{operator+=} (const \hyperlink{class_vector}{Vector} \&\-\_\-v)
\begin{DoxyCompactList}\small\item\em operator dodawania \end{DoxyCompactList}\item 
\hyperlink{class_vector}{Vector} \& \hyperlink{class_vector_aaaf87dbf15cd9492aa0c11874ae5afef}{operator-\/=} (const \hyperlink{class_vector}{Vector} \&\-\_\-v)
\begin{DoxyCompactList}\small\item\em operator odejmowania \end{DoxyCompactList}\item 
\hyperlink{class_vector}{Vector} \& \hyperlink{class_vector_a91ebac6d502ca1d54645e7c711549867}{operator$\ast$=} (const float \&\-\_\-c)
\begin{DoxyCompactList}\small\item\em operator skalowania \end{DoxyCompactList}\item 
\hyperlink{class_vector}{Vector} \hyperlink{class_vector_aa78eb4c9e5ac236c89f0853eefa347ac}{operator+} (const \hyperlink{class_vector}{Vector} \&\-\_\-v) const 
\begin{DoxyCompactList}\small\item\em operator dodawania \end{DoxyCompactList}\item 
\hyperlink{class_vector}{Vector} \hyperlink{class_vector_a94b6fde82bef6532c00358a0af448fc1}{operator-\/} (const \hyperlink{class_vector}{Vector} \&\-\_\-v) const 
\begin{DoxyCompactList}\small\item\em operator odejmowania \end{DoxyCompactList}\item 
\hyperlink{class_vector}{Vector} \hyperlink{class_vector_a8f0e64ee9a688803b1efce30fb0b2869}{operator$\ast$} (const float \&\-\_\-c) const 
\begin{DoxyCompactList}\small\item\em operator skalowania \end{DoxyCompactList}\item 
\hyperlink{class_vector}{Vector} \& \hyperlink{class_vector_ad44f6d9721d9584e7f847e449df73e11}{operator=} (const \hyperlink{class_vector}{Vector} \&\-\_\-v)
\begin{DoxyCompactList}\small\item\em operator przypisania \end{DoxyCompactList}\item 
float \hyperlink{class_vector_a18d3f2110be751ac3a658016bd3dca69}{norm\-Squared} ()
\begin{DoxyCompactList}\small\item\em kwadrat normy \end{DoxyCompactList}\end{DoxyCompactItemize}
\subsection*{Atrybuty prywatne}
\begin{DoxyCompactItemize}
\item 
float \hyperlink{class_vector_aca49165049a1e21ae47afcfc078819ed}{x}
\item 
float \hyperlink{class_vector_a81be9102fca6d9beea3efef522c4c09d}{y}
\end{DoxyCompactItemize}
\subsection*{Przyjaciele}
\begin{DoxyCompactItemize}
\item 
std\-::ostream \& \hyperlink{class_vector_a7973d0032b0df4c3ab89e1da167b86a9}{operator$<$$<$} (std\-::ostream \&\-\_\-os, const \hyperlink{class_vector}{Vector} \&\-\_\-s)
\end{DoxyCompactItemize}


\subsection{Opis szczegółowy}
Posiada metody do obsługi dwuelementowych wektorów o współrzędnych typu float 

Definicja w linii 19 pliku vector.\-hh.



\subsection{Dokumentacja konstruktora i destruktora}
\hypertarget{class_vector_af3c1b04bfbb10e29433842202365a6c4}{\index{Vector@{Vector}!Vector@{Vector}}
\index{Vector@{Vector}!Vector@{Vector}}
\subsubsection[{Vector}]{\setlength{\rightskip}{0pt plus 5cm}Vector\-::\-Vector (
\begin{DoxyParamCaption}
\item[{float}]{\-\_\-x = {\ttfamily 0}, }
\item[{float}]{\-\_\-y = {\ttfamily 0}}
\end{DoxyParamCaption}
)\hspace{0.3cm}{\ttfamily [inline]}}}\label{class_vector_af3c1b04bfbb10e29433842202365a6c4}
konstruktor wektora

Ustawia współrzędne wektora 
\begin{DoxyParams}[1]{Parametry}
\mbox{\tt in}  & {\em \-\_\-x} & -\/ pierwsza współrzędna (domyślnie\-: 0) \\
\hline
\mbox{\tt in}  & {\em \-\_\-y} & -\/ druga współrzędna (domyślnie\-: 0) \\
\hline
\end{DoxyParams}


Definicja w linii 31 pliku vector.\-hh.



Oto graf wywoływań tej funkcji\-:\nopagebreak
\begin{figure}[H]
\begin{center}
\leavevmode
\includegraphics[width=288pt]{class_vector_af3c1b04bfbb10e29433842202365a6c4_icgraph}
\end{center}
\end{figure}




\subsection{Dokumentacja funkcji składowych}
\hypertarget{class_vector_ab2878a1bb81982dc83363646e25ce665}{\index{Vector@{Vector}!get\-X@{get\-X}}
\index{get\-X@{get\-X}!Vector@{Vector}}
\subsubsection[{get\-X}]{\setlength{\rightskip}{0pt plus 5cm}float Vector\-::get\-X (
\begin{DoxyParamCaption}
{}
\end{DoxyParamCaption}
) const\hspace{0.3cm}{\ttfamily [inline]}}}\label{class_vector_ab2878a1bb81982dc83363646e25ce665}
\begin{DoxyReturn}{Zwraca}
pierwsza współrzędna wektora jako stała 
\end{DoxyReturn}


Definicja w linii 36 pliku vector.\-hh.



Oto graf wywoływań tej funkcji\-:\nopagebreak
\begin{figure}[H]
\begin{center}
\leavevmode
\includegraphics[width=350pt]{class_vector_ab2878a1bb81982dc83363646e25ce665_icgraph}
\end{center}
\end{figure}


\hypertarget{class_vector_aeca06c929d4ab3078a828723a88621e6}{\index{Vector@{Vector}!get\-X@{get\-X}}
\index{get\-X@{get\-X}!Vector@{Vector}}
\subsubsection[{get\-X}]{\setlength{\rightskip}{0pt plus 5cm}float\& Vector\-::get\-X (
\begin{DoxyParamCaption}
{}
\end{DoxyParamCaption}
)\hspace{0.3cm}{\ttfamily [inline]}}}\label{class_vector_aeca06c929d4ab3078a828723a88621e6}
\begin{DoxyReturn}{Zwraca}
pierwsza współrzędna wektora jako stała 
\end{DoxyReturn}


Definicja w linii 46 pliku vector.\-hh.

\hypertarget{class_vector_a86293fe7a035979fd252be6071488b6a}{\index{Vector@{Vector}!get\-Y@{get\-Y}}
\index{get\-Y@{get\-Y}!Vector@{Vector}}
\subsubsection[{get\-Y}]{\setlength{\rightskip}{0pt plus 5cm}float Vector\-::get\-Y (
\begin{DoxyParamCaption}
{}
\end{DoxyParamCaption}
) const\hspace{0.3cm}{\ttfamily [inline]}}}\label{class_vector_a86293fe7a035979fd252be6071488b6a}
\begin{DoxyReturn}{Zwraca}
druga współrzędna wektora jako stała 
\end{DoxyReturn}


Definicja w linii 41 pliku vector.\-hh.



Oto graf wywoływań tej funkcji\-:\nopagebreak
\begin{figure}[H]
\begin{center}
\leavevmode
\includegraphics[width=350pt]{class_vector_a86293fe7a035979fd252be6071488b6a_icgraph}
\end{center}
\end{figure}


\hypertarget{class_vector_ab0cc77ce300a60de0ab734555886ad5d}{\index{Vector@{Vector}!get\-Y@{get\-Y}}
\index{get\-Y@{get\-Y}!Vector@{Vector}}
\subsubsection[{get\-Y}]{\setlength{\rightskip}{0pt plus 5cm}float\& Vector\-::get\-Y (
\begin{DoxyParamCaption}
{}
\end{DoxyParamCaption}
)\hspace{0.3cm}{\ttfamily [inline]}}}\label{class_vector_ab0cc77ce300a60de0ab734555886ad5d}
\begin{DoxyReturn}{Zwraca}
druga współrzędna wektora jako stała 
\end{DoxyReturn}


Definicja w linii 51 pliku vector.\-hh.

\hypertarget{class_vector_a18d3f2110be751ac3a658016bd3dca69}{\index{Vector@{Vector}!norm\-Squared@{norm\-Squared}}
\index{norm\-Squared@{norm\-Squared}!Vector@{Vector}}
\subsubsection[{norm\-Squared}]{\setlength{\rightskip}{0pt plus 5cm}float Vector\-::norm\-Squared (
\begin{DoxyParamCaption}
{}
\end{DoxyParamCaption}
)\hspace{0.3cm}{\ttfamily [inline]}}}\label{class_vector_a18d3f2110be751ac3a658016bd3dca69}
Zwraca sumę kwadratów współrzędnych wektora \begin{DoxyReturn}{Zwraca}
wartość kwadratu normy 
\end{DoxyReturn}


Definicja w linii 114 pliku vector.\-hh.



Oto graf wywoływań tej funkcji\-:\nopagebreak
\begin{figure}[H]
\begin{center}
\leavevmode
\includegraphics[width=350pt]{class_vector_a18d3f2110be751ac3a658016bd3dca69_icgraph}
\end{center}
\end{figure}


\hypertarget{class_vector_a8f0e64ee9a688803b1efce30fb0b2869}{\index{Vector@{Vector}!operator$\ast$@{operator$\ast$}}
\index{operator$\ast$@{operator$\ast$}!Vector@{Vector}}
\subsubsection[{operator$\ast$}]{\setlength{\rightskip}{0pt plus 5cm}{\bf Vector} Vector\-::operator$\ast$ (
\begin{DoxyParamCaption}
\item[{const float \&}]{\-\_\-c}
\end{DoxyParamCaption}
) const\hspace{0.3cm}{\ttfamily [inline]}}}\label{class_vector_a8f0e64ee9a688803b1efce30fb0b2869}
Skaluje wektor 
\begin{DoxyParams}[1]{Parametry}
\mbox{\tt in}  & {\em \-\_\-c} & -\/ współczynnik skalowania \\
\hline
\end{DoxyParams}
\begin{DoxyReturn}{Zwraca}
przeskalowany wektor 
\end{DoxyReturn}


Definicja w linii 99 pliku vector.\-hh.



Oto graf wywołań dla tej funkcji\-:\nopagebreak
\begin{figure}[H]
\begin{center}
\leavevmode
\includegraphics[width=286pt]{class_vector_a8f0e64ee9a688803b1efce30fb0b2869_cgraph}
\end{center}
\end{figure}


\hypertarget{class_vector_a91ebac6d502ca1d54645e7c711549867}{\index{Vector@{Vector}!operator$\ast$=@{operator$\ast$=}}
\index{operator$\ast$=@{operator$\ast$=}!Vector@{Vector}}
\subsubsection[{operator$\ast$=}]{\setlength{\rightskip}{0pt plus 5cm}{\bf Vector}\& Vector\-::operator$\ast$= (
\begin{DoxyParamCaption}
\item[{const float \&}]{\-\_\-c}
\end{DoxyParamCaption}
)\hspace{0.3cm}{\ttfamily [inline]}}}\label{class_vector_a91ebac6d502ca1d54645e7c711549867}
Skaluje wektor 
\begin{DoxyParams}[1]{Parametry}
\mbox{\tt in}  & {\em \-\_\-c} & -\/ współczynnik \\
\hline
\end{DoxyParams}
\begin{DoxyReturn}{Zwraca}
referencja na pierwotny obiekt 
\end{DoxyReturn}


Definicja w linii 75 pliku vector.\-hh.

\hypertarget{class_vector_aa78eb4c9e5ac236c89f0853eefa347ac}{\index{Vector@{Vector}!operator+@{operator+}}
\index{operator+@{operator+}!Vector@{Vector}}
\subsubsection[{operator+}]{\setlength{\rightskip}{0pt plus 5cm}{\bf Vector} Vector\-::operator+ (
\begin{DoxyParamCaption}
\item[{const {\bf Vector} \&}]{\-\_\-v}
\end{DoxyParamCaption}
) const\hspace{0.3cm}{\ttfamily [inline]}}}\label{class_vector_aa78eb4c9e5ac236c89f0853eefa347ac}
Dodaje wartość dwóch wektorów 
\begin{DoxyParams}[1]{Parametry}
\mbox{\tt in}  & {\em \-\_\-v} & -\/ inny wektor \\
\hline
\end{DoxyParams}
\begin{DoxyReturn}{Zwraca}
wektor będący sumą 
\end{DoxyReturn}


Definicja w linii 83 pliku vector.\-hh.



Oto graf wywołań dla tej funkcji\-:\nopagebreak
\begin{figure}[H]
\begin{center}
\leavevmode
\includegraphics[width=288pt]{class_vector_aa78eb4c9e5ac236c89f0853eefa347ac_cgraph}
\end{center}
\end{figure}


\hypertarget{class_vector_a4eeab5be24ee846de3012e67a4e34820}{\index{Vector@{Vector}!operator+=@{operator+=}}
\index{operator+=@{operator+=}!Vector@{Vector}}
\subsubsection[{operator+=}]{\setlength{\rightskip}{0pt plus 5cm}{\bf Vector}\& Vector\-::operator+= (
\begin{DoxyParamCaption}
\item[{const {\bf Vector} \&}]{\-\_\-v}
\end{DoxyParamCaption}
)\hspace{0.3cm}{\ttfamily [inline]}}}\label{class_vector_a4eeab5be24ee846de3012e67a4e34820}
Dodaje wartość innego wektora do klasy 
\begin{DoxyParams}[1]{Parametry}
\mbox{\tt in}  & {\em \-\_\-v} & -\/ inny wektor \\
\hline
\end{DoxyParams}
\begin{DoxyReturn}{Zwraca}
referencja na pierwotny obiekt 
\end{DoxyReturn}


Definicja w linii 59 pliku vector.\-hh.

\hypertarget{class_vector_a94b6fde82bef6532c00358a0af448fc1}{\index{Vector@{Vector}!operator-\/@{operator-\/}}
\index{operator-\/@{operator-\/}!Vector@{Vector}}
\subsubsection[{operator-\/}]{\setlength{\rightskip}{0pt plus 5cm}{\bf Vector} Vector\-::operator-\/ (
\begin{DoxyParamCaption}
\item[{const {\bf Vector} \&}]{\-\_\-v}
\end{DoxyParamCaption}
) const\hspace{0.3cm}{\ttfamily [inline]}}}\label{class_vector_a94b6fde82bef6532c00358a0af448fc1}
Odejmuje wartość innego wektora od klasy 
\begin{DoxyParams}[1]{Parametry}
\mbox{\tt in}  & {\em \-\_\-v} & -\/ inny wektor \\
\hline
\end{DoxyParams}
\begin{DoxyReturn}{Zwraca}
wektor będący różnicą 
\end{DoxyReturn}


Definicja w linii 91 pliku vector.\-hh.



Oto graf wywołań dla tej funkcji\-:\nopagebreak
\begin{figure}[H]
\begin{center}
\leavevmode
\includegraphics[width=286pt]{class_vector_a94b6fde82bef6532c00358a0af448fc1_cgraph}
\end{center}
\end{figure}


\hypertarget{class_vector_aaaf87dbf15cd9492aa0c11874ae5afef}{\index{Vector@{Vector}!operator-\/=@{operator-\/=}}
\index{operator-\/=@{operator-\/=}!Vector@{Vector}}
\subsubsection[{operator-\/=}]{\setlength{\rightskip}{0pt plus 5cm}{\bf Vector}\& Vector\-::operator-\/= (
\begin{DoxyParamCaption}
\item[{const {\bf Vector} \&}]{\-\_\-v}
\end{DoxyParamCaption}
)\hspace{0.3cm}{\ttfamily [inline]}}}\label{class_vector_aaaf87dbf15cd9492aa0c11874ae5afef}
Odejmuje wartość innego wektora od klasy 
\begin{DoxyParams}[1]{Parametry}
\mbox{\tt in}  & {\em \-\_\-v} & -\/ inny wektor \\
\hline
\end{DoxyParams}
\begin{DoxyReturn}{Zwraca}
referencja na pierwotny obiekt 
\end{DoxyReturn}


Definicja w linii 67 pliku vector.\-hh.

\hypertarget{class_vector_ad44f6d9721d9584e7f847e449df73e11}{\index{Vector@{Vector}!operator=@{operator=}}
\index{operator=@{operator=}!Vector@{Vector}}
\subsubsection[{operator=}]{\setlength{\rightskip}{0pt plus 5cm}{\bf Vector}\& Vector\-::operator= (
\begin{DoxyParamCaption}
\item[{const {\bf Vector} \&}]{\-\_\-v}
\end{DoxyParamCaption}
)\hspace{0.3cm}{\ttfamily [inline]}}}\label{class_vector_ad44f6d9721d9584e7f847e449df73e11}
Przypisuje współrzędne innego wektora do klasy 
\begin{DoxyParams}[1]{Parametry}
\mbox{\tt in}  & {\em \-\_\-v} & -\/ inny wektor \\
\hline
\end{DoxyParams}
\begin{DoxyReturn}{Zwraca}
referencja na pierwotny obiekt 
\end{DoxyReturn}


Definicja w linii 107 pliku vector.\-hh.



\subsection{Dokumentacja przyjaciół i funkcji związanych}
\hypertarget{class_vector_a7973d0032b0df4c3ab89e1da167b86a9}{\index{Vector@{Vector}!operator$<$$<$@{operator$<$$<$}}
\index{operator$<$$<$@{operator$<$$<$}!Vector@{Vector}}
\subsubsection[{operator$<$$<$}]{\setlength{\rightskip}{0pt plus 5cm}std\-::ostream\& operator$<$$<$ (
\begin{DoxyParamCaption}
\item[{std\-::ostream \&}]{\-\_\-os, }
\item[{const {\bf Vector} \&}]{\-\_\-s}
\end{DoxyParamCaption}
)\hspace{0.3cm}{\ttfamily [friend]}}}\label{class_vector_a7973d0032b0df4c3ab89e1da167b86a9}


Definicja w linii 11 pliku vector.\-cpp.



\subsection{Dokumentacja atrybutów składowych}
\hypertarget{class_vector_aca49165049a1e21ae47afcfc078819ed}{\index{Vector@{Vector}!x@{x}}
\index{x@{x}!Vector@{Vector}}
\subsubsection[{x}]{\setlength{\rightskip}{0pt plus 5cm}float Vector\-::x\hspace{0.3cm}{\ttfamily [private]}}}\label{class_vector_aca49165049a1e21ae47afcfc078819ed}


Definicja w linii 21 pliku vector.\-hh.

\hypertarget{class_vector_a81be9102fca6d9beea3efef522c4c09d}{\index{Vector@{Vector}!y@{y}}
\index{y@{y}!Vector@{Vector}}
\subsubsection[{y}]{\setlength{\rightskip}{0pt plus 5cm}float Vector\-::y\hspace{0.3cm}{\ttfamily [private]}}}\label{class_vector_a81be9102fca6d9beea3efef522c4c09d}


Definicja w linii 21 pliku vector.\-hh.



Dokumentacja dla tej klasy została wygenerowana z pliku\-:\begin{DoxyCompactItemize}
\item 
\hyperlink{vector_8hh}{vector.\-hh}\end{DoxyCompactItemize}

\hypertarget{class_zbiornik}{\section{Dokumentacja klasy Zbiornik}
\label{class_zbiornik}\index{Zbiornik@{Zbiornik}}
}


{\ttfamily \#include $<$ciecz.\-hh$>$}



Diagram dziedziczenia dla Zbiornik
\nopagebreak
\begin{figure}[H]
\begin{center}
\leavevmode
\includegraphics[width=134pt]{class_zbiornik__inherit__graph}
\end{center}
\end{figure}


Diagram współpracy dla Zbiornik\-:
\nopagebreak
\begin{figure}[H]
\begin{center}
\leavevmode
\includegraphics[width=134pt]{class_zbiornik__coll__graph}
\end{center}
\end{figure}
\subsection*{Metody publiczne}
\begin{DoxyCompactItemize}
\item 
\hyperlink{class_zbiornik_aa432f548eb9dfd62bb7312a745a70cc7}{Zbiornik} (Q\-Widget $\ast$w\-Rodzic=0\-L)
\begin{DoxyCompactList}\small\item\em Konstruktor. \end{DoxyCompactList}\item 
virtual void \hyperlink{class_zbiornik_af7a9c185e95b92de342c6dc69f020765}{paint\-Event} (Q\-Paint\-Event $\ast$event)
\item 
void \hyperlink{class_zbiornik_ae8bda443a783485029083b039e1019e8}{Rysuj\-Zbiornik} (Q\-Painter \&Rysownik, int Podstawa, int Wysokosc, int Grubosc)
\begin{DoxyCompactList}\small\item\em Wyrysowuje zbiornik. \end{DoxyCompactList}\end{DoxyCompactItemize}


\subsection{Opis szczegółowy}


Definicja w linii 54 pliku ciecz.\-hh.



\subsection{Dokumentacja konstruktora i destruktora}
\hypertarget{class_zbiornik_aa432f548eb9dfd62bb7312a745a70cc7}{\index{Zbiornik@{Zbiornik}!Zbiornik@{Zbiornik}}
\index{Zbiornik@{Zbiornik}!Zbiornik@{Zbiornik}}
\subsubsection[{Zbiornik}]{\setlength{\rightskip}{0pt plus 5cm}Zbiornik\-::\-Zbiornik (
\begin{DoxyParamCaption}
\item[{Q\-Widget $\ast$}]{w\-Rodzic = {\ttfamily 0L}}
\end{DoxyParamCaption}
)}}\label{class_zbiornik_aa432f548eb9dfd62bb7312a745a70cc7}

\begin{DoxyParams}[1]{Parametry}
\mbox{\tt in,out}  & {\em w\-Rodzic} & -\/ wskaznik na rodzica \\
\hline
\end{DoxyParams}


Definicja w linii 32 pliku ciecz.\-cpp.



\subsection{Dokumentacja funkcji składowych}
\hypertarget{class_zbiornik_af7a9c185e95b92de342c6dc69f020765}{\index{Zbiornik@{Zbiornik}!paint\-Event@{paint\-Event}}
\index{paint\-Event@{paint\-Event}!Zbiornik@{Zbiornik}}
\subsubsection[{paint\-Event}]{\setlength{\rightskip}{0pt plus 5cm}void Zbiornik\-::paint\-Event (
\begin{DoxyParamCaption}
\item[{Q\-Paint\-Event $\ast$}]{event}
\end{DoxyParamCaption}
)\hspace{0.3cm}{\ttfamily [virtual]}}}\label{class_zbiornik_af7a9c185e95b92de342c6dc69f020765}


Definicja w linii 53 pliku ciecz.\-cpp.



Oto graf wywołań dla tej funkcji\-:
\nopagebreak
\begin{figure}[H]
\begin{center}
\leavevmode
\includegraphics[width=338pt]{class_zbiornik_af7a9c185e95b92de342c6dc69f020765_cgraph}
\end{center}
\end{figure}


\hypertarget{class_zbiornik_ae8bda443a783485029083b039e1019e8}{\index{Zbiornik@{Zbiornik}!Rysuj\-Zbiornik@{Rysuj\-Zbiornik}}
\index{Rysuj\-Zbiornik@{Rysuj\-Zbiornik}!Zbiornik@{Zbiornik}}
\subsubsection[{Rysuj\-Zbiornik}]{\setlength{\rightskip}{0pt plus 5cm}void Zbiornik\-::\-Rysuj\-Zbiornik (
\begin{DoxyParamCaption}
\item[{Q\-Painter \&}]{Rysownik, }
\item[{int}]{Podstawa, }
\item[{int}]{Wysokosc, }
\item[{int}]{Grubosc}
\end{DoxyParamCaption}
)}}\label{class_zbiornik_ae8bda443a783485029083b039e1019e8}

\begin{DoxyParams}[1]{Parametry}
\mbox{\tt in,out}  & {\em Rysownik} & -\/ referencja na obiekt klasy Q\-Painter \\
\hline
\mbox{\tt in}  & {\em Podstawa} & -\/ dlugosc podstawy zbiornika \\
\hline
\mbox{\tt in}  & {\em Wysokosc} & -\/ wysokosc zbiornika \\
\hline
\mbox{\tt in}  & {\em Grubosc} & -\/ grubosc wyrysowywanego odcinka \\
\hline
\end{DoxyParams}


Definicja w linii 38 pliku ciecz.\-cpp.



Oto graf wywoływań tej funkcji\-:
\nopagebreak
\begin{figure}[H]
\begin{center}
\leavevmode
\includegraphics[width=338pt]{class_zbiornik_ae8bda443a783485029083b039e1019e8_icgraph}
\end{center}
\end{figure}




Dokumentacja dla tej klasy została wygenerowana z plików\-:\begin{DoxyCompactItemize}
\item 
\hyperlink{ciecz_8hh}{ciecz.\-hh}\item 
\hyperlink{ciecz_8cpp}{ciecz.\-cpp}\end{DoxyCompactItemize}

\chapter{Dokumentacja plików}
\hypertarget{czasteczka_8cpp}{\section{Dokumentacja pliku czasteczka.\-cpp}
\label{czasteczka_8cpp}\index{czasteczka.\-cpp@{czasteczka.\-cpp}}
}


Zawiera definicje metod klasy \hyperlink{class_czasteczka}{Czasteczka}.  


{\ttfamily \#include \char`\"{}czasteczka.\-hh\char`\"{}}\\*
Wykres zależności załączania dla czasteczka.\-cpp\-:\nopagebreak
\begin{figure}[H]
\begin{center}
\leavevmode
\includegraphics[width=278pt]{czasteczka_8cpp__incl}
\end{center}
\end{figure}


\subsection{Opis szczegółowy}
W pliku znajduja sie\-:
\begin{DoxyItemize}
\item definicje konstruktorow, metod i przeciazen klasy \hyperlink{class_czasteczka}{Czasteczka}. 
\end{DoxyItemize}

Definicja w pliku \hyperlink{czasteczka_8cpp_source}{czasteczka.\-cpp}.


\hypertarget{czasteczka_8hh}{\section{Dokumentacja pliku czasteczka.\-hh}
\label{czasteczka_8hh}\index{czasteczka.\-hh@{czasteczka.\-hh}}
}


Zawiera definicje klasy \hyperlink{class_kolor}{Kolor} oraz deklaracje jej metod.  


{\ttfamily \#include \char`\"{}kolor.\-hh\char`\"{}}\\*
Wykres zależności załączania dla czasteczka.\-hh\-:\nopagebreak
\begin{figure}[H]
\begin{center}
\leavevmode
\includegraphics[width=160pt]{czasteczka_8hh__incl}
\end{center}
\end{figure}
Ten wykres pokazuje, które pliki bezpośrednio lub pośrednio załączają ten plik\-:
\nopagebreak
\begin{figure}[H]
\begin{center}
\leavevmode
\includegraphics[width=350pt]{czasteczka_8hh__dep__incl}
\end{center}
\end{figure}
\subsection*{Komponenty}
\begin{DoxyCompactItemize}
\item 
class \hyperlink{class_czasteczka}{Czasteczka}
\begin{DoxyCompactList}\small\item\em Klasa modelująca czasteczke. \end{DoxyCompactList}\end{DoxyCompactItemize}
\subsection*{Zmienne}
\begin{DoxyCompactItemize}
\item 
const int \hyperlink{czasteczka_8hh_aa77f856f3142a9e81752665a9bc2e6de}{P\-R\-O\-M\-I\-E\-N} = 10
\begin{DoxyCompactList}\small\item\em Promien czasteczki. \end{DoxyCompactList}\end{DoxyCompactItemize}


\subsection{Opis szczegółowy}
W pliku znajduja sie\-:
\begin{DoxyItemize}
\item definicja klasy \hyperlink{class_czasteczka}{Czasteczka} (modeluje pojecie czasteczki),
\item definicje konstruktorow,
\item deklaracje atrybutow. 
\end{DoxyItemize}

Definicja w pliku \hyperlink{czasteczka_8hh_source}{czasteczka.\-hh}.



\subsection{Dokumentacja zmiennych}
\hypertarget{czasteczka_8hh_aa77f856f3142a9e81752665a9bc2e6de}{\index{czasteczka.\-hh@{czasteczka.\-hh}!P\-R\-O\-M\-I\-E\-N@{P\-R\-O\-M\-I\-E\-N}}
\index{P\-R\-O\-M\-I\-E\-N@{P\-R\-O\-M\-I\-E\-N}!czasteczka.hh@{czasteczka.\-hh}}
\subsubsection[{P\-R\-O\-M\-I\-E\-N}]{\setlength{\rightskip}{0pt plus 5cm}const int P\-R\-O\-M\-I\-E\-N = 10}}\label{czasteczka_8hh_aa77f856f3142a9e81752665a9bc2e6de}
Promien czasteczki. 

Definicja w linii 26 pliku czasteczka.\-hh.


\hypertarget{dmainwindow_8cpp}{\section{Dokumentacja pliku dmainwindow.\-cpp}
\label{dmainwindow_8cpp}\index{dmainwindow.\-cpp@{dmainwindow.\-cpp}}
}
{\ttfamily \#include \char`\"{}dmainwindow.\-h\char`\"{}}\\*
{\ttfamily \#include \char`\"{}ui\-\_\-dmainwindow.\-h\char`\"{}}\\*
Wykres zależności załączania dla dmainwindow.\-cpp\-:
\nopagebreak
\begin{figure}[H]
\begin{center}
\leavevmode
\includegraphics[width=350pt]{dmainwindow_8cpp__incl}
\end{center}
\end{figure}

\hypertarget{dmainwindow_8h}{\section{Dokumentacja pliku dmainwindow.\-h}
\label{dmainwindow_8h}\index{dmainwindow.\-h@{dmainwindow.\-h}}
}
{\ttfamily \#include $<$Q\-Main\-Window$>$}\\*
Wykres zależności załączania dla dmainwindow.\-h\-:\nopagebreak
\begin{figure}[H]
\begin{center}
\leavevmode
\includegraphics[width=162pt]{dmainwindow_8h__incl}
\end{center}
\end{figure}
Ten wykres pokazuje, które pliki bezpośrednio lub pośrednio załączają ten plik\-:\nopagebreak
\begin{figure}[H]
\begin{center}
\leavevmode
\includegraphics[width=350pt]{dmainwindow_8h__dep__incl}
\end{center}
\end{figure}
\subsection*{Komponenty}
\begin{DoxyCompactItemize}
\item 
class \hyperlink{class_d_main_window}{D\-Main\-Window}
\end{DoxyCompactItemize}
\subsection*{Przestrzenie nazw}
\begin{DoxyCompactItemize}
\item 
\hyperlink{namespace_ui}{Ui}
\end{DoxyCompactItemize}

\hypertarget{flagi_8hh}{}\section{Dokumentacja pliku flagi.\+hh}
\label{flagi_8hh}\index{flagi.\+hh@{flagi.\+hh}}


Zawiera globalne zmienne opisujace symulacje.  


Ten wykres pokazuje, które pliki bezpośrednio lub pośrednio załączają ten plik\+:\nopagebreak
\begin{figure}[H]
\begin{center}
\leavevmode
\includegraphics[width=350pt]{flagi_8hh__dep__incl}
\end{center}
\end{figure}
\subsection*{Wyliczenia}
\begin{DoxyCompactItemize}
\item 
enum \{ \hyperlink{flagi_8hh_a06fc87d81c62e9abb8790b6e5713c55ba4957581ee0386c284fd318121e335af6}{e\+S\+T\+O\+P}, 
\hyperlink{flagi_8hh_a06fc87d81c62e9abb8790b6e5713c55ba6a2f69efc37338427ecd0db296923a79}{e\+P\+A\+U\+S\+E}, 
\hyperlink{flagi_8hh_a06fc87d81c62e9abb8790b6e5713c55baecdbae639704e0b7e4b5478734e45b8d}{e\+P\+L\+A\+Y}
 \}
\begin{DoxyCompactList}\small\item\em Typ wyliczeniowy okreslajacy stany symulacji. \end{DoxyCompactList}\end{DoxyCompactItemize}
\subsection*{Zmienne}
\begin{DoxyCompactItemize}
\item 
int \hyperlink{flagi_8hh_ae3a120c63186a17e4127a68187b3e9e8}{S\+T\+A\+N}
\begin{DoxyCompactList}\small\item\em Stan symulacji. \end{DoxyCompactList}\item 
const int \hyperlink{flagi_8hh_aa77f856f3142a9e81752665a9bc2e6de}{P\+R\+O\+M\+I\+E\+N} = 8
\begin{DoxyCompactList}\small\item\em Promien czasteczki \mbox{[}px\mbox{]}. \end{DoxyCompactList}\item 
const int \hyperlink{flagi_8hh_acd3c5814c051e565bf7854f6403acf49}{P\+O\+D\+S\+T\+A\+W\+A} = 280
\begin{DoxyCompactList}\small\item\em Dlugosc podstawy zbiornika \mbox{[}px\mbox{]}. \end{DoxyCompactList}\item 
const int \hyperlink{flagi_8hh_a073767f0ac7dbf009a42b00de1092b52}{W\+Y\+S\+O\+K\+O\+S\+C} = 280
\begin{DoxyCompactList}\small\item\em Wysokosc zbiornika \mbox{[}px\mbox{]}. \end{DoxyCompactList}\item 
const int \hyperlink{flagi_8hh_a359a95636f17b8e9b7a01389d75b521d}{G\+R\+U\+B\+O\+S\+C} = 8
\begin{DoxyCompactList}\small\item\em Grubosc krawedzi zbiornika \mbox{[}px\mbox{]}. \end{DoxyCompactList}\item 
const int \hyperlink{flagi_8hh_afa380d01dc08ee237b4eea9046704397}{P\+A\+S\+K\+I} = 8
\begin{DoxyCompactList}\small\item\em Suma grubosci paskow statusu itd. okienka \mbox{[}px\mbox{]}. \end{DoxyCompactList}\end{DoxyCompactItemize}


\subsection{Opis szczegółowy}
W pliku znajduja sie\+:
\begin{DoxyItemize}
\item zmienne opisujaca stan symulacji,
\item promien czasteczek,
\item zalozone wymiary zbiornika, 
\end{DoxyItemize}

\subsection{Dokumentacja typów wyliczanych}
\hypertarget{flagi_8hh_a06fc87d81c62e9abb8790b6e5713c55b}{}\subsubsection[{anonymous enum}]{\setlength{\rightskip}{0pt plus 5cm}anonymous enum}\label{flagi_8hh_a06fc87d81c62e9abb8790b6e5713c55b}
Stany symulacji to Play/\+Pause/\+Stop. \begin{Desc}
\item[Wartości wyliczeń]\par
\begin{description}
\index{e\+S\+T\+O\+P@{e\+S\+T\+O\+P}!flagi.\+hh@{flagi.\+hh}}\index{flagi.\+hh@{flagi.\+hh}!e\+S\+T\+O\+P@{e\+S\+T\+O\+P}}\item[{\em 
\hypertarget{flagi_8hh_a06fc87d81c62e9abb8790b6e5713c55ba4957581ee0386c284fd318121e335af6}{}e\+S\+T\+O\+P\label{flagi_8hh_a06fc87d81c62e9abb8790b6e5713c55ba4957581ee0386c284fd318121e335af6}
}]\index{e\+P\+A\+U\+S\+E@{e\+P\+A\+U\+S\+E}!flagi.\+hh@{flagi.\+hh}}\index{flagi.\+hh@{flagi.\+hh}!e\+P\+A\+U\+S\+E@{e\+P\+A\+U\+S\+E}}\item[{\em 
\hypertarget{flagi_8hh_a06fc87d81c62e9abb8790b6e5713c55ba6a2f69efc37338427ecd0db296923a79}{}e\+P\+A\+U\+S\+E\label{flagi_8hh_a06fc87d81c62e9abb8790b6e5713c55ba6a2f69efc37338427ecd0db296923a79}
}]\index{e\+P\+L\+A\+Y@{e\+P\+L\+A\+Y}!flagi.\+hh@{flagi.\+hh}}\index{flagi.\+hh@{flagi.\+hh}!e\+P\+L\+A\+Y@{e\+P\+L\+A\+Y}}\item[{\em 
\hypertarget{flagi_8hh_a06fc87d81c62e9abb8790b6e5713c55baecdbae639704e0b7e4b5478734e45b8d}{}e\+P\+L\+A\+Y\label{flagi_8hh_a06fc87d81c62e9abb8790b6e5713c55baecdbae639704e0b7e4b5478734e45b8d}
}]\end{description}
\end{Desc}


Definicja w linii 27 pliku flagi.\+hh.



\subsection{Dokumentacja zmiennych}
\hypertarget{flagi_8hh_a359a95636f17b8e9b7a01389d75b521d}{}\index{flagi.\+hh@{flagi.\+hh}!G\+R\+U\+B\+O\+S\+C@{G\+R\+U\+B\+O\+S\+C}}
\index{G\+R\+U\+B\+O\+S\+C@{G\+R\+U\+B\+O\+S\+C}!flagi.\+hh@{flagi.\+hh}}
\subsubsection[{G\+R\+U\+B\+O\+S\+C}]{\setlength{\rightskip}{0pt plus 5cm}const int G\+R\+U\+B\+O\+S\+C = 8}\label{flagi_8hh_a359a95636f17b8e9b7a01389d75b521d}
Grubosc krawedzi zbiornika \mbox{[}px\mbox{]}. Grubosc jest symetryczna wzgledem osi. 

Definicja w linii 59 pliku flagi.\+hh.

\hypertarget{flagi_8hh_afa380d01dc08ee237b4eea9046704397}{}\index{flagi.\+hh@{flagi.\+hh}!P\+A\+S\+K\+I@{P\+A\+S\+K\+I}}
\index{P\+A\+S\+K\+I@{P\+A\+S\+K\+I}!flagi.\+hh@{flagi.\+hh}}
\subsubsection[{P\+A\+S\+K\+I}]{\setlength{\rightskip}{0pt plus 5cm}const int P\+A\+S\+K\+I = 8}\label{flagi_8hh_afa380d01dc08ee237b4eea9046704397}
Suma grubosci paskow statusu itd. okienka \mbox{[}px\mbox{]}. Potrzebne do liczenia pozycji. Jest zależna od komputera. 

Definicja w linii 68 pliku flagi.\+hh.

\hypertarget{flagi_8hh_acd3c5814c051e565bf7854f6403acf49}{}\index{flagi.\+hh@{flagi.\+hh}!P\+O\+D\+S\+T\+A\+W\+A@{P\+O\+D\+S\+T\+A\+W\+A}}
\index{P\+O\+D\+S\+T\+A\+W\+A@{P\+O\+D\+S\+T\+A\+W\+A}!flagi.\+hh@{flagi.\+hh}}
\subsubsection[{P\+O\+D\+S\+T\+A\+W\+A}]{\setlength{\rightskip}{0pt plus 5cm}const int P\+O\+D\+S\+T\+A\+W\+A = 280}\label{flagi_8hh_acd3c5814c051e565bf7854f6403acf49}
Dlugosc podstawy zbiornika \mbox{[}px\mbox{]}. 

Definicja w linii 44 pliku flagi.\+hh.

\hypertarget{flagi_8hh_aa77f856f3142a9e81752665a9bc2e6de}{}\index{flagi.\+hh@{flagi.\+hh}!P\+R\+O\+M\+I\+E\+N@{P\+R\+O\+M\+I\+E\+N}}
\index{P\+R\+O\+M\+I\+E\+N@{P\+R\+O\+M\+I\+E\+N}!flagi.\+hh@{flagi.\+hh}}
\subsubsection[{P\+R\+O\+M\+I\+E\+N}]{\setlength{\rightskip}{0pt plus 5cm}const int P\+R\+O\+M\+I\+E\+N = 8}\label{flagi_8hh_aa77f856f3142a9e81752665a9bc2e6de}
Promien czasteczki \mbox{[}px\mbox{]}. 

Definicja w linii 37 pliku flagi.\+hh.

\hypertarget{flagi_8hh_ae3a120c63186a17e4127a68187b3e9e8}{}\index{flagi.\+hh@{flagi.\+hh}!S\+T\+A\+N@{S\+T\+A\+N}}
\index{S\+T\+A\+N@{S\+T\+A\+N}!flagi.\+hh@{flagi.\+hh}}
\subsubsection[{S\+T\+A\+N}]{\setlength{\rightskip}{0pt plus 5cm}int S\+T\+A\+N}\label{flagi_8hh_ae3a120c63186a17e4127a68187b3e9e8}
Stan symulacji to Play/\+Pause/\+Stop. Inicjalizacja w pliku \hyperlink{zbiornik_8cpp}{zbiornik.\+cpp}. Wybor poczatkowego stanu w konstruktorze okienka.

Stan symulacji.

Poczatkowy stan symulacji to Play/\+Pauza/\+Stop. 

Definicja w linii 21 pliku zbiornik.\+cpp.

\hypertarget{flagi_8hh_a073767f0ac7dbf009a42b00de1092b52}{}\index{flagi.\+hh@{flagi.\+hh}!W\+Y\+S\+O\+K\+O\+S\+C@{W\+Y\+S\+O\+K\+O\+S\+C}}
\index{W\+Y\+S\+O\+K\+O\+S\+C@{W\+Y\+S\+O\+K\+O\+S\+C}!flagi.\+hh@{flagi.\+hh}}
\subsubsection[{W\+Y\+S\+O\+K\+O\+S\+C}]{\setlength{\rightskip}{0pt plus 5cm}const int W\+Y\+S\+O\+K\+O\+S\+C = 280}\label{flagi_8hh_a073767f0ac7dbf009a42b00de1092b52}
Wysokosc zbiornika \mbox{[}px\mbox{]}. 

Definicja w linii 51 pliku flagi.\+hh.


\hypertarget{kolor_8hh}{\section{Dokumentacja pliku kolor.\-hh}
\label{kolor_8hh}\index{kolor.\-hh@{kolor.\-hh}}
}


Zawiera definicje klasy \hyperlink{class_kolor}{Kolor} oraz deklaracje jej metod.  


Ten wykres pokazuje, które pliki bezpośrednio lub pośrednio załączają ten plik\-:\nopagebreak
\begin{figure}[H]
\begin{center}
\leavevmode
\includegraphics[width=239pt]{kolor_8hh__dep__incl}
\end{center}
\end{figure}
\subsection*{Komponenty}
\begin{DoxyCompactItemize}
\item 
class \hyperlink{class_kolor}{Kolor}
\begin{DoxyCompactList}\small\item\em Klasa modelująca kolor. \end{DoxyCompactList}\end{DoxyCompactItemize}


\subsection{Opis szczegółowy}
W pliku znajduja sie\-:
\begin{DoxyItemize}
\item definicja klasy \hyperlink{class_kolor}{Kolor} (modeluje pojecie Koloru),
\item definicje konstruktorow,
\item deklaracje atrybutow. 
\end{DoxyItemize}

Definicja w pliku \hyperlink{kolor_8hh_source}{kolor.\-hh}.


\hypertarget{main_8cpp}{}\section{Dokumentacja pliku main.\+cpp}
\label{main_8cpp}\index{main.\+cpp@{main.\+cpp}}


Zawiera ogolna strukture funkcji main.  


{\ttfamily \#include $<$Q\+Application$>$}\\*
{\ttfamily \#include $<$stdlib.\+h$>$}\\*
{\ttfamily \#include $<$time.\+h$>$}\\*
{\ttfamily \#include \char`\"{}okno\+\_\+glowne.\+hh\char`\"{}}\\*
Wykres zależności załączania dla main.\+cpp\+:\nopagebreak
\begin{figure}[H]
\begin{center}
\leavevmode
\includegraphics[width=350pt]{main_8cpp__incl}
\end{center}
\end{figure}
\subsection*{Funkcje}
\begin{DoxyCompactItemize}
\item 
int \hyperlink{main_8cpp_a0ddf1224851353fc92bfbff6f499fa97}{main} (int argc, char $\ast$argv\mbox{[}$\,$\mbox{]})
\end{DoxyCompactItemize}


\subsection{Opis szczegółowy}
W funkcji main wyroznia sie\+:
\begin{DoxyItemize}
\item utworznie obiektu klasy Q\+Application i wstepne przetworzenie argumentow z linii wywolania,
\item utworzenie okna aplikacji,
\item wymuszenie ukazania sie okna,
\item uruchomienie obslugi petli zdarzen dla calej aplikacji. 
\end{DoxyItemize}

\subsection{Dokumentacja funkcji}
\hypertarget{main_8cpp_a0ddf1224851353fc92bfbff6f499fa97}{}\index{main.\+cpp@{main.\+cpp}!main@{main}}
\index{main@{main}!main.\+cpp@{main.\+cpp}}
\subsubsection[{main}]{\setlength{\rightskip}{0pt plus 5cm}int main (
\begin{DoxyParamCaption}
\item[{int}]{argc, }
\item[{char $\ast$}]{argv\mbox{[}$\,$\mbox{]}}
\end{DoxyParamCaption}
)}\label{main_8cpp_a0ddf1224851353fc92bfbff6f499fa97}


Definicja w linii 19 pliku main.\+cpp.


\hypertarget{moc__dmainwindow_8cpp}{}\section{Dokumentacja pliku moc\+\_\+dmainwindow.\+cpp}
\label{moc__dmainwindow_8cpp}\index{moc\+\_\+dmainwindow.\+cpp@{moc\+\_\+dmainwindow.\+cpp}}
{\ttfamily \#include \char`\"{}../inc/dmainwindow.\+h\char`\"{}}\\*
Wykres zależności załączania dla moc\+\_\+dmainwindow.\+cpp\+:
\nopagebreak
\begin{figure}[H]
\begin{center}
\leavevmode
\includegraphics[width=205pt]{moc__dmainwindow_8cpp__incl}
\end{center}
\end{figure}

\hypertarget{moc__okno__glowne_8cpp}{}\section{Dokumentacja pliku moc\+\_\+okno\+\_\+glowne.\+cpp}
\label{moc__okno__glowne_8cpp}\index{moc\+\_\+okno\+\_\+glowne.\+cpp@{moc\+\_\+okno\+\_\+glowne.\+cpp}}
{\ttfamily \#include \char`\"{}../inc/okno\+\_\+glowne.\+hh\char`\"{}}\\*
Wykres zależności załączania dla moc\+\_\+okno\+\_\+glowne.\+cpp\+:\nopagebreak
\begin{figure}[H]
\begin{center}
\leavevmode
\includegraphics[width=350pt]{moc__okno__glowne_8cpp__incl}
\end{center}
\end{figure}

\hypertarget{moc__zbiornik_8cpp}{\section{Dokumentacja pliku moc\-\_\-zbiornik.\-cpp}
\label{moc__zbiornik_8cpp}\index{moc\-\_\-zbiornik.\-cpp@{moc\-\_\-zbiornik.\-cpp}}
}
{\ttfamily \#include \char`\"{}../inc/zbiornik.\-hh\char`\"{}}\\*
Wykres zależności załączania dla moc\-\_\-zbiornik.\-cpp\-:
\nopagebreak
\begin{figure}[H]
\begin{center}
\leavevmode
\includegraphics[width=350pt]{moc__zbiornik_8cpp__incl}
\end{center}
\end{figure}

\hypertarget{okno__glowne_8cpp}{\section{Dokumentacja pliku okno\-\_\-glowne.\-cpp}
\label{okno__glowne_8cpp}\index{okno\-\_\-glowne.\-cpp@{okno\-\_\-glowne.\-cpp}}
}


Zawiera definicje metod klasy \hyperlink{class_okno_glowne}{Okno\-Glowne}.  


{\ttfamily \#include \char`\"{}okno\-\_\-glowne.\-hh\char`\"{}}\\*
Wykres zależności załączania dla okno\-\_\-glowne.\-cpp\-:\nopagebreak
\begin{figure}[H]
\begin{center}
\leavevmode
\includegraphics[width=350pt]{okno__glowne_8cpp__incl}
\end{center}
\end{figure}


\subsection{Opis szczegółowy}
W pliku znajduja sie\-:
\begin{DoxyItemize}
\item definicje konstruktorow oraz metod klasy \hyperlink{class_okno_glowne}{Okno\-Glowne}. 
\end{DoxyItemize}

Definicja w pliku \hyperlink{okno__glowne_8cpp_source}{okno\-\_\-glowne.\-cpp}.


\hypertarget{okno__glowne_8hh}{\section{Dokumentacja pliku okno\-\_\-glowne.\-hh}
\label{okno__glowne_8hh}\index{okno\-\_\-glowne.\-hh@{okno\-\_\-glowne.\-hh}}
}


Zawiera definicje klasy \hyperlink{class_okno_glowne}{Okno\-Glowne} i deklaracje jej metod.  


{\ttfamily \#include $<$Q\-Variant$>$}\\*
{\ttfamily \#include $<$Q\-Action$>$}\\*
{\ttfamily \#include $<$Q\-Application$>$}\\*
{\ttfamily \#include $<$Q\-Button\-Group$>$}\\*
{\ttfamily \#include $<$Q\-H\-Box\-Layout$>$}\\*
{\ttfamily \#include $<$Q\-Header\-View$>$}\\*
{\ttfamily \#include $<$Q\-Main\-Window$>$}\\*
{\ttfamily \#include $<$Q\-Menu$>$}\\*
{\ttfamily \#include $<$Q\-Menu\-Bar$>$}\\*
{\ttfamily \#include $<$Q\-Push\-Button$>$}\\*
{\ttfamily \#include $<$Q\-Status\-Bar$>$}\\*
{\ttfamily \#include $<$Q\-Tool\-Bar$>$}\\*
{\ttfamily \#include $<$Q\-Widget$>$}\\*
{\ttfamily \#include $<$Q\-Color$>$}\\*
{\ttfamily \#include $<$Q\-Painter$>$}\\*
{\ttfamily \#include $<$Q\-Time$>$}\\*
{\ttfamily \#include $<$Q\-Date$>$}\\*
{\ttfamily \#include $<$Q\-Locale$>$}\\*
{\ttfamily \#include $<$Q\-Slider$>$}\\*
{\ttfamily \#include $<$Q\-Spin\-Box$>$}\\*
{\ttfamily \#include $<$Q\-Progress\-Bar$>$}\\*
{\ttfamily \#include $<$Q\-L\-C\-D\-Number$>$}\\*
{\ttfamily \#include $<$iostream$>$}\\*
{\ttfamily \#include $<$sstream$>$}\\*
{\ttfamily \#include $<$list$>$}\\*
{\ttfamily \#include \char`\"{}kolor.\-hh\char`\"{}}\\*
{\ttfamily \#include \char`\"{}czasteczka.\-hh\char`\"{}}\\*
{\ttfamily \#include \char`\"{}zbiornik.\-hh\char`\"{}}\\*
{\ttfamily \#include \char`\"{}dmainwindow.\-h\char`\"{}}\\*
Wykres zależności załączania dla okno\-\_\-glowne.\-hh\-:
\nopagebreak
\begin{figure}[H]
\begin{center}
\leavevmode
\includegraphics[width=350pt]{okno__glowne_8hh__incl}
\end{center}
\end{figure}
Ten wykres pokazuje, które pliki bezpośrednio lub pośrednio załączają ten plik\-:
\nopagebreak
\begin{figure}[H]
\begin{center}
\leavevmode
\includegraphics[width=350pt]{okno__glowne_8hh__dep__incl}
\end{center}
\end{figure}
\subsection*{Komponenty}
\begin{DoxyCompactItemize}
\item 
class \hyperlink{class_okno_glowne}{Okno\-Glowne}
\begin{DoxyCompactList}\small\item\em Klasa modelujaca głowne okno aplikacji. \end{DoxyCompactList}\end{DoxyCompactItemize}


\subsection{Opis szczegółowy}
W pliku znajduja sie\-:
\begin{DoxyItemize}
\item definicja klasy \hyperlink{class_okno_glowne}{Okno\-Glowne} (modeluje glowne okno aplikacji),
\item deklaracje konstruktorow, metod i przeciazen ww. klasy. 
\end{DoxyItemize}

Definicja w pliku \hyperlink{okno__glowne_8hh_source}{okno\-\_\-glowne.\-hh}.


\hypertarget{simulation_8cpp}{\section{Dokumentacja pliku simulation.\-cpp}
\label{simulation_8cpp}\index{simulation.\-cpp@{simulation.\-cpp}}
}


Plik z kodem symulatora cieczy.  


{\ttfamily \#include $<$iostream$>$}\\*
{\ttfamily \#include $<$iomanip$>$}\\*
{\ttfamily \#include $<$cstring$>$}\\*
{\ttfamily \#include $<$cmath$>$}\\*
{\ttfamily \#include $<$cassert$>$}\\*
Wykres zależności załączania dla simulation.\-cpp\-:\nopagebreak
\begin{figure}[H]
\begin{center}
\leavevmode
\includegraphics[width=350pt]{simulation_8cpp__incl}
\end{center}
\end{figure}
\subsection*{Komponenty}
\begin{DoxyCompactItemize}
\item 
class \hyperlink{class_vector}{Vector}
\begin{DoxyCompactList}\small\item\em klasa \hyperlink{class_vector}{Vector} \end{DoxyCompactList}\item 
struct \hyperlink{structparams__t}{params\-\_\-t}
\item 
class \hyperlink{classsimulation}{simulation}
\end{DoxyCompactItemize}
\subsection*{Definicje}
\begin{DoxyCompactItemize}
\item 
\#define \hyperlink{simulation_8cpp_a484c60de5b8ebd1bbce3ca8a60074186}{D\-A\-M\-P}~0.\-75
\item 
\#define \hyperlink{simulation_8cpp_a0312cb6d6cbc719075d4e5380c387ab3}{X\-M\-A\-X}~2.\-0
\item 
\#define \hyperlink{simulation_8cpp_a610d6ad95b18966b70b6845de2a9c56b}{Y\-M\-A\-X}~2.\-0
\item 
\#define \hyperlink{simulation_8cpp_acbd91a8a9a62657e2252a0a1f7c876e1}{L\-O\-G}(M\-S\-G)
\begin{DoxyCompactList}\small\item\em makro do logowania \end{DoxyCompactList}\end{DoxyCompactItemize}
\subsection*{Funkcje}
\begin{DoxyCompactItemize}
\item 
std\-::ostream \& \hyperlink{simulation_8cpp_a7973d0032b0df4c3ab89e1da167b86a9}{operator$<$$<$} (std\-::ostream \&\-\_\-os, const \hyperlink{class_vector}{Vector} \&\-\_\-s)
\item 
void \hyperlink{simulation_8cpp_afc3040ee6fc2feb310dd23f1d2c0d165}{setup} (\hyperlink{structparams__t}{params\-\_\-t} $\ast$params)
\item 
std\-::ostream \& \hyperlink{simulation_8cpp_a1f6414b078a2823f5cea77fc1235e1d9}{operator$<$$<$} (std\-::ostream \&\-\_\-os, const \hyperlink{classsimulation}{simulation} \&\-\_\-s)
\item 
int \hyperlink{simulation_8cpp_a6423366f08654ee809bc247f57042f18}{funkcja\-\_\-main} ()
\end{DoxyCompactItemize}


\subsection{Opis szczegółowy}
Zawiera definicje klas i funkcji pozwalających na przeprowadzenie symulacji zachowania sparametryzowanego modelu cieczy w zadanym środowisku od określonego warunku początkowego. 

Definicja w pliku \hyperlink{simulation_8cpp_source}{simulation.\-cpp}.



\subsection{Dokumentacja definicji}
\hypertarget{simulation_8cpp_a484c60de5b8ebd1bbce3ca8a60074186}{\index{simulation.\-cpp@{simulation.\-cpp}!D\-A\-M\-P@{D\-A\-M\-P}}
\index{D\-A\-M\-P@{D\-A\-M\-P}!simulation.cpp@{simulation.\-cpp}}
\subsubsection[{D\-A\-M\-P}]{\setlength{\rightskip}{0pt plus 5cm}\#define D\-A\-M\-P~0.\-75}}\label{simulation_8cpp_a484c60de5b8ebd1bbce3ca8a60074186}


Definicja w linii 6 pliku simulation.\-cpp.

\hypertarget{simulation_8cpp_acbd91a8a9a62657e2252a0a1f7c876e1}{\index{simulation.\-cpp@{simulation.\-cpp}!L\-O\-G@{L\-O\-G}}
\index{L\-O\-G@{L\-O\-G}!simulation.cpp@{simulation.\-cpp}}
\subsubsection[{L\-O\-G}]{\setlength{\rightskip}{0pt plus 5cm}\#define L\-O\-G(
\begin{DoxyParamCaption}
\item[{}]{M\-S\-G}
\end{DoxyParamCaption}
)}}\label{simulation_8cpp_acbd91a8a9a62657e2252a0a1f7c876e1}
{\bfseries Wartość\-:}
\begin{DoxyCode}
\textcolor{keywordflow}{do} \{ std::cerr << \textcolor{stringliteral}{"\(\backslash\)e[32m"} << \_\_FILE\_\_ << \textcolor{stringliteral}{"\(\backslash\)e[31m:\(\backslash\)e[33m"} << \_\_LINE\_\_ \(\backslash\)
    << \textcolor{stringliteral}{"\(\backslash\)e[35m:\(\backslash\)e[34m"} << std::setw(16) << \_\_FUNCTION\_\_ << \textcolor{stringliteral}{"\(\backslash\)e[36m : \(\backslash\)e[37m"} << MSG << \textcolor{stringliteral}{"\(\backslash\)e[0m\(\backslash\)n"}; \} \textcolor{keywordflow}{while}( \textcolor{keyword}{
      false} )
\end{DoxyCode}
Łatwe w użyciu makro poprawiające czytelność komunikatów. Pętla do...while zastosowana aby wymusić poprawne użycie średnika po poleceniu. 
\begin{DoxyParams}[1]{Parametry}
\mbox{\tt in}  & {\em M\-S\-G} & -\/ wiadomość do wyświetlenia \\
\hline
\end{DoxyParams}


Definicja w linii 18 pliku simulation.\-cpp.

\hypertarget{simulation_8cpp_a0312cb6d6cbc719075d4e5380c387ab3}{\index{simulation.\-cpp@{simulation.\-cpp}!X\-M\-A\-X@{X\-M\-A\-X}}
\index{X\-M\-A\-X@{X\-M\-A\-X}!simulation.cpp@{simulation.\-cpp}}
\subsubsection[{X\-M\-A\-X}]{\setlength{\rightskip}{0pt plus 5cm}\#define X\-M\-A\-X~2.\-0}}\label{simulation_8cpp_a0312cb6d6cbc719075d4e5380c387ab3}


Definicja w linii 7 pliku simulation.\-cpp.

\hypertarget{simulation_8cpp_a610d6ad95b18966b70b6845de2a9c56b}{\index{simulation.\-cpp@{simulation.\-cpp}!Y\-M\-A\-X@{Y\-M\-A\-X}}
\index{Y\-M\-A\-X@{Y\-M\-A\-X}!simulation.cpp@{simulation.\-cpp}}
\subsubsection[{Y\-M\-A\-X}]{\setlength{\rightskip}{0pt plus 5cm}\#define Y\-M\-A\-X~2.\-0}}\label{simulation_8cpp_a610d6ad95b18966b70b6845de2a9c56b}


Definicja w linii 8 pliku simulation.\-cpp.



\subsection{Dokumentacja funkcji}
\hypertarget{simulation_8cpp_a6423366f08654ee809bc247f57042f18}{\index{simulation.\-cpp@{simulation.\-cpp}!funkcja\-\_\-main@{funkcja\-\_\-main}}
\index{funkcja\-\_\-main@{funkcja\-\_\-main}!simulation.cpp@{simulation.\-cpp}}
\subsubsection[{funkcja\-\_\-main}]{\setlength{\rightskip}{0pt plus 5cm}int funkcja\-\_\-main (
\begin{DoxyParamCaption}
{}
\end{DoxyParamCaption}
)}}\label{simulation_8cpp_a6423366f08654ee809bc247f57042f18}


Definicja w linii 369 pliku simulation.\-cpp.



Oto graf wywołań dla tej funkcji\-:\nopagebreak
\begin{figure}[H]
\begin{center}
\leavevmode
\includegraphics[width=350pt]{simulation_8cpp_a6423366f08654ee809bc247f57042f18_cgraph}
\end{center}
\end{figure}


\hypertarget{simulation_8cpp_a7973d0032b0df4c3ab89e1da167b86a9}{\index{simulation.\-cpp@{simulation.\-cpp}!operator$<$$<$@{operator$<$$<$}}
\index{operator$<$$<$@{operator$<$$<$}!simulation.cpp@{simulation.\-cpp}}
\subsubsection[{operator$<$$<$}]{\setlength{\rightskip}{0pt plus 5cm}std\-::ostream\& operator$<$$<$ (
\begin{DoxyParamCaption}
\item[{std\-::ostream \&}]{\-\_\-os, }
\item[{const {\bf Vector} \&}]{\-\_\-s}
\end{DoxyParamCaption}
)}}\label{simulation_8cpp_a7973d0032b0df4c3ab89e1da167b86a9}


Definicja w linii 133 pliku simulation.\-cpp.

\hypertarget{simulation_8cpp_a1f6414b078a2823f5cea77fc1235e1d9}{\index{simulation.\-cpp@{simulation.\-cpp}!operator$<$$<$@{operator$<$$<$}}
\index{operator$<$$<$@{operator$<$$<$}!simulation.cpp@{simulation.\-cpp}}
\subsubsection[{operator$<$$<$}]{\setlength{\rightskip}{0pt plus 5cm}std\-::ostream\& operator$<$$<$ (
\begin{DoxyParamCaption}
\item[{std\-::ostream \&}]{\-\_\-os, }
\item[{const {\bf simulation} \&}]{\-\_\-s}
\end{DoxyParamCaption}
)}}\label{simulation_8cpp_a1f6414b078a2823f5cea77fc1235e1d9}


Definicja w linii 360 pliku simulation.\-cpp.

\hypertarget{simulation_8cpp_afc3040ee6fc2feb310dd23f1d2c0d165}{\index{simulation.\-cpp@{simulation.\-cpp}!setup@{setup}}
\index{setup@{setup}!simulation.cpp@{simulation.\-cpp}}
\subsubsection[{setup}]{\setlength{\rightskip}{0pt plus 5cm}void setup (
\begin{DoxyParamCaption}
\item[{{\bf params\-\_\-t} $\ast$}]{params}
\end{DoxyParamCaption}
)}}\label{simulation_8cpp_afc3040ee6fc2feb310dd23f1d2c0d165}


Definicja w linii 150 pliku simulation.\-cpp.



Oto graf wywoływań tej funkcji\-:\nopagebreak
\begin{figure}[H]
\begin{center}
\leavevmode
\includegraphics[width=232pt]{simulation_8cpp_afc3040ee6fc2feb310dd23f1d2c0d165_icgraph}
\end{center}
\end{figure}



\input{strona_8dox}
\hypertarget{ui__dmainwindow_8h}{}\section{Dokumentacja pliku ui\+\_\+dmainwindow.\+h}
\label{ui__dmainwindow_8h}\index{ui\+\_\+dmainwindow.\+h@{ui\+\_\+dmainwindow.\+h}}
{\ttfamily \#include $<$Qt\+Core/\+Q\+Variant$>$}\\*
{\ttfamily \#include $<$Qt\+Gui/\+Q\+Action$>$}\\*
{\ttfamily \#include $<$Qt\+Gui/\+Q\+Application$>$}\\*
{\ttfamily \#include $<$Qt\+Gui/\+Q\+Button\+Group$>$}\\*
{\ttfamily \#include $<$Qt\+Gui/\+Q\+H\+Box\+Layout$>$}\\*
{\ttfamily \#include $<$Qt\+Gui/\+Q\+Header\+View$>$}\\*
{\ttfamily \#include $<$Qt\+Gui/\+Q\+L\+C\+D\+Number$>$}\\*
{\ttfamily \#include $<$Qt\+Gui/\+Q\+Line\+Edit$>$}\\*
{\ttfamily \#include $<$Qt\+Gui/\+Q\+Main\+Window$>$}\\*
{\ttfamily \#include $<$Qt\+Gui/\+Q\+Menu$>$}\\*
{\ttfamily \#include $<$Qt\+Gui/\+Q\+Menu\+Bar$>$}\\*
{\ttfamily \#include $<$Qt\+Gui/\+Q\+Push\+Button$>$}\\*
{\ttfamily \#include $<$Qt\+Gui/\+Q\+Slider$>$}\\*
{\ttfamily \#include $<$Qt\+Gui/\+Q\+Status\+Bar$>$}\\*
{\ttfamily \#include $<$Qt\+Gui/\+Q\+Tool\+Bar$>$}\\*
{\ttfamily \#include $<$Qt\+Gui/\+Q\+Widget$>$}\\*
Wykres zależności załączania dla ui\+\_\+dmainwindow.\+h\+:
\nopagebreak
\begin{figure}[H]
\begin{center}
\leavevmode
\includegraphics[width=350pt]{ui__dmainwindow_8h__incl}
\end{center}
\end{figure}
Ten wykres pokazuje, które pliki bezpośrednio lub pośrednio załączają ten plik\+:
\nopagebreak
\begin{figure}[H]
\begin{center}
\leavevmode
\includegraphics[width=182pt]{ui__dmainwindow_8h__dep__incl}
\end{center}
\end{figure}
\subsection*{Komponenty}
\begin{DoxyCompactItemize}
\item 
class \hyperlink{class_ui___d_main_window}{Ui\+\_\+\+D\+Main\+Window}
\item 
class \hyperlink{class_ui_1_1_d_main_window}{Ui\+::\+D\+Main\+Window}
\end{DoxyCompactItemize}
\subsection*{Przestrzenie nazw}
\begin{DoxyCompactItemize}
\item 
 \hyperlink{namespace_ui}{Ui}
\end{DoxyCompactItemize}

\hypertarget{vector_8hh}{\section{Dokumentacja pliku vector.\-hh}
\label{vector_8hh}\index{vector.\-hh@{vector.\-hh}}
}


Klasa przechowująca wektor Zawiera podstawowe metody pozwalające stworzyć odpowiedni obiekt, odczytywać i zapisywać dane do niego, wczytać i wydrukować na standardowych strumieniach, a także odjąć od niego inny wektor.  


{\ttfamily \#include $<$iostream$>$}\\*
Wykres zależności załączania dla vector.\-hh\-:
\nopagebreak
\begin{figure}[H]
\begin{center}
\leavevmode
\includegraphics[width=136pt]{vector_8hh__incl}
\end{center}
\end{figure}
\subsection*{Komponenty}
\begin{DoxyCompactItemize}
\item 
class \hyperlink{class_vector}{Vector$<$ T\-Y\-P\-E $>$}
\item 
class \hyperlink{class_vector}{Vector$<$ T\-Y\-P\-E $>$}
\end{DoxyCompactItemize}
\subsection*{Funkcje}
\begin{DoxyCompactItemize}
\item 
{\footnotesize template$<$typename T\-Y\-P\-E $>$ }\\std\-::istream \& \hyperlink{vector_8hh_af9f1d76a42b7800e17a6da47fce9e179}{operator$>$$>$} (std\-::istream \&, \hyperlink{class_vector}{Vector}$<$ T\-Y\-P\-E $>$ \&)
\item 
{\footnotesize template$<$typename T\-Y\-P\-E $>$ }\\std\-::ostream \& \hyperlink{vector_8hh_aebb7c04457be6d2378fa43ce22cfb408}{operator$<$$<$} (std\-::ostream \&, const \hyperlink{class_vector}{Vector}$<$ T\-Y\-P\-E $>$ \&)
\end{DoxyCompactItemize}


\subsection{Dokumentacja funkcji}
\hypertarget{vector_8hh_aebb7c04457be6d2378fa43ce22cfb408}{\index{vector.\-hh@{vector.\-hh}!operator$<$$<$@{operator$<$$<$}}
\index{operator$<$$<$@{operator$<$$<$}!vector.hh@{vector.\-hh}}
\subsubsection[{operator$<$$<$}]{\setlength{\rightskip}{0pt plus 5cm}template$<$typename T\-Y\-P\-E $>$ std\-::ostream\& operator$<$$<$ (
\begin{DoxyParamCaption}
\item[{std\-::ostream \&}]{, }
\item[{const {\bf Vector}$<$ T\-Y\-P\-E $>$ \&}]{}
\end{DoxyParamCaption}
)}}\label{vector_8hh_aebb7c04457be6d2378fa43ce22cfb408}

\begin{DoxyParams}[1]{Parametry}
\mbox{\tt in,out}  & {\em out} & -\/ strumień wyjściowy \\
\hline
\mbox{\tt in}  & {\em vect} & -\/ wektor \\
\hline
\end{DoxyParams}
\begin{DoxyReturn}{Zwraca}
Zwraca referencję do zmodyfikowanego strumienia. 
\end{DoxyReturn}
\hypertarget{vector_8hh_af9f1d76a42b7800e17a6da47fce9e179}{\index{vector.\-hh@{vector.\-hh}!operator$>$$>$@{operator$>$$>$}}
\index{operator$>$$>$@{operator$>$$>$}!vector.hh@{vector.\-hh}}
\subsubsection[{operator$>$$>$}]{\setlength{\rightskip}{0pt plus 5cm}template$<$typename T\-Y\-P\-E $>$ std\-::istream\& operator$>$$>$ (
\begin{DoxyParamCaption}
\item[{std\-::istream \&}]{, }
\item[{{\bf Vector}$<$ T\-Y\-P\-E $>$ \&}]{}
\end{DoxyParamCaption}
)}}\label{vector_8hh_af9f1d76a42b7800e17a6da47fce9e179}

\begin{DoxyParams}[1]{Parametry}
\mbox{\tt in,out}  & {\em inp} & -\/ strumień wejściowy \\
\hline
\mbox{\tt out}  & {\em vect} & -\/ wektor \\
\hline
\end{DoxyParams}
\begin{DoxyReturn}{Zwraca}
Zwraca referencję do zmodyfikowanego strumienia. 
\end{DoxyReturn}

\hypertarget{zbiornik_8cpp}{\section{Dokumentacja pliku zbiornik.\-cpp}
\label{zbiornik_8cpp}\index{zbiornik.\-cpp@{zbiornik.\-cpp}}
}


Zawiera definicje metod klasy \hyperlink{class_zbiornik}{Zbiornik}.  


{\ttfamily \#include \char`\"{}zbiornik.\-hh\char`\"{}}\\*
Wykres zależności załączania dla zbiornik.\-cpp\-:
\nopagebreak
\begin{figure}[H]
\begin{center}
\leavevmode
\includegraphics[width=350pt]{zbiornik_8cpp__incl}
\end{center}
\end{figure}


\subsection{Opis szczegółowy}
W pliku znajduja sie\-:
\begin{DoxyItemize}
\item definicje konstruktorow, metod i przeciazen klasy \hyperlink{class_zbiornik}{Zbiornik}. 
\end{DoxyItemize}

Definicja w pliku \hyperlink{zbiornik_8cpp_source}{zbiornik.\-cpp}.


\hypertarget{zbiornik_8hh}{\section{Dokumentacja pliku zbiornik.\-hh}
\label{zbiornik_8hh}\index{zbiornik.\-hh@{zbiornik.\-hh}}
}


Zawiera definicje klasy \hyperlink{class_zbiornik}{Zbiornik} i deklaracje jej metod.  


{\ttfamily \#include $<$Q\-Application$>$}\\*
{\ttfamily \#include $<$Q\-Color$>$}\\*
{\ttfamily \#include $<$Q\-Painter$>$}\\*
{\ttfamily \#include $<$Q\-Status\-Bar$>$}\\*
{\ttfamily \#include $<$Q\-Time$>$}\\*
{\ttfamily \#include $<$Q\-Timer$>$}\\*
{\ttfamily \#include $<$Q\-Date$>$}\\*
{\ttfamily \#include $<$Q\-Locale$>$}\\*
{\ttfamily \#include $<$Q\-Widget$>$}\\*
{\ttfamily \#include $<$Q\-Slider$>$}\\*
{\ttfamily \#include $<$Q\-Main\-Window$>$}\\*
{\ttfamily \#include $<$Q\-Spin\-Box$>$}\\*
{\ttfamily \#include $<$Q\-Progress\-Bar$>$}\\*
{\ttfamily \#include $<$Q\-L\-C\-D\-Number$>$}\\*
{\ttfamily \#include $<$Q\-Push\-Button$>$}\\*
{\ttfamily \#include $<$iostream$>$}\\*
{\ttfamily \#include $<$sstream$>$}\\*
{\ttfamily \#include \char`\"{}czasteczka.\-hh\char`\"{}}\\*
Wykres zależności załączania dla zbiornik.\-hh\-:
\nopagebreak
\begin{figure}[H]
\begin{center}
\leavevmode
\includegraphics[width=350pt]{zbiornik_8hh__incl}
\end{center}
\end{figure}
Ten wykres pokazuje, które pliki bezpośrednio lub pośrednio załączają ten plik\-:
\nopagebreak
\begin{figure}[H]
\begin{center}
\leavevmode
\includegraphics[width=350pt]{zbiornik_8hh__dep__incl}
\end{center}
\end{figure}
\subsection*{Komponenty}
\begin{DoxyCompactItemize}
\item 
class \hyperlink{class_zbiornik}{Zbiornik}
\begin{DoxyCompactList}\small\item\em Klasa modelująca zbiornik. \end{DoxyCompactList}\end{DoxyCompactItemize}
\subsection*{Zmienne}
\begin{DoxyCompactItemize}
\item 
const int \hyperlink{zbiornik_8hh_acd3c5814c051e565bf7854f6403acf49}{P\-O\-D\-S\-T\-A\-W\-A} = 200
\begin{DoxyCompactList}\small\item\em Dlugosc podstawy zbiornika. \end{DoxyCompactList}\item 
const int \hyperlink{zbiornik_8hh_a073767f0ac7dbf009a42b00de1092b52}{W\-Y\-S\-O\-K\-O\-S\-C} = 200
\begin{DoxyCompactList}\small\item\em Wysokosc zbiornika. \end{DoxyCompactList}\item 
const int \hyperlink{zbiornik_8hh_a359a95636f17b8e9b7a01389d75b521d}{G\-R\-U\-B\-O\-S\-C} = 3
\begin{DoxyCompactList}\small\item\em Grubosc krawedzi zbiornika. \end{DoxyCompactList}\item 
const int \hyperlink{zbiornik_8hh_a3a09b0fc9bed85242f7783147af182ae}{O\-D\-P\-O\-W\-I\-E\-D\-N\-I\-\_\-\-C\-Z\-A\-S} = 100
\begin{DoxyCompactList}\small\item\em Interwal dla timeout'ow z timera \mbox{[}ms\mbox{]}. \end{DoxyCompactList}\end{DoxyCompactItemize}


\subsection{Opis szczegółowy}
W pliku znajduja sie\-:
\begin{DoxyItemize}
\item definicja klasy \hyperlink{class_zbiornik}{Zbiornik} (modeluje pojecie Zbiornika),
\item deklaracje konstruktorow, metod i przeciazen ww. klasy. 
\end{DoxyItemize}

Definicja w pliku \hyperlink{zbiornik_8hh_source}{zbiornik.\-hh}.



\subsection{Dokumentacja zmiennych}
\hypertarget{zbiornik_8hh_a359a95636f17b8e9b7a01389d75b521d}{\index{zbiornik.\-hh@{zbiornik.\-hh}!G\-R\-U\-B\-O\-S\-C@{G\-R\-U\-B\-O\-S\-C}}
\index{G\-R\-U\-B\-O\-S\-C@{G\-R\-U\-B\-O\-S\-C}!zbiornik.hh@{zbiornik.\-hh}}
\subsubsection[{G\-R\-U\-B\-O\-S\-C}]{\setlength{\rightskip}{0pt plus 5cm}const int G\-R\-U\-B\-O\-S\-C = 3}}\label{zbiornik_8hh_a359a95636f17b8e9b7a01389d75b521d}
Grubosc krawedzi zbiornika. 

Definicja w linii 58 pliku zbiornik.\-hh.

\hypertarget{zbiornik_8hh_a3a09b0fc9bed85242f7783147af182ae}{\index{zbiornik.\-hh@{zbiornik.\-hh}!O\-D\-P\-O\-W\-I\-E\-D\-N\-I\-\_\-\-C\-Z\-A\-S@{O\-D\-P\-O\-W\-I\-E\-D\-N\-I\-\_\-\-C\-Z\-A\-S}}
\index{O\-D\-P\-O\-W\-I\-E\-D\-N\-I\-\_\-\-C\-Z\-A\-S@{O\-D\-P\-O\-W\-I\-E\-D\-N\-I\-\_\-\-C\-Z\-A\-S}!zbiornik.hh@{zbiornik.\-hh}}
\subsubsection[{O\-D\-P\-O\-W\-I\-E\-D\-N\-I\-\_\-\-C\-Z\-A\-S}]{\setlength{\rightskip}{0pt plus 5cm}const int O\-D\-P\-O\-W\-I\-E\-D\-N\-I\-\_\-\-C\-Z\-A\-S = 100}}\label{zbiornik_8hh_a3a09b0fc9bed85242f7783147af182ae}
Interwal dla timeout'ow z timera \mbox{[}ms\mbox{]}. 

Definicja w linii 65 pliku zbiornik.\-hh.

\hypertarget{zbiornik_8hh_acd3c5814c051e565bf7854f6403acf49}{\index{zbiornik.\-hh@{zbiornik.\-hh}!P\-O\-D\-S\-T\-A\-W\-A@{P\-O\-D\-S\-T\-A\-W\-A}}
\index{P\-O\-D\-S\-T\-A\-W\-A@{P\-O\-D\-S\-T\-A\-W\-A}!zbiornik.hh@{zbiornik.\-hh}}
\subsubsection[{P\-O\-D\-S\-T\-A\-W\-A}]{\setlength{\rightskip}{0pt plus 5cm}const int P\-O\-D\-S\-T\-A\-W\-A = 200}}\label{zbiornik_8hh_acd3c5814c051e565bf7854f6403acf49}
Dlugosc podstawy zbiornika. 

Definicja w linii 44 pliku zbiornik.\-hh.

\hypertarget{zbiornik_8hh_a073767f0ac7dbf009a42b00de1092b52}{\index{zbiornik.\-hh@{zbiornik.\-hh}!W\-Y\-S\-O\-K\-O\-S\-C@{W\-Y\-S\-O\-K\-O\-S\-C}}
\index{W\-Y\-S\-O\-K\-O\-S\-C@{W\-Y\-S\-O\-K\-O\-S\-C}!zbiornik.hh@{zbiornik.\-hh}}
\subsubsection[{W\-Y\-S\-O\-K\-O\-S\-C}]{\setlength{\rightskip}{0pt plus 5cm}const int W\-Y\-S\-O\-K\-O\-S\-C = 200}}\label{zbiornik_8hh_a073767f0ac7dbf009a42b00de1092b52}
Wysokosc zbiornika. 

Definicja w linii 51 pliku zbiornik.\-hh.


%--- End generated contents ---

% Index
\newpage
\phantomsection
\addcontentsline{toc}{chapter}{Indeks}
\printindex

\end{document}
