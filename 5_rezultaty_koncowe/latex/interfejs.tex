% opis interfejsu graficznego aplikacji

\section{Interfejs graficzny}
Wygląd interfejsu graficznego aplikacji przedstawiony jest na rysunku \ref{fig:gui}.

\begin{figure}
 \begin{center}
  \includegraphics[width=\textwidth]{./rysunki/gui}
 \end{center}
 \caption{Interfejs graficzny aplikacji}
 \label{fig:gui}
\end{figure}
\begin{itemize}
    \item
W centralnej części aplikacji widoczny jest jej główny element, czyli zbiornik z~cząsteczkami. Pod nim znajduje się \textit{slider} pozwalający na przechylanie zbiornika.

    \item
Cząsteczki oraz okienko aplikacji są częściowo przezroczyste, co nadaje mu nowoczesny wygląd.

    \item
Użytkownik za pomocą trzech przycisków umiejscowionych na dole ekranu może sterować symulacją, a mianowicie ją: uruchomić, zamrozić lub zatrzymać.

    \item
W lewej górnej części okienka znajdują się elementy informacyjne pozwalające śledzić parametry symulacji: jej czas trwania, symulowaną liczbę cząsteczek oraz szybkość odświeżania wizualizacji. Na prawo od nich umiejscowione są elementy umożliwiające interakcję użytkownika z aplikacją: \textit{slider} pozwalający na zmianę szybkości symulacji oraz pole typu \textit{checkbox} zmieniające wektor grawitacji.

    \item
Na widocznej w dole ekranu belce statusowej wyświetlana jest aktualna data. W~pasku menu dostępna jest opcja zamknięcia aplikacji.

    \item
Rozkład elementów interfejsu graficznego jest odpowiednio modyfikowany przy zmianie wymiarów okienka.
\end{itemize}
