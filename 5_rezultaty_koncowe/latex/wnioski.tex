%wnioski i podsumowanie  
%TODO
\section{Wnioski}
Pisanie aplikacji przy użyciu biblioteki \textsf{Qt} pozwala na osiągnięcie dobrych efektów przy stosunkowo małym nakładzie pracy. Po zaznajomieniu się z biblioteką oraz aplikacją \textsf{designer} programy można pisać w efektywny sposób. Duża ilość gotowych elementów umożliwia projektowanie aplikacji bez potrzeby implementacji podstawowych składników każdego programu graficznego.

W \textsf{Qt} używany jest mechanizm slotów i sygnałów. Pozwala on na wywoływanie funkcji w odpowiednich momentach jako efekt jakiegoś wydarzenia. Dzięki temu możliwa jest interakcja użytkownika z aplikacją.

Metoda SPH w sposób realistyczny modeluje zachowanie cząsteczek i pozwala na ich sparametryzowanie. Okazała się być ona dobrym wyborem dla komputerowej realizacji symulacji cieczy.