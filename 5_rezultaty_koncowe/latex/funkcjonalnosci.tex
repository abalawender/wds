% opis funkcjonalnosci aplikacji

\section{Funkcjonalności aplikacji}
Najistotniejsze funkcjonalności aplikacji to:
\begin{itemize}
    \item symulacja zachowania modelu cieczy przy obracaniu zbiornikiem,
    \item możliwość wyzerowania pionowej składowej wektora grawitacji,
    \item programowa możliwość przedefiniowania parametrów cieczy (eg. gęstości, lepkości) oraz warunków początkowych,
    \item zmiana szybkości symulacji,
    \item możliwość obserwacji wyniku symulacji (położenia cząsteczek i~rozkładu ciśnień) oraz interakcji.
\end{itemize}

Aplikacja umożliwia zasymulowanie zachowania cieczy od zadanych warunków początkowych. Daje użytkownikowi możliwość interakcji - poruszania zbiornikiem za pomocą \textit{slidera}. Symulacja odzwierciedla zachowanie cząsteczek na Ziemi, a dzięki możliwości wyzerowania pionowej składowej wektora grawitacji, również w~Kosmosie i w warunkach mikrograwitacji. Programowo dostępna jest zmiana wszystkich założonych parametrów cząsteczki cieczy. Wybrano model wody uznając go za najbardziej intuicyjny i atrakcyjny wizualnie w symulacji. Podczas działania aplikacji w~jej okienku obserwować można komputerowo zamodelowany ruch płynu wraz z rozkładem panujących w~nim ciśnień. Symulacja może być zatrzymywana i ponownie uruchamiana, a jej szybkość zmieniana. Rysunek \ref{fig:gui} przedstawia przykładowy zrzut ekranu działającej aplikacji.
